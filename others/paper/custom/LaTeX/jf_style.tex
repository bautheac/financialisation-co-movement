% --------------------------------------------------------------------------------------------
%:JF STYLE COMMANDS START HERE --- START CUTTING FROM HERE

% (1) The line below says asks LaTeX to use the ``article'' class for typesetting,
%      with the options being: 11pt font, letterpaper, and double-spacing 
% \documentclass[11pt, letterpaper, doublespacing]{article}



% (2) This package enables the ``doublespacing'' option in the first line
\usepackage{setspace} 

% (3) Changing format of Section number and formatting of section headers

% The ``makatletter'' command is a special LaTeX switch
%  that changes the meaning of the ``@'' character, so that
%  this character can be used in the commands that follow.
% The ``makeatletter'' switch will be turned off using
%  the ``makeatother'' command below.

\makeatletter  

% (3.1)  Put a period after section number
\renewcommand{\@seccntformat}[1]{\@nameuse{the#1}.~{}}

% Change format of numbers
\renewcommand{\thesection}{\Roman{section}}
\renewcommand{\thesubsection}{\Alph{subsection}}
\renewcommand{\thesubsubsection}{\arabic{subsubsection}}

% (3.2)  Change format of header text
\renewcommand{\section}{\@startsection {section}{1}{\z@}%
  {-3.5ex \@plus -1ex \@minus -.2ex}%
  {2.3ex \@plus.2ex}%
  {\centering\normalfont\large\bfseries}}

\renewcommand{\subsection}{\@startsection{subsection}{2}{\z@}%
  {-3.25ex\@plus -1ex \@minus -.2ex}%
  {1.5ex \@plus .2ex}%
  {\normalfont\large\slshape}}

\renewcommand{\subsubsection}{\@startsection{subsubsection}{3}{\z@}%
  {-3.25ex\@plus -1ex \@minus -.2ex}%
  {1.5ex \@plus .2ex}%
  {\centering\normalfont\slshape}}

% (3.3)  The next command turns of the switch that changed the meaning of the “@” character.
\makeatother

%: (4) Changing formatting of theorem-like structures

% The next command loads the ``amstheorem'' package
% to adjust formatting of theorem-like structures
\usepackage{amsthm}

%Define ``plain'' style
\newtheoremstyle{plain}{9pt}{9pt}{\itshape}{0pt}{\scshape}{:}{3pt}{}

% Load ``plain'' style, so that it will apply to all theorem-style structures
% that are defined after this command
\theoremstyle{plain} 

% Then, for the Assumption, Corollary, Proposition, etc. ``plain'' style is active
% You can add more of these structures to suit your needs
\newtheorem{assumption}{Assumption}
\newtheorem{corollary}{Corollary}
\newtheorem{definition}{Definition}
\newtheorem{lemma}{Lemma}  
\newtheorem{proposition}{Proposition}
\newtheorem{remark}{Remark}
\newtheorem{theorem}{Theorem}

% (5)  Caption headings are bold and small
\usepackage[bf,small,nooneline,normal]{caption2}
\renewcommand*{\captionlabeldelim}{\nobreak}
\renewcommand*\captionlabeldelim{.}
\renewcommand{\captionfont}{\bfseries}

% (6) Table numbers are uppercase roman
\renewcommand{\thetable}{\Roman{table}}

% (7) First paragraphs are indented except the very first (unlabeled) section
\usepackage{indentfirst}

% (8) Change the formatting of the section header for Abstract
\renewcommand{\abstractname}{\bf \small ABSTRACT}

% (9) Change the formatting of the section header for References
\renewcommand{\refname}{\bf \small REFERENCES}

% (10) To move footnotes to the end of the text
\usepackage{endnotes}
\renewcommand{\footnote}{\endnote}
\renewcommand{\notesname}{\bf Footnotes}

% (11) Try to make the damn this compile
\usepackage{longtable}


% --------------------------------------------------------------------------------------------
%:JF STYLE COMMANDS END HERE --- STOP CUTTING HERE
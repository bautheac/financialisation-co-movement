\documentclass[]{elsarticle} %review=doublespace preprint=single 5p=2 column
%%% Begin My package additions %%%%%%%%%%%%%%%%%%%
\usepackage[hyphens]{url}

  \journal{no journal yet} % Sets Journal name


\usepackage{lineno} % add

\usepackage{graphicx}
%%%%%%%%%%%%%%%% end my additions to header

\usepackage[T1]{fontenc}
\usepackage{lmodern}
\usepackage{amssymb,amsmath}
\usepackage{ifxetex,ifluatex}
\usepackage{fixltx2e} % provides \textsubscript
% use upquote if available, for straight quotes in verbatim environments
\IfFileExists{upquote.sty}{\usepackage{upquote}}{}
\ifnum 0\ifxetex 1\fi\ifluatex 1\fi=0 % if pdftex
  \usepackage[utf8]{inputenc}
\else % if luatex or xelatex
  \usepackage{fontspec}
  \ifxetex
    \usepackage{xltxtra,xunicode}
  \fi
  \defaultfontfeatures{Mapping=tex-text,Scale=MatchLowercase}
  \newcommand{\euro}{€}
\fi
% use microtype if available
\IfFileExists{microtype.sty}{\usepackage{microtype}}{}
\bibliographystyle{elsarticle-harv}
\ifxetex
  \usepackage[setpagesize=false, % page size defined by xetex
              unicode=false, % unicode breaks when used with xetex
              xetex]{hyperref}
\else
  \usepackage[unicode=true]{hyperref}
\fi
\hypersetup{breaklinks=true,
            bookmarks=true,
            pdfauthor={},
            pdftitle={Financialisation of commodities: co-movement behind-the-scenes},
            colorlinks=false,
            urlcolor=blue,
            linkcolor=magenta,
            pdfborder={0 0 0}}
\urlstyle{same}  % don't use monospace font for urls

\setcounter{secnumdepth}{5}
% Pandoc toggle for numbering sections (defaults to be off)


% tightlist command for lists without linebreak
\providecommand{\tightlist}{%
  \setlength{\itemsep}{0pt}\setlength{\parskip}{0pt}}

% From pandoc table feature
\usepackage{longtable,booktabs,array}
\usepackage{calc} % for calculating minipage widths
% Correct order of tables after \paragraph or \subparagraph
\usepackage{etoolbox}
\makeatletter
\patchcmd\longtable{\par}{\if@noskipsec\mbox{}\fi\par}{}{}
\makeatother
% Allow footnotes in longtable head/foot
\IfFileExists{footnotehyper.sty}{\usepackage{footnotehyper}}{\usepackage{footnote}}
\makesavenoteenv{longtable}


\usepackage{geometry}
\usepackage{hyperref}
\usepackage{lscape}
\usepackage[margin=0pt]{caption}
\usepackage{booktabs}
\usepackage{longtable}
\usepackage{array}
\usepackage{multirow}
\usepackage{wrapfig}
\usepackage{float}
\usepackage{colortbl}
\usepackage{pdflscape}
\usepackage{tabu}
\usepackage{threeparttable}
\usepackage{threeparttablex}
\usepackage[normalem]{ulem}
\usepackage{makecell}
\usepackage{xcolor}



\begin{document}


\begin{frontmatter}

  \title{Financialisation of commodities: co-movement behind-the-scenes}
    \author[strath]{Devraj Basu\corref{1}}
   \ead{devraj.basu@strath.ac.uk} 
    \author[strath]{Olivier Bauthéac}
  
      \address[strath]{University of Strathclyde Business school}
      \cortext[1]{Corresponding Author}
  
  \begin{abstract}
  In the early 2000s institutional investors entered the commodity futures markets en masse with passive, long only, index type positions in sharp contrast with those typically assumed by traditional expert participants. A heated public debate soon erupted over the perceived consequences of the phenomenon--commonly referred to as ``financialisation''---and, in response to immediate policy concerns, the matter was thrust onto the academic sphere as a burning issue. With the benefit of hindsight it now seems that the academic debate was framed rather narrowly, used contentious research methods, and eventually led to regulatory changes that were therefore perhaps unwarranted. In contrast, we take a broader approach where we consider a large cross-section of liquid commodities as suggested by the nature of financilisation that comprehends commodity futures as an asset class. Using a bespoke asset pricing factors based framework we study the issue through the lens of co-movement and find that the phenomenon had ontological consequences for the commodity complex with its impact extending beyond the mechanical effects induced by indexation. The onset of the financial crisis and the monetary policy regimes that followed, on the other hand, seem to have set off a motion of reversion to legacy pre-financialisation fundamentals.
  \end{abstract}
  
 \end{frontmatter}

\newgeometry{a4paper,left=1in,right=1in,top=1in,bottom=1in,nohead}\savegeometry{tables}\restoregeometry

\cleardoublepage

\hypertarget{index}{%
\section{Introduction}\label{index}}

The last twenty years have seen major upheavals in the commodity markets with the onset of financialisation\footnote{Starting around 2004, with a view of commodity markets as an emerging asset class and in an effort to hedge against inflation and seek diversification benefits (Büyükşahin and Robe, 2014; Singleton, 2013), institutional investors sent forth an unprecedented flow of capital into commodity futures markets. This phenomenon is commonly referred to as the financialisation of commodities (Domanski and Heath, 2007).} as well as the period of the 2008 financial crisis and its aftermath. Financialisation appears to have affected multiple economic sectors, including agriculture and energy, and its impact has been hotly debated in the policy, legislative, regulatory, and academic spheres. A vigorous legislative debate accompanied by an equally intense policy and academic debate across a number of disciplines about the perceived effects on commodity prices of the unprecedented inflow of institutional funds brought into commodity futures markets by new index type investors\footnote{See section \ref{co-movement-background} for a detailed discussion.} eventually led to regulatory changes in several individual markets.\\
In response to immediate policy concerns, the academic debate was initially framed rather narrowly around adequacy of speculation--the burning issue of the day--at the individual commodity market level and the ensuing academic analysis focused on the more mechanical effects of financialisation. With the benefit of hindsight, this approach seems to lack a fundamental understanding of the phenomenon it endeavoured to study and which consequences it attempted to address.\\
Financialisation was a phenomenon global in nature; through the channel of index investment it affected the whole cross-section of liquid commodities at once, eventually spurring its transition away from a collection of idiosyncratic markets to its emergence as an asset class. Focusing on individual markets in this context thus seems limited. Most of the studies underpinning the regulatory response to financialisation took the form of pairwise correlation, causation and to some extent co-integration analysis, all of which are idiosyncratic in nature for they study single pairs of individual assets. Besides, some of these techniques since proved controversial and seem to produce results the quality of which may fall short of the standards that may reasonably be expected to lay the ground for legislative action. Although appealing, for it comes with the promise of prompt ``significant'' results, this approach misses a critical corollary of the substantial advent of a new category of players in a (set of) market(s): the alteration of market fundamentals it potentially induces, best observed in this context through the lens of commercial hedging pressure (Hicks, 1939; Keynes, 1930).\\
While essential to the understanding of the phenomenon, the global nature of financialisation and its impact on market fundamentals therefore appear to have been originally overlooked and another approach, perhaps more refined, seems warranted.

\medskip\setlength{\parindent}{0pt}

We strive to provide one in this study by taking a fundamentally different empirical perspective that is broader in scope. We consider the entire cross-section of actively traded commodities on futures markets and use a futures-based asset pricing framework that includes factors constructed using both theoretical fundamentals and liquidity considerations.\\
The global nature of financialisation makes cross-sectional co-movement a cornerstone issue that the direct impact of increased market participation alone might not be able to fully explain for it potentially had deeper effects, for example on market fundamentals as our results suggest. Commodity futures markets have a history of being the classic physical commodity price risk shifting conduit for commodity specialists. Over the financialisation period we observe a decrease in effectiveness of commercial hedging pressure (CHP) at explaining commodity futures returns at the individual level accompanied by a more pronounced increase at the aggregate level. We further note an overall shift in the nature of hedger's behaviour away from the traditional Keynesian view of risk transfer with higher individual risk premiums over phases of cross-sectional contango as opposed to backwardation. Both patterns suggest that the phenomenon may have altered this long-standing paradigm of commercial risk transfer with the entry of long only commodity index traders (CITs) with very different hedging and volume demands from traditional participants. This change could have altered the hedging ontology in these markets and our asset pricing framework is well suited to detect these more subtle effects driven by changes in the nature of market participation.\\
Factor model techniques are well suited to isolating common driving factors and detecting co-movement (Asness and Frazzini, 2013; Asness et al., 2015, 2019; Carhart, 1997; Fama and French, 2015, 1993; Frazzini and Pedersen, 2014; Hou et al., 2015) at a broader level, beyond individual pairs of assets, and are commonly used to study commodity market fundamentals (Bakshi et al., 2019; Boons and Prado, 2019; Cortazar et al., 2013; Daskalaki et al., 2014; Fernandez-Perez et al., 2018; Gorton et al., 2012; Miffre and Rallis, 2007; Sakkas and Tessaromatis, 2020; Schwartz and Smith, 2000; Szymanowska et al., 2014; Yang, 2013); they consequently seem well adapted to study these two issues simultaneously. Building on theory we therefore develop a bespoke commodity-based asset pricing framework comprised of factor models based both on fundamentals as well as liquidity considerations.\\
One of these is based on the Keynes-Hicks (Hicks, 1946, 1939; Keynes, 1930, 1923) risk transfer approach to the term structure; the corresponding ``theory of normal backwardation'', complemented for the possibility of contango (Cootner, 1960; H. Houthakker, 1957), postulates that futures prices for a given commodity are inversely related to the extent that commercial hedgers are short or long and using commercial hedging pressure (CHP) as a proxy for hedgers net market position we construct factor mimicking portfolios that capture the impact of hedging pressure as a systemic factor (Basu and Miffre, 2013). Working's contending approach that relates futures price formation to storage costs in the theory of ``the price of storage'' (Working, 1949) forms the basis of another model that we consider where, following Szymanowska et al. (2014) and Fuertes et al. (2015) who demonstrate that the term structure factor has explanatory power for the cross-section of commodity returns, we construct factor mimicking portfolios using roll-yield as a proxy for the front end shape of individual commodity term structures. We further consider a liquidity-based model following Hong and Yogo (2012) who find that open interest growth has predictive power for individual commodity returns and again, using the latter measure as a proxy for individual commodity liquidity, we construct long-short factor mimicking portfolios.

\bigskip\bigskip\setlength{\parindent}{0pt}

We study the performance of these models on a set of twenty-four US traded commodities and six UK traded (London Metal Exchange) base metals over the 1997/2018 period that we divide into four distinct sub-periods: pre-financialisation (1997/2003); financialisation (2004/2008\footnote{Starting point based on earlier studies (Baker, 2014; Christoffersen et al., 2014).}); financial crisis and aftermath, including loose monetary policy regimes that followed (2008/2013); and post-crisis (2014-2018).\\
Consistent with earlier studies, we observe that mean returns for the US traded commodities are sharply higher over the financialisation period compared to pre-financialisation and this pattern extends to UK traded metals.

\medskip\setlength{\parindent}{0pt}

The cross-sectional nature of financialisation (Basak and Pavlova, 2016; Cheng and Xiong, 2014) leads us to extend the Keynesian individual hedging pressure paradigm by incorporating aggregate CHP (\textbf{CHP}), a market wide measure of hedging pressure first considered in Hong and Yogo (2012). This paradigm implies that returns for individual commodities should be high in periods where \textbf{CHP} is low (\textbf{backwardation}) and low in periods where \textbf{CHP} is high (\textbf{contango}). Over the first period the cross-sectional average returns during phases of \textbf{backwardation} for both the US and UK commodities are positive and overall higher than during phases of \textbf{contango}, providing broad support for an aggregate version of the Keynesian hedging pressure hypothesis. Over the second period this pattern reverses, with both the US and UK commodities having higher returns during phases of \textbf{contango} as opposed to \textbf{backwardation}. This change in pattern suggests the onset of financialisation altered the traditional risk shifting nature of hedger's behaviour, implied by the Keynesian hypothesis.\\
This change of paradigm due to the arrival of financial investors, on the other hand, raises the question of whether they functioned as liquidity providers or demanders. Arguments have been put forward in favour of both possibilities with Moskowitz et al. (2012) arguing that financial investors provide liquidity and Kang et al. (2020) taking the opposite stance. Our analysis of individual commodity returns' relation with both CHP and \textbf{CHP} before and during financialisation sheds new lights on this issue with our results suggesting that financial investors may have been liquidity demanders leading hedgers to become liquidity providers.

\medskip\setlength{\parindent}{0pt}

Goldstein et al. (2014) and Goldstein and Yang (2017) argue that the arrival of financial investors had an impact on the extent of normal backwardation in commodity futures markets. Based on their analysis we would expect our average CHP factor, based on backwardation as measured by hedger's positions (Keynesian perspective, theory of normal backwardation: Keynes (1930), Hicks (1939), H. S. Houthakker (1957), Telser (1958)), as well our term structure factor, based on backwardation as measured by roll yield (Working's perspective, theory of storage: Working (1933), Kaldor (1939), Working (1948), Brennan (1958), Cootner (1960), Weymar (1966), Danthine (1978), Turnovsky (1983), Schwartz (1997)), to be best suited to measure the impact of financialisation.\\
We analyse the pricing dynamics over the different periods by first studying the time series explanatory performance of the various factors on all the US traded commodities with the factors constructed using the latter set of commodities. The market-based factor outperforms the other three in the first period and, although its performance increases over the financialisation period, that of the CHP factor shows a greater improvement in a relative sense. The performance of the market factor is driven by agricultural and energy commodities in the first period while metals strongly contribute to the performance improvement observed over the second period. Similarly, the dramatic improvement of the CHP factor in the second period, is driven by the metals complex, particularly that of the long leg where they predominantly load. Average pairwise correlations amongst individual factors' best performing commodities increase across the board over the financialisation period, far beyond the average for the whole cross-section of US traded commodities. The increase is particularly strong for the long-short factors, in a relative sense, on which the metal complex loads significantly more during financialisation. This increase is due to the dramatic rise in average pairwise correlation amongst the US traded metals all of which exhibited high \(R^{2}\) with respect to the CHP factor during financialisation which in turn is due to their common low level of CHP over this period. The CHP factor model thus seems able to establish the strengthening link between a non-price attribute and price dynamics in the US metal sub-sector during financialisation.\\
We explore this issue further by constructing the four factors independently with the top eight commodities for a given factor (factor picks) in each period\footnote{See section \ref{co-movement-methods} for further details.} and evaluating its performance on these selected commodities as well as on all US traded commodities. The results of this exercise provide evidence of convergence of Keynesian and Working's theories of the term structure in detecting price co-movement during financialisation and strongly suggest that the term structure related attributes (hedging pressure/roll) co-movement and price co-movement link was concentrated in the metals sub-sector.

\medskip\setlength{\parindent}{0pt}

The UK traded base metals provide a useful set of assets to test how widespread were the changes in pricing dynamics brought about by the onset of financialisation. To that end, we examine the performance of the four factors on the six UK metals, when constructed from each set of factor picks. The results show that there was co-movement in global metals returns in the second period and that this co-movement could be detected by our term structure related models, particularly so by our hedging pressure-based model.

\medskip\setlength{\parindent}{0pt}

The onset of the financial crisis and the monetary policy regimes that followed, on the other hand, also appear to have induced significant changes. Commodity futures returns fell dramatically into the crisis and the pricing results indicate the emergence of a systematic factor across the entire cross-section of the US commodities as well as evidence of cross-market linkages with the UK traded commodity subset. The post-crisis results, in contrast, show a substancial pullback in performance for the market factor across the board as well as evidence of decoupling between the US and UK markets. Besides, starting in the crisis the Keneysian paradigm of returns begins to revert with higher mean returns during low (\textbf{backwardation}) relative to high (\textbf{contango}) \textbf{CHP} phases and this pattern continues, more pronounced, over the post-crisis period. As pointed out in Basu and Bauthéac (2021), the financial crisis and ensuing accommodative monetary policy therefore seem to have initiated a process of reversion toward pre-financialisation fundamentals in the global commodity futures complex.

\bigskip\bigskip\setlength{\parindent}{0pt}

Financialisation was an issue of such policy importance that it triggered legislative action. The debate was initially framed around adequacy of speculation, the burning issue of the day, and the ensuing academic analysis focused on the more mechanical effects of financialisation.\\
With the benefit of hindsight, it appears this approach was perhaps too narrow, and it now seems necessary to address the phenomenon from a broader perspective. Commodity price dynamics appear to have altered substantially in quite different ways over the financialisation and the financial crisis periods and our commodity futures asset pricing factors based approach seems able to provide new insights into the nature of these changes: financialisation was a phenomenon endogenous to the commodity markets transmitted via the commodity futures markets and which effects were particularly strong for the metals sector while the crisis and its aftermath seems to have delivered an exogenous shock across the entire cross-section of liquid commodities and thereby potentially triggering a reversion to pre-financialisation dynamics.

The initial view of financialisation was that it consisted of speculative flows which had the effect of driving up and creating bubble like conditions in commodity futures prices\footnote{Masters (2008); Masters and White (2008); UNCTAD and Cooperation (2009); De Schutter (2010); Gilbert (2010a); Gilbert (2010b); Herman et al. (2011); Schumann (2011); Singleton (2013).}. However, the fundamental question of the nature of the impact of financialisation across the entire cross-section of commodity futures markets has not yet been completely answered. We provide an empirical complement to a new stream of theoretical studies that try to model the impact of financialisation on various aspects of commodity futures markets\footnote{Etula (2013), Acharya et al. (2013), Cheng et al. (2014), Leclercq and Praz (2014), Sockin and Xiong (2015), Goldstein et al. (2014), Ekeland et al. (2016), Goldstein and Yang (2017).} and demonstrate that the effects of financialisation extend beyond the mechanical effects induced by indexation as outlined in Tang and Xiong (2012) and Basak and Pavlova (2016).

\newpage

\hypertarget{co-movement-data-methods}{%
\section{Data \& methods}\label{co-movement-data-methods}}

\hypertarget{co-movement-background}{%
\subsection{Background}\label{co-movement-background}}

The period from 1998 to 2008 saw an influx of investment into commodity index linked products most of which made its way into the commodity futures market. Popular commodity indexes whose goal was to track the broad movement of commodity prices became accessible by means of swaps, exchange traded funds and exchange traded notes and attracted at least \$100 billion of net super-hoc investment over the 1998-2008 period.

\medskip\setlength{\parindent}{0pt}

Concerns over the consequences of this financialisation phenomenon eventually led to legislative changes including the approval of Rule 76 FR 4752 issued by the US Commodity Futures Trading Commission (CFTC) on January 26\textsuperscript{th}, 2011. This provision emanates from the Dodd-Frank Wall Street and Consumer Protection Act of 2010 (Title VII, Section 737) that mandates the CFTC to use position limits to restrict the flow of speculative capital into a number of commodity markets. The Rule was approved in a close 3-2 vote and the ensuing rule-making process was extremely contentious with several commissioners (Michael Dunn and Scott D. O'Malia in particular) expressing reservations about the lack of supporting evidence and the Rule also triggering thousands of comment letters as well as a lawsuit against the CFTC spearheaded by two Wall Street trade groups, the International Swaps \& Derivatives Association (ISDA) and the Securities Industries \& Financial Markets Association (SIFMA)\footnote{ISDA \& SIFMA v US CFTC; Complaint, 1:11-cv-02146; December 2\textsuperscript{nd}, 2011.}.

\medskip\setlength{\parindent}{0pt}

A world-wide debate ensued about the role of index funds in commodity markets. The first responses to the 2007/2008 crisis of escalating food and energy prices took the form of policy reports, many of which reasoned that the growth of commodity index funds came along with an influx of largely speculative capital that was responsible for driving commodity prices beyond their historic highs (De Schutter, 2010; Gilbert, 2010b; Herman et al., 2011; Schumann, 2011; UNCTAD and Cooperation, 2009).\\
Early US Senate investigations on the matter drew on pricing and trading data supplied by the CFTC as well as interviews with numerous experts including hedge fund manager Michael Masters who linked the growing presence of index investors in US agricultural and energy markets to the observed surges in both futures and cash prices (Masters, 2008; Masters and White, 2008). This contention has since come to be known as the Masters' hypothesis and was largely endorsed by the final US Senate report on the issue (Senate, 2009).\\
At this stage, the question thrust on the academic sphere was whether ``excessive speculation'' (understood as the market activity peculiar to index type investors) was linked to escalating energy and agricultural commodity prices. A number of ensuing academic studies, most of which used commodity specific correlation and causality based analysis have since disproved this contention: Irwin et al. (2009), Sanders et al. (2010), Sanders and Irwin (2011a), Sanders and Irwin (2011b), Irwin (2013), Brunetti and Reiffen (2014), Hamilton and Wu (2015) and Bruno et al. (2017) investigated the issue in the context of the agricultural markets; meanwhile, Büyükşahin and Harris (2011), Tokic (2012), Fattouh et al. (2013), Kilian and Murphy (2014), Knittel and Pindyck (2016) and Manera et al. (2016) examined the energy markets while Bohl and Stephan (2013), Kim (2015) and Boyd et al. (2016) studied both energy and agricultural markets and Irwin and Sanders (2011), Irwin and Sanders (2012) as well as Stoll and Whaley (2011) examined the commodity markets in general. Other studies have underlined various alterations in commodity pricing dynamics over the period. Gilbert and Pfuderer (2014) reject the view that financialisation has not had any effects in the grains markets and demonstrate that trades originated by financial market participants, and specifically index investors, can move prices but tend to be typically volatility-reducing. Juvenal and Petrella (2015) contend that the oil price increase between 2004 and 2008 was mainly driven by the strength of global demand but that the financialisation process of commodity markets also played a role. Likewise, Henderson et al. (2015) show that non-information-based financial investments have important impacts on commodity prices. Cheng and Xiong (2014), on the other hand, show that in the case of index commodities, financialisation has transformed the risk sharing, and information discovery functions of commodity futures markets.

\hypertarget{co-movement-data}{%
\subsection{Data}\label{co-movement-data}}

Commodity index swaps, commodity index funds (CIFs), commodity-based exchange traded funds (ETFs) and notes (ETNs) or commodity linked notes (CLNs) were amongst the main investment vehicles that allowed institutional and retail investors to build up commodity exposure in the early 2000s (Boons et al., 2012; Henderson et al., 2015; Irwin and Sanders, 2011; Schumann, 2011; UNCTAD and Cooperation, 2009). Most of these products were primarily designed to track prominent commodity indexes among which the S\&P-GSCI, the most popular.\\
Most of its constituents at the time of writing form the basis of the broad cross-section of commodities that we consider in this study: corn-\#2 yellow (XCBT), oats (XCBT), soybean meal (XCBT), soybean oil (XCBT), soybeans (XCBT), wheat-SRW (XCBT), cattle-feeder (XCME), cattle-live (XCME), lean hogs (XCME), cocoa (IFUS), coffee-C (IFUS), cotton-\#2 (IFUS), lumber (XCME), orange juice (IFUS), sugar-\#11 (IFUS), natural gas (XNYM), crude oil-WTI (XNYM), gasoline (XNYM), heating oil (XNYM), copper (XCEC), gold (XCEC), palladium (XNYM), platinum (XNYM) and silver (XCEC). Unleaded gasoline, the NYMEX legacy contract for gasoline, stopped trading in 2006 when it was replaced by Reformulated Gasoline Blendstock for Oxygen Blending (RBOB) with the two futures series trading alongside for most of the year. For our NYMEX gasoline data series, we consider unleaded gasoline up to September 1\textsuperscript{st}, 2006 and RBOB gasoline thereafter, date at which liquidity for the latter overtook that for the former.\\
We also consider the UK metal complex with aluminium (XLME), copper (XLME), lead (XLME), nickel-primary (XLME), tin-refined (XLME) and zinc (XLME).

\medskip\setlength{\parindent}{0pt}

The commodity futures trading commission (CFTC) operates a comprehensive system of collecting information on market participants known as the large trader reporting program (LTRP) where it records and reports to the public position data on market participants who have position levels in excess of a particular market specific threshold\footnote{Current reporting level thresholds available at: \href{https://www.ecfr.gov/cgi-bin/retrieveECFR?gp=\&SID=970471b8455f4bab7db4110cfde50731\&mc=true\&r=SECTION\&n=se17.1.15_103}{www.ecfr.gov}.}. The commission collects market data and position information daily from clearing members, futures commission merchants (FCMs) and foreign brokers and publishes corresponding summaries weekly in a series of market specific reports. In these reports, individual traders are categorised according to the nature of their trading activity on the basis of self-reported information\footnote{CFTC form 40. Available at: \href{https://www.cftc.gov/sites/default/files/idc/groups/public/@forms/documents/file/cftcform40.pdf}{www.cftc.gov}.} that is subject to review by CFTC staff for reasonableness. Position information is then aggregated by category and asset type with each report providing an aggregation for a set of report specific categories and, for each category, a breakdown between futures only positions and positions in futures and options combined.\\
In its legacy report, the commitment of trader report (COT) with data dating back to 1962, the CFTC distinguishes ``commercial'' and ``non-commercial'' market participants where a ``commercial'' participant is defined as one ``{[}\ldots{]} engaged in business activities hedged by the use of futures and option markets''\footnote{See \href{https://www.gpo.gov/fdsys/pkg/CFR-1998-title17-vol1/xml/CFR-1998-title17-vol1-sec1-3.xml}{CFTC regulation 1.3, 17 CFR 1.3(z)} for details.}. A third category, ``non-reportable'', aggregates positions for participants not meeting the reporting threshold. In response to concerns related to category accuracy\footnote{See for example Ederington and Lee (2002).} the CFTC now refines its classification in the disaggregated commitment of trader report (DCOT) where participant categories include ``Producer/Merchant/Processor/User'', ``Swap dealer'', ``Money manager'' and ``Other reportable'' as well as in the supplemental commitment of trader report (SCOT) that details commodity index trader aggregate positions for thirteen agricultural commodity futures markets. Data for DCOT and SCOT date back to 2006 and 2007 respectively.\\
We study the 1997/2018 period where we observe futures only position data from the COT report, the only CFTC report with data available for the whole period of interest; as well as futures term structure price and open interest data from Bloomberg.

\hypertarget{co-movement-methods}{%
\subsection{Methods}\label{co-movement-methods}}

On the one hand, futures markets have a long history of serving commodity producers to alleviate their commodity price and output risks. Keynes (1930) and Hicks (1939) emphasise the ``normal'' behaviour of naturally short hedgers outsourcing their commodity risk to naturally long commodity specialist speculators. This market configuration is referred to as ``backwardation'' where futures are expected to trade at a discount to spot prices, providing speculators with incentives to take the long side. The opposite market paradigm is referred to as ``contango''. Working (Working, 1953, 1948, 1933) and Kaldor (Kaldor, 1939) somehow relax this ``supply-of-speculative-services'' (Till, 2007) approach by introducing processors and merchants, also commodity experts, on the hedgers side and relating the notions of backwardation and contango more directly to the term structure via the theory of storage where the shape of the front end of the term structure (basis) is directly related to the market price of storage.\\
Financialisation, on the other hand, refers to the entry of financial investors into the commodity futures markets who, for the most part, are not commodity experts. Besides, these new market participants tend to exhibit herding like behaviour, often taking massive long-only positions on the whole cross-section of indexed commodities in an attempt to enhance returns while hedging against inflation and reaping further diversification benefits for portfolios with, in most cases, existing large positions across various asset classes (Boyd et al., 2016; Brunetti et al., 2016; Cheng et al., 2014; Juvenal and Petrella, 2015; Singleton, 2013; Tang and Xiong, 2012). This contrasts with the traditional approaches of Keynes and Working and in this context the issues of co-movement and aggregate market participants behaviour seem particularly relevant.

\medskip\setlength{\parindent}{0pt}

Asset pricing techniques are well suited to study co-movement. Building on theory we therefore develop an ad-hoc commodity-based asset pricing framework that we believe is well adapted to study these two issues simultaneously and implement it over four periods of interest. The first, the pre-financialisation period, starts in 1997 and is naturally followed by financialisation, spanning the 2004/2008 period, and the financial crisis and its aftermath, spanning the 2008/2013 period. The last period spans the 2014/2018 period starting with the beginning of the tapering/end of the US Quantitative Easing program (QE).\\
Using the weekly price, open interest and market position data described above we construct market portfolios of individual commodity nearby futures as well as mimicking portfolios for market, commercial hedging pressure, term structure and liquidity risk factors in returns. While our market portfolios are long only, equally weighted portfolios of particular commodity sets, our other mimicking portfolios for risk factors are constructed as combinations of two sub-portfolios (legs), one held long, one held short, which respective constituents are pooled on a weekly basis according to risk factor specific criteria.\\
For each leg, the position weight for a particular asset for a particular period is contingent on the value of the corresponding sorting variable and its rank relative to the other assets in the pool. The sorting variable values for the asset pool are first rescaled so that the highest (lowest) value becomes 1 (0). Weights are then determined pairwise, starting from the extremes with the highest and lowest figures. A coefficient \(\alpha\) is determined by dividing their difference (1 in this case) by the mean of the rescaled values. The high figure is attributed a score of \(\frac{1}{n} \times \alpha\) where \(n\) is the number of assets in the pool while the score for the low figure is \(\frac{1/n}{\alpha}\). The operation is repeated for the pair of assets showing the second highest and second lowest values for the sorting variable and so on until all the assets in the pool are attributed a score. The individual asset weights are then determined by dividing each score by the sum of the scores.\\
The period return for a particular mimicking portfolio is calculated as the difference between the weighted average return for constituents of the long portfolio and that for the constituents of the short portfolio. i.e, return on the mimicking portfolio for risk factor \(i\) (\(f_{i}\)) for period \(t\):

\[r_{t}^{f_{i}}=\sum_{j=1}^{x}w^{long}_{j,t} \times r_{j,t}-\sum_{j=n-x}^{n}w^{short}_{j,t} \times r_{j,t}\]
\(n\equiv\) number of commodities in the set considered for mimicking portfolio construction.\\
\(x = n \cdot s\).\\
\(s\equiv\) selection threshold (\(\frac{1}{3}\) here).\\
\(w^{*}_{j,t}\equiv\) period \(t\) weight for commodity \(j\) on the long or short leg.\\
\(r_{j,t}\equiv\) period \(t\) front contract return for commodity \(j\).\\
\(j\equiv\) commodity rank in the ordered set \(Y_{t}^{f_{i}}\).\\
\(Y_{t}^{f_{i}}\equiv\) period \(t\) ordered set of the commodities considered for mimicking portfolio construction where the ordering rule is specific to \(f_{i}\).

\medskip\setlength{\parindent}{0pt}

For our portfolio CHP (commercial hedging pressure), we sort the set of commodities considered on past twenty-six-week average CHP from lowest to highest. The bottom third constituents (lowest \(\overline{CHP}\)s) of the corresponding ordered set form the long leg of the portfolio while the top third form the short leg:

\[Y_{t}^{CHP}\equiv\left \{ \left \{ \overline{CHP_{1}}, \overline{CHP_{2}}, ..., \overline{CHP_{n}} \right \}, \leq \right \}\]

\(\overline{CHP_{j}}=\frac{1}{26}\sum_{k=0}^{25}\frac{L_{j,t-k}}{L_{j,t-k}+S_{j,t-k}}\)\\
\(L_{j,t}\equiv\) number of long positions held by commercial hedgers for period \(t\) on the futures series of commodity \(j\).\\
\(S_{j,t}\equiv\) number of short positions held by commercial hedgers for period \(t\) on the futures series of commodity \(j\).

\medskip\setlength{\parindent}{0pt}

For our portfolio TS (term structure), we sort the set of commodities considered on past twenty-six-week average roll-yield from highest to lowest. The bottom third constituents (highest \(\overline{RY}\)s) of the corresponding ordered set form the long leg of the portfolio while the top third form the short leg:

\[Y_{t}^{TS}\equiv\left \{ \left \{ \overline{RY_{1}}, \overline{RY_{2}}, ..., \overline{RY_{n}} \right \}, \geq \right \}\]

\(\overline{RY_{j}}=\frac{1}{26}\sum_{k=0}^{25}(\frac{F_{j,t-k}^{1}}{F_{j,t-k}^{2}} - 1)\)\\
\(F_{j,t}^{1}\equiv\) period \(t\) close price for commodity \(j\) first nearby futures contract.\\
\(F_{j,t}^{2}\equiv\) period \(t\) close price for commodity \(j\) second nearby futures contract.

\medskip\setlength{\parindent}{0pt}

For our portfolio OI (open interest), we sort the set of commodities considered on past twenty-six-week average front contract open interest growth from highest to lowest. The bottom third constituents (highest \(\overline{OI^{\Delta \%}}\)s) of the corresponding ordered set form the long leg of the portfolio while the top third form the short leg.

\[Y_{t}^{OI}\equiv\left \{ \left \{ \overline{{OI}_{1}^{\Delta \%}}, \overline{{OI}_{2}^{\Delta \%}}, ..., \overline{{OI}_{n}^{\Delta \%}} \right \}, \geq \right \}\]

\(\overline{{OI}_{j}^{\Delta \%}}=\frac{1}{26}\sum_{k=0}^{25}(\frac{{OI}_{j,t-k}}{{OI}_{j,t-k-1}} - 1)\)\\
\(OI_{j,t}\equiv\) open interest at close of period \(t\) on the front (nearby) contract for commodity \(j\).

\medskip\setlength{\parindent}{0pt}

We use a time series regression approach where we start with regressions of the US traded individual commodity weekly front contract returns on returns to the mimicking portfolio for a particular risk factor where the portfolio is constructed using the whole cross-section of US traded commodities. The corresponding \(R^{2}\)s are averaged and the process is repeated for each of our four risk factors (market, hedging pressure, term structure and liquidity) and periods (1997/2003, 2004/2008, 2008/2013 and 2014/2018). For the long-short factors the results are reported independently for models where the regressor is the factor itself or one of its legs (long vs.~short). The same operation is repeated for the two (\(p=2\)) and three (\(p=3\)) factor model cases, where regressors include in turn all combinations of any two and three risk factors respectively, as well as for the four (\(p=4\)) factor model case where all the risk factor mimicking portfolios are included. Results are reported for the one-factor model case with the other results available on the \href{http://18.135.131.217:3838/financialization-asset-pricing/}{online results dashboard}.

\medskip\setlength{\parindent}{0pt}

For each period we also implement the analysis described above over periods of low and high aggregate CHP (\textbf{CHP}), a notion that we believe is particularly relevant in the context of financialisation where issues of co-movement and aggregate market behaviour play a central role. We define \textbf{CHP} as the cross-sectional average of CHPs for a particular pool of assets. i.e., period \(t\) \textbf{CHP} for asset pool \(x\):

\[\mathbf{CHP}_{t}^{x}=\frac{1}{n}\sum_{j=1}^{n}CHP_{j,t}\]
\(n\equiv\) number of commodities in set \(x\).\\
\(CHP_{i,t}\equiv\) period \(t\) CHP for commodity \(i\) as defined above.

\medskip\setlength{\parindent}{0pt}

We further define aggregate backwardation (\textbf{backwardation}) and contango (\textbf{contango}) \textbf{CHP} regimes as, for a given period, \textbf{CHP} levels below and above the period's median respectively and, in contrast with (Hong and Yogo, 2012) who aggregate across sub-sectors, the asset pool we consider for \textbf{CHP} construction include the whole cross-section of US traded commodities.

\medskip\setlength{\parindent}{0pt}

In the next part of the analysis, our mimicking portfolios for risk factors, constructed from the set of all US traded commodities are used as commodity picking devices on the latter commodity set. For each risk factor and period, individual US traded commodity front returns are regressed on returns for the corresponding mimicking portfolio. Individual commodities are sorted by \(R^{2}\) accordingly from highest to lowest and the top third (highest \(R^{2}\)s) are selected as commodity picks for the risk factor for the period. i.e, set of picks \(P_{t}^{f_{i}}\) for risk factor \(i\) (\(f_{i}\)) over period \(t\):

\[P_{t}^{f_{i}}\equiv\left \{ P_{1}, P_{2}, ..., P_{x} \right \}\]
\(x = n \cdot s\).\\
\(n\equiv\) number of commodities in the set considered for the picking exercise (twenty-four here).\\
\(s\equiv\) selection threshold (\(\frac{1}{3}\) here).\\
\(P_{j}\equiv\) commodity pick \(j\) for risk factor \(f_{i}\) over period \(t\)\\
\(j\equiv\) rank in the ordered set \(Y_{t}^{f_{i}}\)\\
\[Y_{t}^{f_{i}}\equiv\left \{ \left \{ \tilde{R_{1}^{2}}, \tilde{R_{2}^{2}}, ..., \tilde{R_{n}^{2}} \right \}, \geq \right \}\]
\(n\equiv\) number of commodities in the set considered for picking (twenty-four here).\\
\(\tilde{R_{j}^{2}}\equiv\) risk factor \(f_{i}\) returns' explanatory power on commodity \(j\) nearby futures returns over period \(t\) as defined above.

\medskip\setlength{\parindent}{0pt}

For each period we report the proportion of time each pick is held in the long and (/or) short leg of the corresponding risk factor mimicking portfolio as well as the average pairwise correlation amongst every set of picks.

\medskip\setlength{\parindent}{0pt}

In the last part of the analysis, for each period our mimicking portfolios for risk factors in returns are constructed using each set of commodity picks independently as the asset pool for portfolio construction. A similar time series approach to that described above is then implemented using these newly formed mimicking portfolios. In this case, a first set of dependent variables includes the sets of commodity picks themselves, a second set includes the whole cross-section of US traded commodities while a third set includes the six UK traded metals considered in the study.

\newpage

\hypertarget{co-movement-results}{%
\section{Results \& discussion}\label{co-movement-results}}

The descriptive statistics for the commodity series considered in the study displayed in table \ref{tab:co-movement-stats-no-regimes} show a clear pattern of increase in average return during financialisation with an equally weighted portfolio of the US traded assets showing a mean returns of 15.9\% significant at the 5\% level vs.~6.9\% with a 10\% significance in the pre-financialisation period. The increase is even more pronounced for an equally weighted portfolio of UK traded assets (metals), 2.47\% (not significant) to 21.8\% (significant at the 10\% level), which sets out the global nature of the phenomenon. Most commodities show a higher mean return while five--Lumber (XCME), Natural gas (XNYM), Palladium (XNYM), Platinum (XNYM) and Nickel-primary (XLME)--show a lower figure with five significant figures--Crude oil-WTI (XNYM), Heating oil (XNYM), Copper (XCEC), Copper (XLME) and Tin-refined (XLME)--against none pre-financialisation. Sugar-\#11 wins the race with a mean return increasing from -4.6\% to 24.9\% followed by copper with sizeable increases for both markets: 1.5\% to 28.2\% (XCEC) and 0.5\% to 28.6\% (XLME).\\
The pattern of results for volatilities is similar although the increase is less pronounced. Both equally weighted portfolios shows an increase, more pronounced for the UK assets, 10\% to 13.6\% (US) vs.~15.7\% to 26.9\% (UK), while out of the thirty assets, eleven show a drop in volatility: Oats (XCBT), Lean hogs (XCME), Cocoa (IFUS), Coffee-C (IFUS), Lumber (XCME), Sugar-\#11 (IFUS), Natural gas (XNYM), Crude oil-WTI (XNYM), Gasoline (XNYM), Heating oil (XNYM), Palladium (XNYM). Financialisation seems however to have impacted the commodity complex beyond these mechanical effects of increased long participation ensuing from the entry on index type investments as our results suggest next.

\medskip\setlength{\parindent}{0pt}

The cross-sectional nature of financialisation naturally leads us to extend the individual hedging pressure paradigm by incorporating aggregate CHP (\textbf{CHP}), a market wide measure of hedging pressure. The traditional Keynesian hedging pressure paradigm postulates a negative relationship between the return on an individual commodity futures and its hedging pressure. The extended Keynesian hedging pressure paradigm, in the context of financialisation, should also imply that returns for individual commodities should be high in periods where \textbf{CHP} is low and vice et versa. In table \ref{tab:co-movement-stats-regimes} the mean returns are shown over high (\textbf{contango}) and low (\textbf{backwardation}) \textbf{CHP} regimes for each time period.\\
Over the first period eighteen of the twenty-four US commodities have higher mean returns during periods of \textbf{backwardation} as compared to \textbf{contango}, with the difference being statistically significant at the 10\% level for two commodities, at the 5\% level for one and at the 1\% level for another. For the remaining six commodities the higher mean return during \textbf{contango} is not statistically significant for any of them. For the six UK commodities, all metals, five of them had higher mean returns during phases of \textbf{backwardation} as compared to \textbf{contango} in the first period, with one of these differences being statistically significant at the 10\% level while for the remaining asset (Zinc) the higher mean return during \textbf{contango} is not statistically significant. An equally weighted portfolio of the twenty-four US commodities had a mean return of 15.4\% during \textbf{backwardation} phases, statistically significant at the 1\% level, and a mean return of -1.1\% during \textbf{contango}, both in the first period, with the difference between them statistically significant at the 5\% level. The corresponding figures for an equally weighted portfolio of the UK metals were 4.4\% and 0.3\% respectively. These results provide broad support for an aggregate version of the Keynesian hedging pressure hypothesis and suggest that hedgers in the aggregate were engaged in risk transfer during this period.\\
Over the second period the pattern essentially reverses. For the twenty-four US commodities over this period, fourteen have higher mean returns during phases of \textbf{contango} over \textbf{backwardation}. Four of the six UK metals achieved higher mean returns over \textbf{contango} relative to \textbf{backwardation} phases with the difference being statistically significant for one commodity. The equally weighted US portfolio had a mean return of 15.7\% over \textbf{backwardation} against 16.2\% during phases of \textbf{contango}, both statistically significant at the 10\% level. The difference was considerably more pronounced for the equally weighted UK metals portfolio with a mean return of 8\% during \textbf{backwardation} rising to 35.6\% during \textbf{contango}.

\medskip\setlength{\parindent}{0pt}

The onset of financialisation thus seems to have engendered a change in the nature of hedger's behaviour, at the aggregate level in particular, taking it away from the traditional Keynesian view of risk shifting. This phenomenon has been noted in earlier theoretical and empirical work (Danthine, 1978; Stout, 1998) in different contexts from which two competing models of behaviour have emerged.\\
The information arbitrage model (Danthine, 1978; Grossman and Stiglitz, 1980; Kyle, 1989) implies that hedgers may often be seeking out counterparties to trade with rather than being purely passive. This issue is also raised in both Cheng and Xiong (2014) and Stulz (1996) who point out that hedgers may be taking a view on prices just as speculators do. In fact, by hedging away some of their risk, hedgers are able to speculate more heavily based on their disagreement against speculators regarding futures price movement (Simsek, 2013), which fits into a heterogeneous expectation theory of speculation\footnote{Hirshleifer (1975); Hirshleifer (1976); Hirshleifer (1977)}. As Stout (1998) points out, this disagreement-based theory of trading is one of the main reason the public and the law disapproves speculation as this form of trading is regarded as non-productive\footnote{Duffie (2014) also discusses some of the challenges faced by a policy treatment of speculative trading motivated by differences in beliefs.}.\\
In the second model hedgers are seen as liquidity providers as in Kang et al. (2020) with this explanation also consistent with the nature of financialisation which led to the arrival of long term, long only investors who require substantial liquidity to roll-over their positions. Table \ref{tab:co-movement-time-reg-US-CHP} supports this view. We regress commodity futures front returns on individual commercial hedging pressure (CHP) as well as aggregate CHP (\textbf{CHP}), our market wide measure of CHP, in separate models and find that the explanatory power of CHP at the individual commodity level decreases over the financialisation period with the US cross-sectional average \(R^{2}\) falling down to 20.8\% from 25.8\% in the previous period while that of \textbf{CHP} increases (7.1\% vs 5.4\%). Financialisation therefore seems to have had the effect of loosening the extent to which futures risk premiums relate to commodity specific commercial hedging motives while strengthening that to which they relate to broader dynamics such as cross-sectional liquidity provision. The pattern is strongest for the metals sector where the average \(R^{2}\) drops down to 19.1\% from 28.1\% between the two periods at the individual level while it rises from 8.4\% to 15.7\% at the aggregate level. The symmetry breaks in the crisis period, with the US cross-sectional average \(R^{2}\) at the individual level bouncing back up (22.5\% vs 20.8\%) while that at the aggregate level rises further (14.7\% vs 7.1\%), before eventually resuming post-crisis with a further rise at the individual level (27.2\% vs 22.5\%) and a drop at the aggregate level (7.5\% vs 14.7\%) suggesting a return to legacy commodity specific dynamics.

\bigskip\bigskip\setlength{\parindent}{0pt}

Using our bespoke commodity-based asset pricing framework we strive to shed new light on the phenomenon by first regressing returns series for the twenty four individual US commodities against our factor mimicking portfolios independently. The resulting individual explanatory powers (\(R^2\)s), once averaged by factor and period, offer a refined perspective on the relation between return and factor attribute co-movement. A low figure indicates low cross-sectional return co-movement for, in this context, return on the corresponding factor mimicking portfolio only has explanatory power on return for commodities that are constituents of one of its legs over the corresponding period. The explanatory power for the other commodities is typically low or even 0 for the case where commodities are perfectly uncorrelated with one-another. A rising figure on the other hand thus indicates increased pairwise correlations, positive or negative, including amongst factor leg mimicking portfolio constituent commodities, which suggests that pairwise correlation relate to factor attribute.\\
The average time series explanatory power of the various one factor models on the US traded assets over the four periods is shown in Table \ref{tab:co-movement-time-reg-US-facts-US}. Over the first period the market factor has the best average time series performance with the highest average \(R^{2}\) with all three futures market factors performing poorly, indicating that the commodities with extreme factor attributes (high or low CHP for example) are uncorrelated with the rest of the commodity cross-section, and the overall performance consistent across periods of high and low \textbf{CHP} (below period median). The two factor models average \(R^{2}\)s (results available on the \href{http://18.135.131.217:3838/financialization-asset-pricing/}{online results dashboard}) are roughly the sum of the one factor \(R^{2}\)s indicating that the factors are almost orthogonal over this period.\\
The onset of financialisation leads to a substantial improvement in the performance of the CHP factor with its average \(R^{2}\) increasing to 8.2\% in the second period relative to 1.7\% in the first with this improvement coming from the long leg that shows a greater improvement (4.5\% to 15.1\%) relative to the short (3\% to 4\%), suggesting that ``normally'' backwardated commodities (Keynesian perspective) have higher correlation with the rest of the commodity cross-section, and thereby bringing more evidence of the important role played by long only investors in the change in nature of these markets over that period. The performance of the market factor also improves though not by as much in a relative sense (21.6\% from 11.7\%) and both the term structure factor and the open interest growth factor show a much smaller improvement in performance with smaller differences between legs. Over all periods, all of our factor models show similar performance across regimes of high and low \textbf{CHP}.

\medskip\setlength{\parindent}{0pt}

We next investigate the top eight of the US traded commodities which achieved the highest time series \(R^{2}\) for the CHP, market, open interest and term structure factors, in each of the four time periods with the commodity picks shown in Table \ref{tab:co-movement-picks}.\\
For the CHP factor the commodities that achieve high \(R^{2}\) in a given period are likely to be those with relatively high or low CHP over that period, or those commodities that co-vary with the short or the long side of the factor. The same applies to the open interest and term structure factors with CHP replaced by open interest growth and roll yield respectively; while for the market factor the picks are those that co-vary most strongly with the market factor over that period.\\
For the CHP factor, we observe a change in the picks between the first and the second period; the picks in the first period included five agricultural commodities and three metals while in the second period they include all the five base and precious metals and only three agricultural commodities. This pattern is also visible for the term structure factor for which there are four metals (copper, silver, gold and platinum) picks in the second period against two in the first (palladium and platinum) as well as for the open interest nearby growth factor for which there are three metals (silver, gold and platinum) picks in the second period against one in the first (platinum). The market factor has three metal picks in the second period (palladium, gold and silver) while all the picks in the first period were agricultural or energy commodities.\\
Over the third period, the average CHP factor shows the same metal picks with the agriculturals replaced by energy commodity; the open interest nearby growth factor still has three metal picks with gold swapped for copper and both the market and term structure factors have one, copper.\\
The table also shows the proportion of time that these picks were constituents of the corresponding factor. All the picks of the long-short factors (average CHP, open interest nearby growth and term structure factors) were constituents of the corresponding factor for some proportion of time in all of the time periods. In this respect, there is a contrast between the first and second periods with the average CHP and open interest nearby growth factor picks appearing predominantly on the long side of the corresponding factor in the second period, as opposed to both legs in the first. The pattern is particularly strong for the metals and energy commodities respectively, with the exception of copper for the average CHP factor that lives in the short leg a higher proportion of the time over the financialisation period. The pattern is reversed for the term structure factor where the picks appear predominantly on the short leg in the second period, as opposed to both legs in the first with the pattern again strongest for the metals. This behaviour continues over the third period for the average CHP and open interest nearby growth factors and indicates that their picks consistently exhibited low commercial hedging pressure and high front contract open interest growth respectively while indicating that the term structure factor picks showed low roll yield in the first and second period.

\medskip\setlength{\parindent}{0pt}

In table \ref{tab:co-movement-correlations}, we show the average pairwise returns correlation between each set of picks in each time period. Following on form the earlier observations, this exercise gives an indication of the extent to which attribute co-movement reveals price co-movement, which is particularly relevant for the long-short factors for which the corresponding attribute is not directly related to price co-movement.\\
In the first period the average picks pairwise correlation for the long-short factors ranges from 7.1\% (term structure factor picks) to 12.4\% (open interest factor picks) while the market factor picks show the highest figure (23.7\%). In the second period the figures for the long-short factors rise sharply; up to 22.9\% from 8.1\% for the average CHP factor picks, up to 32.4\% from 12.4\% for the open interest factor picks and up to 23.4\% from 7.1\% for the term structure factor picks. This increase is stronger than that for the market factor picks in a relative sense and is a consequence of the dramatic increase in average pairwise price correlation amongst the US traded metals which dominate the average CHP factor set of picks and which presence is stronger amongst both the open interest and term structure factor sets of picks over the financialisation period. The higher figures for the open interest factor are driven by the strong presence of the energy complex amongst the corresponding picks. The latter exhibits much higher average pairwise correlations than the other sectors in both periods although lower in the second relative to the first which reinforces the role played by the metals in the sharp increase in average pairwise correlation amongst picks for this factor.\\
With regards to \textbf{CHP} regimes, although not very pronounced, a pattern is noticeable. For the market, open interest and term structure factor picks, average pairwise correlations are stronger over phases of \textbf{contango} in the first period and over phases of \textbf{backwardation} in the financialisation period, eventually reverting back to being stronger over \textbf{contango} phases in the crisis period with the CHP factor sets of picks showing the opposite pattern. Overall, these results suggest an increase in correlation between non-price attributes co-movement and price co-movement over the financialisation period. It is also interesting to note that US traded metals returns were negatively correlated with the rest of US traded commodities in the first period and positively in the second.

\medskip\setlength{\parindent}{0pt}

To further understand the link between attribute and price co-movement we build all of the factors from every set of factor picks and assess their explanatory performance on these sets of picks over the four periods independently. The results are shown in table \ref{tab:co-movement-time-reg-picks-facts-picks}. The long-short factors for a particular attribute select the top and bottom two of the corresponding picks based on that attribute while the market factor takes an equally weighted combination of the eight corresponding picks. In the absence of price co-movement between the picks, the market factor will achieve an average \(R^{2}\) of around 12.5\%\footnote{Market factor lemma: \(\overline{R_{r_{mkt}, p}^{2}} = \frac{1}{n} \sum_{j} R_{r_{j}, r_{mkt}, p}^{2} = \frac{1}{n}; \: n\equiv\) size of market portfolio (eight here).}, while there is no non-zero \(R^{2}\) lower bound for the corresponding long-short factors.\\
A high average \(R^{2}\) for the market factor is indicative of strong price co-movement. In this respect, the most remarkable change in performance in a relative sense is that of the market factor built from the CHP factor sets of picks with the average \(R^{2}\) more than doubling, from 18.6\% up to 38.1\%. For the long-short factors on the other hand it is indicative of simultaneous attribute and price co-movement. For the factors built from the market factor picks the increase is strongest in a relative sense for the CHP factor with the \(R^{2}\) more than doubling, going from 5.6\% in the first period up to 12.4\% in the second with the increase stronger for the long leg (17.1\% vs.~27.9\%) relative to the short (16.5\% vs.~21.4\%). In the case of the factors built from the CHP factor picks, the CHP factor built with its own picks shows a remarkable increase in a relative sense with the \(R^{2}\) tripling, from 7.2\% to 21.4\% and the long leg showing a much greater increase relative to the short (10.4\% to 31.1\% vs.~8.3\% to 9.8\%). The performance of the term structure factor built from the CHP factor sets of picks also shows a sizable increase in a relative sense, from 3.5\% to 5.3\% and even more remarkable in this case is the substantial increase in performance of the short leg, from 6.3\% to 15.7\% relative to the long (7.5\% to 9.8\%). The pattern is reversed for the CHP factor built from the term structure factor picks where the average \(R^{2}\) increases from 4.7\% to 8\% with the performance of the long leg increasing from 11.9\% to 28.1\% vs.~5.5\% to 9.1\% for the short. The performance of the term structure factor constructed from its own sets of picks also shows a substantial increase, from 5.1\% to 8.4\% with, here again the increase in performance relatively stronger for the short leg, from 6\% to 16.1\% vs.~7.6\% to 12.1\% for the long.\\
Taken together with the previous results showing that the metal complex predominantly loads on the long and short legs of the original CHP and term structure factors respectively, these results provide evidence of convergence of Keynesian and Working's theories of the term structure in detecting price co-movement during financialisation perhaps as a result of spillover effects between the corresponding futures and spot markets as suggested in Basak and Pavlova (2016).

\medskip\setlength{\parindent}{0pt}

In table \ref{tab:co-movement-time-reg-US-facts-picks} we assess the performance of the same factors on all the commodities instead of just the picks. While the average \(R^{2}\)s are expectedly lower, the pattern of results mostly holds with the relative performance of the market factors built with different sets of picks as well as that of the CHP and term structure factors built with each other's picks similar to table \ref{tab:co-movement-time-reg-picks-facts-picks}. In contrast, the average CHP factor built with average CHP factor picks shows a smaller absolute increase in average \(R^{2}\) indicating that the link between CHP co-movement and price co-movement is stronger amongst the picks which are predominantly metals. Taken together, the results in tables \ref{tab:co-movement-time-reg-picks-facts-picks} and \ref{tab:co-movement-time-reg-US-facts-picks} suggest that the onset of financialisation had a major impact on the dynamics of US traded precious and precious-base metals.\\
To assess whether this phenomenon extended beyond US borders we examine the performance of the factors used in the last two tables above on the six UK metals futures traded on the London Metal Exchange with the results reported in table \ref{tab:co-movement-time-reg-UK-facts-picks}. In the first period, although all sets of picks, except that of the market factor, included at least one metal, the average \(R^{2}\)s are very low; close to zero in most cases. These figures drastically increase across the board in the financialisation period and show a pattern similar to that observed in the US complex. The market factor built from the CHP and term structure factor sets of picks shows the sharpest increase in performance, relative to using the other sets of factor picks; rising to 20.1\% from 1.5\% and to 15\% from 0.8\% respectively. The increase in \(R^{2}\) for the market factor built with market factor picks is more modest, providing further evidence of concentration of co-movement in the metals sector during financialisation.\\
Here again the change in performance for the CHP and term structure factor legs stands out where the factors are built with their own as well as each other's sets of picks. The average \(R^{2}\) for the long leg of the CHP factor built with its own sets of picks increases from 1\% to 11\% while the short leg shows no significant change. The opposite holds for the term structure factor built from its own set of picks with the short leg performance increasing from 0.1\% to 14\% while that of the long leg remains below 1\%. Building these factors from each other's sets of picks produces a similar pattern with the performance of the long leg of the CHP factor built from the term structure factor picks increasing from 0.7\% to 11.4\%; a much greater increase relative to the short leg although in this case the latter also shows a sizable increase in performance (0.3\% to 6\%). The performance of the short leg of the term structure factor built from the CHP factor sets of picks on the other hand increases from 0.3\% to 19.2\% while the performance for the long leg in the second period remains below 1.5\%.\\
Taken together these results emphasise the global nature of the impact of financialisation on the metals complex and provide further evidence that global metals co-movement could be detected by US based factor models that combine the Keynesian and Working's term structure paradigms. There seems thus to be information in the factor constructed from hedging pressure based on the Keynesian paradigm relevant to the term structure factor built on the Working paradigm and these findings therefore appear to complement the results of Kang et al. (2020) who find that hedging pressure both conveys information about liquidity provision in the short term (Working paradigm) and risk transfer in the longer term (Keynesian paradigm).

\bigskip\bigskip\setlength{\parindent}{0pt}

The crisis period (late 2008 to mid-2013) shows a contrasting pattern of results. Returns fall sharply relative to the financialisation period (table \ref{tab:co-movement-stats-no-regimes}) with eighteen US commodities along with all six UK traded metals showing lower mean returns. Returns for equally weighted portfolios of both sets of assets fall down to 9.4\% and 5.3\% from 15.9\% and 21.8\% respectively. Returns are mostly higher during phases of \textbf{backwardation} relative to \textbf{contango} (table \ref{tab:co-movement-stats-regimes}) with fifteen US assets and five of the UK traded metals showing higher mean returns in \textbf{backwardation} and the pattern extending to equally weighted portfolios of the two sets of assets. Volatility is moderately higher for both sets with fifteen US commodities and three UK metals showing higher figures and interestingly eminently higher over phases of \textbf{contango}; out of the US complex only Sugar-\#11 (IFUS) and none of the UK traded metals show higher volatility figures over \textbf{backwardation} phases. The symmetry between the dynamics of CHP and \textbf{CHP}'s explanatory power on individual commodity returns breaks in the crisis period with both now seemingly converging; the US cross-sectional average \(R^{2}\) at the individual level bouncing back up (22.5\% vs 20.8\%) while that at the aggregate level rises further (14.7\% vs 7.1\%).\\
The performance of the market factor dominates the US commodity complex over the period with a 50\% increase in mean average \(R^{2}\) to over 30\% (table \ref{tab:co-movement-time-reg-US-facts-US}) while that of the CHP factor decreases and, although increasing, remains modest for the open interest growth and term structure factors. The average pairwise correlation amongst the market factor picks increases beyond 50\% while there is a leveling in that amongst the picks of the long-short factors as shown in table \ref{tab:co-movement-correlations}. Notwithstanding, the market factor built from any sets of picks shows a significant rise across the board in performance on the US complex (table \ref{tab:co-movement-time-reg-US-facts-picks}).\\
Taken together, these results indicate the presence of a systematic factor across the entire cross-section of the US commodities. Table \ref{tab:co-movement-time-reg-UK-facts-picks} provides further evidence of systematic linkages, in this case between US and non-US traded commodities with a similar, albeit more pronounced pattern, where the market factor constructed with any sets of picks (US traded commodities) shows a dramatic increase in performance on the UK metals with for example, when constructed from its own sets of picks, the average \(R^{2}\) increasing to 31.7\% from 10.2\% in the previous period.

\medskip\setlength{\parindent}{0pt}

The returns pullback continues into the post-crisis period (table \ref{tab:co-movement-stats-no-regimes}) with twenty-two US and four UK traded assets showing lower mean returns relative to the crisis while the same number of US commodities and all UK metals show lower figures relative to the financialisation period; mean returns for equally weighted portfolios of US and UK assets drop down to -0.2\% and 2.7\% respectively. Interestingly volatility also recedes substantially with twenty US assets and all UK metals showing lower volatility figures relative to both the crisis and financialisation periods suggesting that investment outflows from the commodity futures complex have had a lower impact on volatility than the corresponding inflows witnessed over the financialisation period. The figures for the equally weighted portfolios of US and UK assets drop to 9.9\% and 16.2\% respectively, very close to pre-financialisation levels.\\
The pattern of reversion toward the pre-financialisation Keynesian paradigm initiated during the crisis continues stronger into the post-crisis period (table \ref{tab:co-movement-stats-regimes}) with mean returns higher during phases of \textbf{backwardation} relative to \textbf{contango} (table \ref{tab:co-movement-stats-regimes}), more so than over the crisis period with seventeen US commodities showing higher mean returns over \textbf{backwardation} phases. The difference is significant at the 5\% level for Lumber (XCME) and at the 10\% level for Palladium (XNYM) while no difference is significant for the assets showing higher figures over phases of \textbf{contango}. The pattern is more pronounced for the UK traded subset with all six assets showing higher mean returns in \textbf{backwardation} with the difference significant at the 1\% level for Nickel-primary (XLME), 5\% level for Aluminium (XLME) and 10\% level for Zinc (XLME). It also extends to equally weighted portfolios of US and UK traded assets that both show higher mean returns during \textbf{backwardation} with the difference significant at the 5\% level for the UK portfolio and the figure close to the 10\% level significance threshold for the portfolio of US assets. Volatilities are overall higher in \textbf{contango} although less so than over the previous period with seventeen US commodities showing higher figures over \textbf{contango} against twenty-three in the crisis period. The pattern extends to three of the six UK traded metals with the difference significant at the 1\% level for Copper (XLME) and Tin (XLME) and at the 5\% level for Zinc (XLME) while differences are not significant for Aluminium (XLME), Lead (XLME) and Nickel-primary (XLME) that exhibit higher volatility over \textbf{backwardation} phases. Equally weighted portfolios of US and UK traded assets on the other hand both show higher volatility in \textbf{contango} with the difference significant at the 1\% level for both. Over the financialisation period in contrast volatilities were predominantly higher in \textbf{backwardation} for US traded assets with fourteen commodities showing higher volatility figures in these phases; the post-crisis pattern of results for volatilities thus likewise shows signs of reversion to pre-financialisation standards. After showing signs of convergence in the crisis period the dynamics of the explanatory power of CHP and \textbf{CHP} part ways (table \ref{tab:co-movement-time-reg-US-CHP}), seemingly reverting to the Keynesian paradigm observed pre-financialisation, further suggesting a return to legacy commodity specific dynamics. The recovery initiated in the crisis period for that of CHP is confirmed with a stronger increase in the post-crisis period, 27.2\% vs.~22.5\%, strongest for the metals sector, 37.3\% vs.~23.7\%, while the energy sector shows the opposite pattern: 11\% vs.~16\%. In contrast, after increasing stronger in the crisis, that for \textbf{CHP} falls overall in the post-crisis period, 7.5\% vs 14.7\%, with the decrease strongest for the energy sector in a relative sense: 2.5\% vs.~14.8\%.\\
Average pairwise correlations amongst factor picks (table \ref{tab:co-movement-correlations}) fall substantially in the post-crisis period with for example the figure for the market factor picks dropping from 50.1\% to 26.1\%, almost down to pre-financialisation levels; results are similar for the long-short factor picks with figures below financialisation levels in all cases and very close to pre-financialisation standards for the term structure and liquidity factor picks. The performance of the market factor on the US complex declines below financialisation levels (table \ref{tab:co-movement-time-reg-US-facts-US}) with the pattern similar albeit stronger when the factor is built from the sets of factor picks with a decline in performance on the US complex (table \ref{tab:co-movement-time-reg-US-facts-picks}) to almost pre-financialisation levels when constructed with its own set of picks as well as with those of the long-short factors, in particular the term structure and liquidity factors. Table \ref{tab:co-movement-time-reg-UK-facts-picks} provides further evidence of decoupling, in this case between US and non-US traded commodities with a sharp decline in performance of the market factor constructed with any set of picks on the UK complex below financialisation, down to almost pre-financialisation levels when constructed with the sets of long-short factors picks.

\medskip\setlength{\parindent}{0pt}

Taken together the patterns of results over the crisis and post-crisis periods echo those in Basu and Bauthéac (2021) in suggesting that the exogenous shock delivered by the financial crisis combined with the accommodative monetary policy that ensued seem to have initiated a reversion toward pre-financialisation fundamentals in the commodity futures complex.

\newpage

\hypertarget{conclusion}{%
\section{Conclusion}\label{conclusion}}

In the early 2000s, against a backdrop of a low yield environment and poor stock market performance, a combination of financial innovations\footnote{Commodity price indexes and the ad-hoc financial instruments enabling investment in them were the main financial innovations which allowed large global banks to offer commodity investment products to institutional and retail investors. In 1991 Goldman Sachs created the S\&P-GSCI which provides investors with buy-side exposure to commodities via the OTC swap market and thus without having to participate in the formal futures markets with their position limit restrictions. At this stage, these restrictions still applied to the issuing institutions though, as they hedged the corresponding commodity swap exposure in the futures markets. The first commodity-based ETFs were created through buying physical precious metals with gold and silver ETFs offered as early as 2002/2003. The regulatory hurdle here related to the licensing of commodity trading professionals. Typically, investors had to sign a statement with their broker stating that they understood the risks of commodity investments; a rather inconvenient paperwork for a product designed to trade like a stock, as set forth by a number of industry players at the time.} and regulatory changes\footnote{In 2000, the Commodity Futures Modernization Act (CFMA) granted non-agricultural commodity futures statutory exemption from regulation (``Enron loophole''). It still required agricultural commodity derivatives be traded on a CFTC-regulated exchange however. Eventually, the CFTC classified swap dealers as ``bona fide'' hedgers, granting them position limits exemption (``swap dealer loophole''). In 2005, the CFTC waived the rule that required commodity investors to sign a statement saying they understood the risks, letting the funds replace it with their prospectus. The regulatory bottleneck that prevented the large-scale expansion of commodity index investment was no more.} led to large inflows of institutional capital into the commodity futures markets. This process known as ``financialisation'' spurred a heated public policy debate about whether the ensuing increase in open interest and trading volume in commodity futures exerted upward pressure on prices.\\
The debate spread to the legislative sphere, with the US senate launching formal investigations, and was thrust onto the academic community as a matter of urgency. Perhaps in response to this, most of the early studies focused on the more mechanical effects of financialisation in individual markets, relying on commodity specific causality and correlation-based analysis. The contention eventually triggered legislative action and new position limits were introduced in a number of grains and energy futures markets.

\medskip\setlength{\parindent}{0pt}

With the benefit of hindsight, this focus appears to have been too narrow. As our results suggests, financialisation was a phenomenon global in nature and studying the direct effects of increased market participation in single markets thus seems amiss in this context.\\
The results of our analysis further suggest that financialisation had cross-sectional effects on market fundamentals across the commodity complex beyond energy and agricultural commodities and in fact strongest for the metals sector at the global level; an aspect largely overlooked in the original approach that we believe the existing literature has not fully analysed.\\
The onset of the financial crisis and the monetary policy regimes that followed, on the other hand, also appear to have induced significant changes in commodity pricing dynamics. In contrast with financialisation, our results suggest that the crisis and its aftermath have delivered an exogenous shock across the entire cross-section of liquid commodities and seem to have triggered a motion of reversion to legacy pre-financialisation fundamentals.

\newpage\loadgeometry{tables}

\hypertarget{tables}{%
\section{Tables}\label{tables}}

\begingroup\fontsize{9}{11}\selectfont

\begin{longtable}[t]{>{}llrrrr}
\caption{\label{tab:co-movement-stats-no-regimes}This table shows mean returns and volatility (sd) for the twenty four individual US commodities and the six LME metals considered in the study as well as for two equally weighted portfolios formed from the US commodities and the LME metals respectively across the four periods of interest (past: 1997-2003; financialisation: 2004-2008; crisis: 2008-2013; post-crisis: 2013-2018). Mean values significant at the 1\%, 5\% and 10\% level are marked with ***, ** and * respectively. The results are discussed in section \ref{co-movement-results}.}\\
\toprule
asset & estimate & past & financialisation & crisis & post-crisis\\
\midrule
\endfirsthead
\caption[]{\textit{(continued)}}\\
\toprule
asset & estimate & past & financialisation & crisis & post-crisis\\
\midrule
\endhead

\endfoot
\bottomrule
\endlastfoot
\addlinespace[0.3em]
\multicolumn{6}{l}{\textbf{individual commodities}}\\
\hspace{1em} & mean & 3.48\% & 21.37\% & 6.89\% & -4.51\%\\
\nopagebreak
\hspace{1em}\multirow[t]{-2}{*}{\raggedright\arraybackslash \textbf{Corn-\#2 yellow (XCBT)}} & sd & 23\% & 29.54\% & 36.15\% & 21.8\%\\
\cmidrule{1-6}\pagebreak[0]
\hspace{1em} & mean & 5.49\% & 21.62\% & 8.89\% & -0.87\%\\
\nopagebreak
\hspace{1em}\multirow[t]{-2}{*}{\raggedright\arraybackslash \textbf{Oats (XCBT)}} & sd & 32.22\% & 31.04\% & 34.67\% & 29.5\%\\
\cmidrule{1-6}\pagebreak[0]
\hspace{1em} & mean & 7.29\% & 12.06\% & 8.12\% & -0.33\%\\
\nopagebreak
\hspace{1em}\multirow[t]{-2}{*}{\raggedright\arraybackslash \textbf{Soybean meal (XCBT)}} & sd & 26.34\% & 32.55\% & 31.78\% & 25.11\%\\
\cmidrule{1-6}\pagebreak[0]
\hspace{1em} & mean & 6.21\% & 15.57\% & 4.04\% & -7.39\%\\
\nopagebreak
\hspace{1em}\multirow[t]{-2}{*}{\raggedright\arraybackslash \textbf{Soybean oil (XCBT)}} & sd & 21.43\% & 28.35\% & 25.75\% & 18.95\%\\
\cmidrule{1-6}\pagebreak[0]
\hspace{1em} & mean & 6.89\% & 13.36\% & 6.19\% & -3.82\%\\
\nopagebreak
\hspace{1em}\multirow[t]{-2}{*}{\raggedright\arraybackslash \textbf{Soybeans (XCBT)}} & sd & 21.47\% & 30.04\% & 28\% & 21.22\%\\
\cmidrule{1-6}\pagebreak[0]
\hspace{1em} & mean & 5.07\% & 17.63\% & 6.68\% & -1.58\%\\
\nopagebreak
\hspace{1em}\multirow[t]{-2}{*}{\raggedright\arraybackslash \textbf{Wheat-SRW (XCBT)}} & sd & 24.93\% & 33.03\% & 37.62\% & 26.58\%\\
\cmidrule{1-6}\pagebreak[0]
\hspace{1em} & mean & 0.39\% & 7.43\% & 7.13\% & 1.87\%\\
\nopagebreak
\hspace{1em}\multirow[t]{-2}{*}{\raggedright\arraybackslash \textbf{Cattle-feeder (XCME)}} & sd & 13.23\% & 15.16\% & 15.24\% & 18.69\%\\
\cmidrule{1-6}\pagebreak[0]
\hspace{1em} & mean & 3.45\% & 8.77\% & 4.13\% & 1.92\%\\
\nopagebreak
\hspace{1em}\multirow[t]{-2}{*}{\raggedright\arraybackslash \textbf{Cattle-live (XCME)}} & sd & 17.1\% & 17.43\% & 16.32\% & 18.3\%\\
\cmidrule{1-6}\pagebreak[0]
\hspace{1em} & mean & 0.73\% & 8.4\% & 12.69\% & -1.09\%\\
\nopagebreak
\hspace{1em}\multirow[t]{-2}{*}{\raggedright\arraybackslash \textbf{Lean hogs (XCME)}} & sd & 38.33\% & 31.89\% & 30.37\% & 36.94\%\\
\cmidrule{1-6}\pagebreak[0]
\hspace{1em} & mean & 3.66\% & 16.35\% & 1.39\% & 3.65\%\\
\nopagebreak
\hspace{1em}\multirow[t]{-2}{*}{\raggedright\arraybackslash \textbf{Cocoa (IFUS)}} & sd & 32.66\% & 29.99\% & 31.03\% & 24.69\%\\
\cmidrule{1-6}\pagebreak[0]
\hspace{1em} & mean & -5.91\% & 20.38\% & 2.2\% & 1.99\%\\
\nopagebreak
\hspace{1em}\multirow[t]{-2}{*}{\raggedright\arraybackslash \textbf{Coffee-C (IFUS)}} & sd & 43.84\% & 32.3\% & 30.17\% & 31.86\%\\
\cmidrule{1-6}\pagebreak[0]
\hspace{1em} & mean & 3.24\% & 0.95\% & 12.53\% & 0.5\%\\
\nopagebreak
\hspace{1em}\multirow[t]{-2}{*}{\raggedright\arraybackslash \textbf{Cotton-\#2 (IFUS)}} & sd & 26.38\% & 29.01\% & 33.26\% & 20.61\%\\
\cmidrule{1-6}\pagebreak[0]
\hspace{1em} & mean & 1.9\% & -3.58\% & 11.74\% & 5.06\%\\
\nopagebreak
\hspace{1em}\multirow[t]{-2}{*}{\raggedright\arraybackslash \textbf{Lumber (XCME)}} & sd & 30.58\% & 29.39\% & 36.31\% & 25.1\%\\
\cmidrule{1-6}\pagebreak[0]
\hspace{1em} & mean & 1.67\% & 13.87\% & 14.4\% & 3.8\%\\
\nopagebreak
\hspace{1em}\multirow[t]{-2}{*}{\raggedright\arraybackslash \textbf{Orange juice (IFUS)}} & sd & 29.5\% & 33.01\% & 34.43\% & 29.37\%\\
\cmidrule{1-6}\pagebreak[0]
\hspace{1em} & mean & -4.56\% & 24.87\% & 11.82\% & -1.16\%\\
\nopagebreak
\hspace{1em}\multirow[t]{-2}{*}{\raggedright\arraybackslash \textbf{Sugar-\#11 (IFUS)}} & sd & 34.88\% & 33.25\% & 37.9\% & 28.72\%\\
\cmidrule{1-6}\pagebreak[0]
\hspace{1em} & mean & 35.38\% & 16.25\% & 0.93\% & 8.33\%\\
\nopagebreak
\hspace{1em}\multirow[t]{-2}{*}{\raggedright\arraybackslash \textbf{Natural gas (XNYM)}} & sd & 61.55\% & 54.6\% & 53.58\% & 44.58\%\\
\cmidrule{1-6}\pagebreak[0]
\hspace{1em} & mean & 15.43\% & *28.74\% & 9.67\% & -5.64\%\\
\nopagebreak
\hspace{1em}\multirow[t]{-2}{*}{\raggedright\arraybackslash \textbf{Crude oil-WTI (XNYM)}} & sd & 39.87\% & 32.56\% & 42.84\% & 34.69\%\\
\cmidrule{1-6}\pagebreak[0]
\hspace{1em} & mean & 16.65\% & 30.95\% & 10.59\% & -5.37\%\\
\nopagebreak
\hspace{1em}\multirow[t]{-2}{*}{\raggedright\arraybackslash \textbf{Gasoline (XNYM)}} & sd & 42.77\% & 41.76\% & 40.07\% & 37.09\%\\
\cmidrule{1-6}\pagebreak[0]
\hspace{1em} & mean & 16.31\% & *29.88\% & 6.83\% & -3.44\%\\
\nopagebreak
\hspace{1em}\multirow[t]{-2}{*}{\raggedright\arraybackslash \textbf{Heating oil (XNYM)}} & sd & 40.75\% & 35.09\% & 33.17\% & 30.56\%\\
\cmidrule{1-6}\pagebreak[0]
\hspace{1em} & mean & 1.45\% & *28.15\% & 6.19\% & -0.1\%\\
\nopagebreak
\hspace{1em}\multirow[t]{-2}{*}{\raggedright\arraybackslash \textbf{Copper (XCEC)}} & sd & 21.23\% & 31.89\% & 34.91\% & 19.56\%\\
\cmidrule{1-6}\pagebreak[0]
\hspace{1em} & mean & 4.5\% & 14.35\% & 14.01\% & 0.53\%\\
\nopagebreak
\hspace{1em}\multirow[t]{-2}{*}{\raggedright\arraybackslash \textbf{Gold (XCEC)}} & sd & 15.14\% & 18.89\% & 21.46\% & 14.41\%\\
\cmidrule{1-6}\pagebreak[0]
\hspace{1em} & mean & 9.12\% & 9.03\% & *29.12\% & 13.75\%\\
\nopagebreak
\hspace{1em}\multirow[t]{-2}{*}{\raggedright\arraybackslash \textbf{Palladium (XNYM)}} & sd & 38.27\% & 32.71\% & 36.14\% & 25.38\%\\
\cmidrule{1-6}\pagebreak[0]
\hspace{1em} & mean & 12.93\% & 10.56\% & 7.29\% & -7.92\%\\
\nopagebreak
\hspace{1em}\multirow[t]{-2}{*}{\raggedright\arraybackslash \textbf{Platinum (XNYM)}} & sd & 22.62\% & 22.66\% & 25.53\% & 19.48\%\\
\cmidrule{1-6}\pagebreak[0]
\hspace{1em} & mean & 6.19\% & 17.54\% & 21.96\% & -2.43\%\\
\nopagebreak
\hspace{1em}\multirow[t]{-2}{*}{\raggedright\arraybackslash \textbf{Silver (XCEC)}} & sd & 22\% & 33.92\% & 39.87\% & 24.35\%\\
\cmidrule{1-6}\pagebreak[0]
\hspace{1em} & mean & 1.13\% & 13.78\% & -3.58\% & 3.15\%\\
\nopagebreak
\hspace{1em}\multirow[t]{-2}{*}{\raggedright\arraybackslash \textbf{Aluminium (XLME)}} & sd & 17.6\% & 26.41\% & 26.48\% & 18.56\%\\
\cmidrule{1-6}\pagebreak[0]
\hspace{1em} & mean & 0.45\% & **28.59\% & 5.74\% & 0.03\%\\
\nopagebreak
\hspace{1em}\multirow[t]{-2}{*}{\raggedright\arraybackslash \textbf{Copper (XLME)}} & sd & 19.25\% & 31.09\% & 34.27\% & 18.92\%\\
\cmidrule{1-6}\pagebreak[0]
\hspace{1em} & mean & 3.83\% & 28.99\% & 10.74\% & 2.07\%\\
\nopagebreak
\hspace{1em}\multirow[t]{-2}{*}{\raggedright\arraybackslash \textbf{Lead (XLME)}} & sd & 19.64\% & 41.74\% & 41.05\% & 23.26\%\\
\cmidrule{1-6}\pagebreak[0]
\hspace{1em} & mean & 19.31\% & 12.5\% & 3.52\% & 0.31\%\\
\nopagebreak
\hspace{1em}\multirow[t]{-2}{*}{\raggedright\arraybackslash \textbf{Nickel-primary (XLME)}} & sd & 32.26\% & 44.27\% & 41.45\% & 28.98\%\\
\cmidrule{1-6}\pagebreak[0]
\hspace{1em} & mean & 4.07\% & *29.1\% & 7.46\% & 1.68\%\\
\nopagebreak
\hspace{1em}\multirow[t]{-2}{*}{\raggedright\arraybackslash \textbf{Tin-refined (XLME)}} & sd & 15.38\% & 31.71\% & 35.61\% & 18.79\%\\
\cmidrule{1-6}\pagebreak[0]
\hspace{1em} & mean & -4.52\% & 20.13\% & 7.62\% & 9.07\%\\
\nopagebreak
\hspace{1em}\multirow[t]{-2}{*}{\raggedright\arraybackslash \textbf{Zinc (XLME)}} & sd & 21.73\% & 38.77\% & 36.17\% & 23.25\%\\
\cmidrule{1-6}\pagebreak[0]
\addlinespace[0.3em]
\multicolumn{6}{l}{\textbf{equally weighted portfolios}}\\
\hspace{1em} & mean & *6.89\% & **15.89\% & 9.39\% & -0.18\%\\
\nopagebreak
\hspace{1em}\multirow[t]{-2}{*}{\raggedright\arraybackslash \textbf{US commodities}} & sd & 10\% & 13.62\% & 17.58\% & 9.86\%\\
\cmidrule{1-6}\pagebreak[0]
\hspace{1em} & mean & 2.47\% & *21.77\% & 5.25\% & 2.72\%\\
\nopagebreak
\hspace{1em}\multirow[t]{-2}{*}{\raggedright\arraybackslash \textbf{GB commodities}} & sd & 15.66\% & 26.91\% & 30.24\% & 16.19\%\\*
\end{longtable}
\endgroup{}

\newpage

\begingroup\fontsize{9}{11}\selectfont

\begin{longtable}[t]{>{}lllrrrr}
\caption{\label{tab:co-movement-stats-regimes}This table shows mean returns and volatility (sd) for the twenty four individual US commodities and the six LME metals considered in the study as well as for two equally weighted portfolios formed from the US commodities and the LME metals respectively across the four periods of interest (past: 1997-2003; financialisation: 2004-2008; crisis: 2008-2013; post-crisis: 2013-2018). Figures are shown independently for phases of aggregate backwardation (aggregate CHP $\leq$ period median) and aggregate contango (aggregate CHP > period median). Mean values significant at the 1\%, 5\% and 10\% level are marked with ***, ** and * respectively. Aggregate CHP construction and corresponding regime definitions are discussed in section \ref{co-movement-methods} while the results are discussed in section \ref{co-movement-results}.}\\
\toprule
asset & regime & estimate & past & financialisation & crisis & post-crisis\\
\midrule
\endfirsthead
\caption[]{\textit{(continued)}}\\
\toprule
asset & regime & estimate & past & financialisation & crisis & post-crisis\\
\midrule
\endhead

\endfoot
\bottomrule
\endlastfoot
\addlinespace[0.3em]
\multicolumn{7}{l}{\textbf{individual commodities}}\\
\hspace{1em} &  & mean & -4.86\% & 19.26\% & 20.48\% & -8.74\%\\
\nopagebreak
\hspace{1em} & \multirow[t]{-2}{*}{\raggedright\arraybackslash backwardation} & sd & 23.07\% & 29.14\% & 31.59\% & 21\%\\
\nopagebreak
\hspace{1em} &  & mean & 11.3\% & 23.54\% & -4.43\% & 0.78\%\\
\nopagebreak
\hspace{1em}\multirow[t]{-4}{*}{\raggedright\arraybackslash \textbf{Corn-\#2 yellow (XCBT)}} & \multirow[t]{-2}{*}{\raggedright\arraybackslash contango} & sd & 23\% & 30.01\% & 40.14\% & 22.68\%\\
\cmidrule{1-7}\pagebreak[0]
\hspace{1em} &  & mean & -5.39\% & 7.82\% & 7.85\% & 2.56\%\\
\nopagebreak
\hspace{1em} & \multirow[t]{-2}{*}{\raggedright\arraybackslash backwardation} & sd & 29.02\% & 33.82\% & 31.43\% & 31.55\%\\
\nopagebreak
\hspace{1em} &  & mean & 16.95\% & *35.38\% & 11.68\% & -3.17\%\\
\nopagebreak
\hspace{1em}\multirow[t]{-4}{*}{\raggedright\arraybackslash \textbf{Oats (XCBT)}} & \multirow[t]{-2}{*}{\raggedright\arraybackslash contango} & sd & 35.22\% & 28.1\% & 37.7\% & 27.5\%\\
\cmidrule{1-7}\pagebreak[0]
\hspace{1em} &  & mean & 13.15\% & 8.22\% & 16.04\% & -3.52\%\\
\nopagebreak
\hspace{1em} & \multirow[t]{-2}{*}{\raggedright\arraybackslash backwardation} & sd & 27.77\% & 33.06\% & 25.7\% & 25.1\%\\
\nopagebreak
\hspace{1em} &  & mean & 1.52\% & 15.91\% & 2.32\% & 2.22\%\\
\nopagebreak
\hspace{1em}\multirow[t]{-4}{*}{\raggedright\arraybackslash \textbf{Soybean meal (XCBT)}} & \multirow[t]{-2}{*}{\raggedright\arraybackslash contango} & sd & 24.94\% & 32.13\% & 36.85\% & 25.32\%\\
\cmidrule{1-7}\pagebreak[0]
\hspace{1em} &  & mean & 14.79\% & 14.62\% & 20.34\% & -9.44\%\\
\nopagebreak
\hspace{1em} & \multirow[t]{-2}{*}{\raggedright\arraybackslash backwardation} & sd & 21.35\% & 28.94\% & 22.33\% & 18.19\%\\
\nopagebreak
\hspace{1em} &  & mean & -2.68\% & 16.62\% & -10.01\% & -4.12\%\\
\nopagebreak
\hspace{1em}\multirow[t]{-4}{*}{\raggedright\arraybackslash \textbf{Soybean oil (XCBT)}} & \multirow[t]{-2}{*}{\raggedright\arraybackslash contango} & sd & 21.55\% & 27.91\% & 28.65\% & 19.79\%\\
\cmidrule{1-7}\pagebreak[0]
\hspace{1em} &  & mean & 13.69\% & 0.93\% & 13.42\% & -11.45\%\\
\nopagebreak
\hspace{1em} & \multirow[t]{-2}{*}{\raggedright\arraybackslash backwardation} & sd & 21.93\% & 34.58\% & 24.39\% & 22.04\%\\
\nopagebreak
\hspace{1em} &  & mean & -0.16\% & 25.71\% & 0.96\% & 3.75\%\\
\nopagebreak
\hspace{1em}\multirow[t]{-4}{*}{\raggedright\arraybackslash \textbf{Soybeans (XCBT)}} & \multirow[t]{-2}{*}{\raggedright\arraybackslash contango} & sd & 21.09\% & 24.74\% & 31.11\% & 20.56\%\\
\cmidrule{1-7}\pagebreak[0]
\hspace{1em} &  & mean & -2.7\% & 19.02\% & 0.98\% & 0.44\%\\
\nopagebreak
\hspace{1em} & \multirow[t]{-2}{*}{\raggedright\arraybackslash backwardation} & sd & 25.79\% & 36\% & 37.53\% & 26.57\%\\
\nopagebreak
\hspace{1em} &  & mean & 11.78\% & 16.29\% & 14.57\% & -2.77\%\\
\nopagebreak
\hspace{1em}\multirow[t]{-4}{*}{\raggedright\arraybackslash \textbf{Wheat-SRW (XCBT)}} & \multirow[t]{-2}{*}{\raggedright\arraybackslash contango} & sd & 24.04\% & 29.84\% & 37.73\% & 26.72\%\\
\cmidrule{1-7}\pagebreak[0]
\hspace{1em} &  & mean & 2.31\% & **20.52\% & 7.51\% & 6.58\%\\
\nopagebreak
\hspace{1em} & \multirow[t]{-2}{*}{\raggedright\arraybackslash backwardation} & sd & 12.79\% & 14.93\% & 13.39\% & 20.14\%\\
\nopagebreak
\hspace{1em} &  & mean & -0.81\% & -5.49\% & 7.28\% & -3.12\%\\
\nopagebreak
\hspace{1em}\multirow[t]{-4}{*}{\raggedright\arraybackslash \textbf{Cattle-feeder (XCME)}} & \multirow[t]{-2}{*}{\raggedright\arraybackslash contango} & sd & 13.67\% & 15.35\% & 16.91\% & 17.23\%\\
\cmidrule{1-7}\pagebreak[0]
\hspace{1em} &  & mean & 8.39\% & 14.2\% & 5.87\% & -2.07\%\\
\nopagebreak
\hspace{1em} & \multirow[t]{-2}{*}{\raggedright\arraybackslash backwardation} & sd & 16.65\% & 17.91\% & 14.97\% & 19.9\%\\
\nopagebreak
\hspace{1em} &  & mean & -0.34\% & 3.43\% & 2.7\% & 5.82\%\\
\nopagebreak
\hspace{1em}\multirow[t]{-4}{*}{\raggedright\arraybackslash \textbf{Cattle-live (XCME)}} & \multirow[t]{-2}{*}{\raggedright\arraybackslash contango} & sd & 17.49\% & 16.99\% & 17.57\% & 16.66\%\\
\cmidrule{1-7}\pagebreak[0]
\hspace{1em} &  & mean & 12.34\% & 16.3\% & 18.23\% & 19.24\%\\
\nopagebreak
\hspace{1em} & \multirow[t]{-2}{*}{\raggedright\arraybackslash backwardation} & sd & 37.6\% & 35.73\% & 27.83\% & 35.61\%\\
\nopagebreak
\hspace{1em} &  & mean & -10.5\% & 0.6\% & 7.5\% & -20.66\%\\
\nopagebreak
\hspace{1em}\multirow[t]{-4}{*}{\raggedright\arraybackslash \textbf{Lean hogs (XCME)}} & \multirow[t]{-2}{*}{\raggedright\arraybackslash contango} & sd & 39.22\% & 27.67\% & 32.78\% & 38.28\%\\
\cmidrule{1-7}\pagebreak[0]
\hspace{1em} &  & mean & 4.12\% & 9.25\% & -9.56\% & -2.98\%\\
\nopagebreak
\hspace{1em} & \multirow[t]{-2}{*}{\raggedright\arraybackslash backwardation} & sd & 33.57\% & 31.05\% & 26.93\% & 25.86\%\\
\nopagebreak
\hspace{1em} &  & mean & 3.67\% & 23.43\% & 12.92\% & 10.29\%\\
\nopagebreak
\hspace{1em}\multirow[t]{-4}{*}{\raggedright\arraybackslash \textbf{Cocoa (IFUS)}} & \multirow[t]{-2}{*}{\raggedright\arraybackslash contango} & sd & 31.81\% & 28.91\% & 34.66\% & 23.57\%\\
\cmidrule{1-7}\pagebreak[0]
\hspace{1em} &  & mean & 4.51\% & 26.49\% & 2.66\% & 14.36\%\\
\nopagebreak
\hspace{1em} & \multirow[t]{-2}{*}{\raggedright\arraybackslash backwardation} & sd & 48.82\% & 32.6\% & 29.92\% & 31.52\%\\
\nopagebreak
\hspace{1em} &  & mean & -16.87\% & 14.38\% & 3.4\% & -10.24\%\\
\nopagebreak
\hspace{1em}\multirow[t]{-4}{*}{\raggedright\arraybackslash \textbf{Coffee-C (IFUS)}} & \multirow[t]{-2}{*}{\raggedright\arraybackslash contango} & sd & 38.35\% & 32.06\% & 30.22\% & 32.24\%\\
\cmidrule{1-7}\pagebreak[0]
\hspace{1em} &  & mean & 22.64\% & 7.85\% & 35.53\% & 10.84\%\\
\nopagebreak
\hspace{1em} & \multirow[t]{-2}{*}{\raggedright\arraybackslash backwardation} & sd & 26.02\% & 29.71\% & 33.94\% & 21.33\%\\
\nopagebreak
\hspace{1em} &  & mean & -16.46\% & -5.92\% & -9.36\% & -8.91\%\\
\nopagebreak
\hspace{1em}\multirow[t]{-4}{*}{\raggedright\arraybackslash \textbf{Cotton-\#2 (IFUS)}} & \multirow[t]{-2}{*}{\raggedright\arraybackslash contango} & sd & 26.77\% & 28.35\% & 32.59\% & 19.95\%\\
\cmidrule{1-7}\pagebreak[0]
\hspace{1em} &  & mean & -8.48\% & -2.37\% & 29.11\% & **26.99\%\\
\nopagebreak
\hspace{1em} & \multirow[t]{-2}{*}{\raggedright\arraybackslash backwardation} & sd & 30.43\% & 30.72\% & 35.45\% & 21.26\%\\
\nopagebreak
\hspace{1em} &  & mean & 11.68\% & -4.77\% & -5.04\% & -17.14\%\\
\nopagebreak
\hspace{1em}\multirow[t]{-4}{*}{\raggedright\arraybackslash \textbf{Lumber (XCME)}} & \multirow[t]{-2}{*}{\raggedright\arraybackslash contango} & sd & 30.56\% & 28.2\% & 37.23\% & 28.45\%\\
\cmidrule{1-7}\pagebreak[0]
\hspace{1em} &  & mean & -2.38\% & -2.22\% & 18.24\% & 18.64\%\\
\nopagebreak
\hspace{1em} & \multirow[t]{-2}{*}{\raggedright\arraybackslash backwardation} & sd & 28.01\% & 33.01\% & 30.09\% & 28.64\%\\
\nopagebreak
\hspace{1em} &  & mean & 5.83\% & 29.93\% & 12.29\% & -8.64\%\\
\nopagebreak
\hspace{1em}\multirow[t]{-4}{*}{\raggedright\arraybackslash \textbf{Orange juice (IFUS)}} & \multirow[t]{-2}{*}{\raggedright\arraybackslash contango} & sd & 31.05\% & 33.1\% & 38.06\% & 30.11\%\\
\cmidrule{1-7}\pagebreak[0]
\hspace{1em} &  & mean & 11.98\% & 29.51\% & 2.7\% & -8.07\%\\
\nopagebreak
\hspace{1em} & \multirow[t]{-2}{*}{\raggedright\arraybackslash backwardation} & sd & 33.23\% & 33.6\% & 41.29\% & 27.69\%\\
\nopagebreak
\hspace{1em} &  & mean & -20.49\% & 20.35\% & 21.47\% & 4.7\%\\
\nopagebreak
\hspace{1em}\multirow[t]{-4}{*}{\raggedright\arraybackslash \textbf{Sugar-\#11 (IFUS)}} & \multirow[t]{-2}{*}{\raggedright\arraybackslash contango} & sd & 36.48\% & 32.98\% & 34.09\% & 29.83\%\\
\cmidrule{1-7}\pagebreak[0]
\hspace{1em} &  & mean & **85.84\% & 4.76\% & -2.32\% & 18.33\%\\
\nopagebreak
\hspace{1em} & \multirow[t]{-2}{*}{\raggedright\arraybackslash backwardation} & sd & 63.46\% & 47.3\% & 48.85\% & 42.59\%\\
\nopagebreak
\hspace{1em} &  & mean & -11.98\% & 27.84\% & 4.77\% & 3.25\%\\
\nopagebreak
\hspace{1em}\multirow[t]{-4}{*}{\raggedright\arraybackslash \textbf{Natural gas (XNYM)}} & \multirow[t]{-2}{*}{\raggedright\arraybackslash contango} & sd & 59.76\% & 61.3\% & 58.18\% & 46.41\%\\
\cmidrule{1-7}\pagebreak[0]
\hspace{1em} &  & mean & 29.17\% & *35.61\% & 11.47\% & 14.24\%\\
\nopagebreak
\hspace{1em} & \multirow[t]{-2}{*}{\raggedright\arraybackslash backwardation} & sd & 36.74\% & 32.09\% & 28.19\% & 26.94\%\\
\nopagebreak
\hspace{1em} &  & mean & 2.21\% & 22.04\% & 9.95\% & -24.85\%\\
\nopagebreak
\hspace{1em}\multirow[t]{-4}{*}{\raggedright\arraybackslash \textbf{Crude oil-WTI (XNYM)}} & \multirow[t]{-2}{*}{\raggedright\arraybackslash contango} & sd & 42.74\% & 33.1\% & 53.7\% & 41.08\%\\
\cmidrule{1-7}\pagebreak[0]
\hspace{1em} &  & mean & 39\% & 19.82\% & 23.98\% & 9.69\%\\
\nopagebreak
\hspace{1em} & \multirow[t]{-2}{*}{\raggedright\arraybackslash backwardation} & sd & 42.47\% & 39.38\% & 31.41\% & 32.1\%\\
\nopagebreak
\hspace{1em} &  & mean & -5.22\% & 42.13\% & -0.09\% & -19.47\%\\
\nopagebreak
\hspace{1em}\multirow[t]{-4}{*}{\raggedright\arraybackslash \textbf{Gasoline (XNYM)}} & \multirow[t]{-2}{*}{\raggedright\arraybackslash contango} & sd & 43.14\% & 44.13\% & 47.11\% & 41.71\%\\
\cmidrule{1-7}\pagebreak[0]
\hspace{1em} &  & mean & 36.32\% & 29.41\% & 25.56\% & 15.16\%\\
\nopagebreak
\hspace{1em} & \multirow[t]{-2}{*}{\raggedright\arraybackslash backwardation} & sd & 41.37\% & 33.33\% & 25.95\% & 24.62\%\\
\nopagebreak
\hspace{1em} &  & mean & -2.74\% & 30.5\% & -10.77\% & -21.62\%\\
\nopagebreak
\hspace{1em}\multirow[t]{-4}{*}{\raggedright\arraybackslash \textbf{Heating oil (XNYM)}} & \multirow[t]{-2}{*}{\raggedright\arraybackslash contango} & sd & 40.19\% & 36.98\% & 39.16\% & 35.69\%\\
\cmidrule{1-7}\pagebreak[0]
\hspace{1em} &  & mean & 3.14\% & 22.17\% & 16.53\% & 12.1\%\\
\nopagebreak
\hspace{1em} & \multirow[t]{-2}{*}{\raggedright\arraybackslash backwardation} & sd & 20.71\% & 30.05\% & 25.73\% & 18.16\%\\
\nopagebreak
\hspace{1em} &  & mean & -0.28\% & 34.17\% & -3.46\% & -12.15\%\\
\nopagebreak
\hspace{1em}\multirow[t]{-4}{*}{\raggedright\arraybackslash \textbf{Copper (XCEC)}} & \multirow[t]{-2}{*}{\raggedright\arraybackslash contango} & sd & 21.78\% & 33.69\% & 42.28\% & 20.87\%\\
\cmidrule{1-7}\pagebreak[0]
\hspace{1em} &  & mean & 9.19\% & 13.37\% & 9.37\% & 4.21\%\\
\nopagebreak
\hspace{1em} & \multirow[t]{-2}{*}{\raggedright\arraybackslash backwardation} & sd & 16.89\% & 19.2\% & 17.03\% & 12.36\%\\
\nopagebreak
\hspace{1em} &  & mean & -0.01\% & 15.37\% & 19.1\% & -3.1\%\\
\nopagebreak
\hspace{1em}\multirow[t]{-4}{*}{\raggedright\arraybackslash \textbf{Gold (XCEC)}} & \multirow[t]{-2}{*}{\raggedright\arraybackslash contango} & sd & 13.24\% & 18.58\% & 25.21\% & 16.23\%\\
\cmidrule{1-7}\pagebreak[0]
\hspace{1em} &  & mean & **49.79\% & 24.47\% & 30.01\% & **31.93\%\\
\nopagebreak
\hspace{1em} & \multirow[t]{-2}{*}{\raggedright\arraybackslash backwardation} & sd & 34.98\% & 34.23\% & 33.49\% & 23.82\%\\
\nopagebreak
\hspace{1em} &  & mean & -30.19\% & -6.3\% & 30.58\% & -4.88\%\\
\nopagebreak
\hspace{1em}\multirow[t]{-4}{*}{\raggedright\arraybackslash \textbf{Palladium (XNYM)}} & \multirow[t]{-2}{*}{\raggedright\arraybackslash contango} & sd & 41.22\% & 31.2\% & 38.71\% & 26.8\%\\
\cmidrule{1-7}\pagebreak[0]
\hspace{1em} &  & mean & **29.46\% & *25.07\% & 10.46\% & 1.17\%\\
\nopagebreak
\hspace{1em} & \multirow[t]{-2}{*}{\raggedright\arraybackslash backwardation} & sd & 22.1\% & 22.51\% & 21.24\% & 18.04\%\\
\nopagebreak
\hspace{1em} &  & mean & -3.22\% & -3.81\% & 8.04\% & -17.28\%\\
\nopagebreak
\hspace{1em}\multirow[t]{-4}{*}{\raggedright\arraybackslash \textbf{Platinum (XNYM)}} & \multirow[t]{-2}{*}{\raggedright\arraybackslash contango} & sd & 23.12\% & 22.81\% & 28.75\% & 20.8\%\\
\cmidrule{1-7}\pagebreak[0]
\hspace{1em} &  & mean & 5.98\% & 14.29\% & 19.66\% & 3.44\%\\
\nopagebreak
\hspace{1em} & \multirow[t]{-2}{*}{\raggedright\arraybackslash backwardation} & sd & 21.07\% & 32.5\% & 37.17\% & 20.3\%\\
\nopagebreak
\hspace{1em} &  & mean & 6.57\% & 20.83\% & 26.77\% & -8.09\%\\
\nopagebreak
\hspace{1em}\multirow[t]{-4}{*}{\raggedright\arraybackslash \textbf{Silver (XCEC)}} & \multirow[t]{-2}{*}{\raggedright\arraybackslash contango} & sd & 22.93\% & 35.34\% & 42.44\% & 27.77\%\\
\cmidrule{1-7}\pagebreak[0]
\hspace{1em} &  & mean & 2.56\% & 14.78\% & 16.25\% & **23.68\%\\
\nopagebreak
\hspace{1em} & \multirow[t]{-2}{*}{\raggedright\arraybackslash backwardation} & sd & 14.82\% & 25.57\% & 24.94\% & 18.94\%\\
\nopagebreak
\hspace{1em} &  & mean & -0.38\% & 12.86\% & -23.04\% & -16.74\%\\
\nopagebreak
\hspace{1em}\multirow[t]{-4}{*}{\raggedright\arraybackslash \textbf{Aluminium (XLME)}} & \multirow[t]{-2}{*}{\raggedright\arraybackslash contango} & sd & 20.13\% & 27.29\% & 28.01\% & 18.23\%\\
\cmidrule{1-7}\pagebreak[0]
\hspace{1em} &  & mean & 1.54\% & 25.22\% & 18.66\% & 11.99\%\\
\nopagebreak
\hspace{1em} & \multirow[t]{-2}{*}{\raggedright\arraybackslash backwardation} & sd & 17.98\% & 29.01\% & 25.53\% & 17.92\%\\
\nopagebreak
\hspace{1em} &  & mean & -0.95\% & 32\% & -6.78\% & -11.89\%\\
\nopagebreak
\hspace{1em}\multirow[t]{-4}{*}{\raggedright\arraybackslash \textbf{Copper (XLME)}} & \multirow[t]{-2}{*}{\raggedright\arraybackslash contango} & sd & 20.41\% & 32.98\% & 41.31\% & 19.86\%\\
\cmidrule{1-7}\pagebreak[0]
\hspace{1em} &  & mean & 7.59\% & 11.21\% & 9.87\% & 17.05\%\\
\nopagebreak
\hspace{1em} & \multirow[t]{-2}{*}{\raggedright\arraybackslash backwardation} & sd & 19.16\% & 39.91\% & 35.35\% & 23.37\%\\
\nopagebreak
\hspace{1em} &  & mean & -0.09\% & *46.48\% & 12.65\% & -13.41\%\\
\nopagebreak
\hspace{1em}\multirow[t]{-4}{*}{\raggedright\arraybackslash \textbf{Lead (XLME)}} & \multirow[t]{-2}{*}{\raggedright\arraybackslash contango} & sd & 20.12\% & 43.16\% & 46.22\% & 23.09\%\\
\cmidrule{1-7}\pagebreak[0]
\hspace{1em} &  & mean & 23.25\% & -39.24\% & 16.3\% & **35.09\%\\
\nopagebreak
\hspace{1em} & \multirow[t]{-2}{*}{\raggedright\arraybackslash backwardation} & sd & 30.23\% & 41.62\% & 34.48\% & 29.13\%\\
\nopagebreak
\hspace{1em} &  & mean & 15.46\% & **63.47\% & -7.67\% & **-34.64\%\\
\nopagebreak
\hspace{1em}\multirow[t]{-4}{*}{\raggedright\arraybackslash \textbf{Nickel-primary (XLME)}} & \multirow[t]{-2}{*}{\raggedright\arraybackslash contango} & sd & 34.28\% & 46.72\% & 47.44\% & 28.6\%\\
\cmidrule{1-7}\pagebreak[0]
\hspace{1em} &  & mean & *15.83\% & 34.62\% & 23.7\% & 4.48\%\\
\nopagebreak
\hspace{1em} & \multirow[t]{-2}{*}{\raggedright\arraybackslash backwardation} & sd & 14.99\% & 31.4\% & 29.09\% & 15.27\%\\
\nopagebreak
\hspace{1em} &  & mean & -7.78\% & 23.88\% & -6.82\% & -1.69\%\\
\nopagebreak
\hspace{1em}\multirow[t]{-4}{*}{\raggedright\arraybackslash \textbf{Tin-refined (XLME)}} & \multirow[t]{-2}{*}{\raggedright\arraybackslash contango} & sd & 15.77\% & 32.08\% & 41.13\% & 21.68\%\\
\cmidrule{1-7}\pagebreak[0]
\hspace{1em} &  & mean & -9.06\% & 2.95\% & 10.05\% & **29.42\%\\
\nopagebreak
\hspace{1em} & \multirow[t]{-2}{*}{\raggedright\arraybackslash backwardation} & sd & 21.15\% & 36.86\% & 33.28\% & 22.12\%\\
\nopagebreak
\hspace{1em} &  & mean & -0.13\% & 37.11\% & 5.43\% & -11.31\%\\
\nopagebreak
\hspace{1em}\multirow[t]{-4}{*}{\raggedright\arraybackslash \textbf{Zinc (XLME)}} & \multirow[t]{-2}{*}{\raggedright\arraybackslash contango} & sd & 22.38\% & 40.51\% & 39.03\% & 24.32\%\\
\cmidrule{1-7}\pagebreak[0]
\addlinespace[0.3em]
\multicolumn{7}{l}{\textbf{equally weighted portfolios}}\\
\hspace{1em} &  & mean & ***15.44\% & *15.66\% & 13.4\% & 6.82\%\\
\nopagebreak
\hspace{1em} & \multirow[t]{-2}{*}{\raggedright\arraybackslash backwardation} & sd & 9.88\% & 14.21\% & 14.83\% & 8.59\%\\
\nopagebreak
\hspace{1em} &  & mean & -1.09\% & *16.16\% & 4.9\% & -6.33\%\\
\nopagebreak
\hspace{1em}\multirow[t]{-4}{*}{\raggedright\arraybackslash \textbf{US commodities}} & \multirow[t]{-2}{*}{\raggedright\arraybackslash contango} & sd & 10.09\% & 13.05\% & 19.98\% & 10.95\%\\
\cmidrule{1-7}\pagebreak[0]
\hspace{1em} &  & mean & 4.41\% & 8.01\% & 15.06\% & **19.95\%\\
\nopagebreak
\hspace{1em} & \multirow[t]{-2}{*}{\raggedright\arraybackslash backwardation} & sd & 14.39\% & 25.92\% & 25.94\% & 15.13\%\\
\nopagebreak
\hspace{1em} &  & mean & 0.34\% & *35.56\% & -5.24\% & -14.05\%\\
\nopagebreak
\hspace{1em}\multirow[t]{-4}{*}{\raggedright\arraybackslash \textbf{GB commodities}} & \multirow[t]{-2}{*}{\raggedright\arraybackslash contango} & sd & 16.87\% & 27.88\% & 34.06\% & 17.1\%\\*
\end{longtable}
\endgroup{}

\newpage

\begingroup\fontsize{9}{11}\selectfont

\begin{longtable}[t]{>{}lllrrrr}
\caption{\label{tab:co-movement-time-reg-US-CHP}This table shows the average time series $R^{2}$ for models where the returns series for the twenty four individual US commodities are independently regressed against relative change in their own commercial hedging pressure (CHP) series as well as against that in aggregate CHP, a market wide measure of CHP which calculation method is discussed in section \ref{co-movement-methods}. Averages are calculated independently across the whole cross-section of US traded commodity assets considered in the study as well as across commodity sectors and subsectors for the four periods of interest (past: 1997-2003; financialisation: 2004-2008; crisis: 2008-2013; post-crisis: 2013-2018). The results are discussed in section \ref{co-movement-results}.}\\
\toprule
regressor & sector & subsector & past & financialisation & crisis & post-crisis\\
\midrule
\endfirsthead
\caption[]{\textit{(continued)}}\\
\toprule
regressor & sector & subsector & past & financialisation & crisis & post-crisis\\
\midrule
\endhead

\endfoot
\bottomrule
\endlastfoot
 & all &  & 25.79\% & 20.72\% & 22.46\% & 27.17\%\\
\nopagebreak
 & agriculturals &  & 25.19\% & 21.6\% & 23.91\% & 28.12\%\\
\nopagebreak
 & energy &  & 25.12\% & 19.45\% & 16\% & 10.96\%\\
\nopagebreak
 & metals & \multirow[t]{-4}{*}{\raggedright\arraybackslash all} & 28.12\% & 19.1\% & 23.27\% & 37.32\%\\
\nopagebreak
 &  & grains & 33.18\% & 27.86\% & 29.15\% & 36.24\%\\
\nopagebreak
 &  & livestock & 6.23\% & 10.53\% & 6.89\% & 4.34\%\\
\nopagebreak
 & \multirow[t]{-3}{*}{\raggedright\arraybackslash agriculturals} & softs & 26.68\% & 20.89\% & 27.19\% & 31.88\%\\
\nopagebreak
 &  & gas & 19.69\% & 15.65\% & 7.36\% & 5.29\%\\
\nopagebreak
 & \multirow[t]{-2}{*}{\raggedright\arraybackslash energy} & petroleum & 26.93\% & 20.71\% & 18.87\% & 12.85\%\\
\nopagebreak
 &  & base & 47.38\% & 25.86\% & 18.07\% & 34.9\%\\
\nopagebreak
\multirow[t]{-11}{*}{\raggedright\arraybackslash \textbf{individual CHP}} & \multirow[t]{-2}{*}{\raggedright\arraybackslash metals} & precious & 23.3\% & 17.41\% & 24.57\% & 37.93\%\\
\cmidrule{1-7}\pagebreak[0]
 & all &  & 5.37\% & 7.12\% & 14.72\% & 7.49\%\\
\nopagebreak
 & agriculturals &  & 4.91\% & 4.5\% & 12.14\% & 6.23\%\\
\nopagebreak
 & energy &  & 3.35\% & 6.27\% & 14.83\% & 2.45\%\\
\nopagebreak
 & metals & \multirow[t]{-4}{*}{\raggedright\arraybackslash all} & 8.36\% & 15.67\% & 22.37\% & 15.33\%\\
\nopagebreak
 &  & grains & 7.25\% & 5.4\% & 19.75\% & 8.45\%\\
\nopagebreak
 &  & livestock & 1.35\% & 0.41\% & 0.52\% & 0.26\%\\
\nopagebreak
 & \multirow[t]{-3}{*}{\raggedright\arraybackslash agriculturals} & softs & 4.34\% & 5.64\% & 10.35\% & 6.98\%\\
\nopagebreak
 &  & gas & 2.65\% & 2.58\% & 2.6\% & 0.27\%\\
\nopagebreak
 & \multirow[t]{-2}{*}{\raggedright\arraybackslash energy} & petroleum & 3.59\% & 7.49\% & 18.9\% & 3.17\%\\
\nopagebreak
 &  & base & 10.12\% & 12.74\% & 23.94\% & 7.67\%\\
\nopagebreak
\multirow[t]{-11}{*}{\raggedright\arraybackslash \textbf{aggregate CHP}} & \multirow[t]{-2}{*}{\raggedright\arraybackslash metals} & precious & 7.92\% & 16.4\% & 21.98\% & 17.25\%\\*
\end{longtable}
\endgroup{}

\newpage

\begingroup\fontsize{9}{11}\selectfont

\begin{longtable}[t]{>{}lllrrrr}
\caption{\label{tab:co-movement-time-reg-US-facts-US}This table shows the average time series $R^{2}$ for models where the returns series for the twenty four individual US commodities are regressed against factor mimicking portfolios independently with the resulting $R^{2}$ averaged across all the twenty four commodities. The asset pool for the factor mimicking portfolio construction includes the twenty four US commodities. For each long-short factor (average CHP: CHP; open interest nearby growth: open interest; and term structure) the models are constructed with the factor itself (factor) as well as both legs independently (long vs. short). The results are reported in order for the market, CHP, open interest and term structure factors, for the four periods of interest (past: 1997-2003; financialisation: 2004-2008; crisis: 2008-2013; post-crisis: 2013-2018) and shown independently for phases of aggregate backwardation (aggregate CHP $\leq$ period median) and aggregate contango (aggregate CHP > period median). Factors as well as aggregate CHP construction and corresponding regime definitions are discussed in section \ref{co-movement-methods} while the results are discussed in section \ref{co-movement-results}.}\\
\toprule
factor & leg & regime & past & financialisation & crisis & post-crisis\\
\midrule
\endfirsthead
\caption[]{\textit{(continued)}}\\
\toprule
factor & leg & regime & past & financialisation & crisis & post-crisis\\
\midrule
\endhead

\endfoot
\bottomrule
\endlastfoot
 &  & all & 11.68\% & 21.6\% & 30.72\% & 15.57\%\\
\nopagebreak
 &  & backwardation & 11.28\% & 23.51\% & 28.95\% & 13.12\%\\
\nopagebreak
\multirow[t]{-3}{*}{\raggedright\arraybackslash \textbf{market}} &  & contango & 12.23\% & 19.96\% & 32.47\% & 17.66\%\\
\nopagebreak
 &  & all & 1.67\% & 8.24\% & 5.67\% & 4.07\%\\
\nopagebreak
 &  & backwardation & 1.84\% & 8.81\% & 5.01\% & 5.74\%\\
\nopagebreak
 & \multirow[t]{-6}{*}{\raggedright\arraybackslash factor} & contango & 2.05\% & 8.25\% & 6.81\% & 3.58\%\\
\nopagebreak
 &  & all & 4.53\% & 15.14\% & 21.88\% & 9.31\%\\
\nopagebreak
 &  & backwardation & 4.77\% & 17.25\% & 21.46\% & 8.07\%\\
\nopagebreak
 & \multirow[t]{-3}{*}{\raggedright\arraybackslash long} & contango & 4.85\% & 13.8\% & 22.56\% & 10.53\%\\
\nopagebreak
 &  & all & 3.02\% & 4\% & 9.1\% & 4.36\%\\
\nopagebreak
 &  & backwardation & 2.51\% & 4.03\% & 7.14\% & 6.24\%\\
\nopagebreak
\multirow[t]{-9}{*}{\raggedright\arraybackslash \textbf{CHP}} & \multirow[t]{-3}{*}{\raggedright\arraybackslash short} & contango & 4\% & 4.81\% & 12.42\% & 3.88\%\\
\cmidrule{1-7}\pagebreak[0]
 &  & all & 2.91\% & 3.84\% & 6.21\% & 5.12\%\\
\nopagebreak
 &  & backwardation & 3.61\% & 2.93\% & 4.99\% & 5.04\%\\
\nopagebreak
 & \multirow[t]{-3}{*}{\raggedright\arraybackslash factor} & contango & 2.5\% & 5.21\% & 7.39\% & 5.41\%\\
\nopagebreak
 &  & all & 5.94\% & 13.15\% & 21.23\% & 8.72\%\\
\nopagebreak
 &  & backwardation & 6.78\% & 14.01\% & 18.83\% & 8.01\%\\
\nopagebreak
 & \multirow[t]{-3}{*}{\raggedright\arraybackslash long} & contango & 5.47\% & 12.78\% & 23.07\% & 9.34\%\\
\nopagebreak
 &  & all & 3.55\% & 7.41\% & 12.39\% & 4.25\%\\
\nopagebreak
 &  & backwardation & 3.32\% & 9.77\% & 10.61\% & 3.97\%\\
\nopagebreak
\multirow[t]{-9}{*}{\raggedright\arraybackslash \textbf{open interest}} & \multirow[t]{-3}{*}{\raggedright\arraybackslash short} & contango & 3.98\% & 5.16\% & 13.83\% & 4.86\%\\
\cmidrule{1-7}\pagebreak[0]
 &  & all & 1.12\% & 1.46\% & 2.93\% & 0.94\%\\
\nopagebreak
 &  & backwardation & 1.16\% & 2.17\% & 2.53\% & 1.28\%\\
\nopagebreak
 & \multirow[t]{-3}{*}{\raggedright\arraybackslash factor} & contango & 1.55\% & 2.02\% & 3.6\% & 1.22\%\\
\nopagebreak
 &  & all & 4.15\% & 6.36\% & 8.39\% & 5.07\%\\
\nopagebreak
 &  & backwardation & 4.38\% & 8.41\% & 9.91\% & 4.98\%\\
\nopagebreak
 & \multirow[t]{-3}{*}{\raggedright\arraybackslash long} & contango & 4.28\% & 5.73\% & 7.99\% & 5.5\%\\
\nopagebreak
 &  & all & 2.59\% & 8.79\% & 14.46\% & 4\%\\
\nopagebreak
 &  & backwardation & 2.78\% & 10.16\% & 10.18\% & 2.69\%\\
\nopagebreak
\multirow[t]{-9}{*}{\raggedright\arraybackslash \textbf{term structure}} & \multirow[t]{-3}{*}{\raggedright\arraybackslash short} & contango & 2.9\% & 7.9\% & 18.16\% & 6.25\%\\*
\end{longtable}
\endgroup{}

\newpage

\begingroup\fontsize{9}{11}\selectfont

\begin{longtable}[t]{>{}lllrr}
\caption{\label{tab:co-movement-picks}This table shows the factor 'picks' for the market as well as the long-short factors for the four periods of interest (past: 1997-2003; financialisation: 2004-2008; crisis: 2008-2013; post-crisis: 2013-2018). For each factor and period the twenty four US commodities are regressed against the corresponding factor mimicking portfolio and ranked by $R^{2}$. The top eight commodities are selected as 'picks' for the corresponding factor and period. The table also shows the proportion of time, for each period, each individual pick lives on both legs (long vs. short) of its picking factor. The results are reported in order for the market, CHP, open interest and term structure factors used as the picking factor. Factors construction details are discussed in section \ref{co-movement-methods} while the results are discussed in section \ref{co-movement-results}.}\\
\toprule
picking factor & period & pick & long & short\\
\midrule
\endfirsthead
\caption[]{\textit{(continued)}}\\
\toprule
picking factor & period & pick & long & short\\
\midrule
\endhead

\endfoot
\bottomrule
\endlastfoot
 &  & Corn-\#2 yellow (XCBT) & 100\% & 0\%\\
\nopagebreak
 &  & Soybean meal (XCBT) & 100\% & 0\%\\
\nopagebreak
 &  & Soybeans (XCBT) & 100\% & 0\%\\
\nopagebreak
 &  & Wheat-SRW (XCBT) & 100\% & 0\%\\
\nopagebreak
 &  & Natural gas (XNYM) & 100\% & 0\%\\
\nopagebreak
 &  & Crude oil-WTI (XNYM) & 100\% & 0\%\\
\nopagebreak
 &  & Gasoline (XNYM) & 100\% & 0\%\\
\nopagebreak
 & \multirow[t]{-8}{*}{\raggedright\arraybackslash past} & Heating oil (XNYM) & 100\% & 0\%\\
\nopagebreak
 &  & Corn-\#2 yellow (XCBT) & 100\% & 0\%\\
\nopagebreak
 &  & Soybean oil (XCBT) & 100\% & 0\%\\
\nopagebreak
 &  & Soybeans (XCBT) & 100\% & 0\%\\
\nopagebreak
 &  & Crude oil-WTI (XNYM) & 100\% & 0\%\\
\nopagebreak
 &  & Heating oil (XNYM) & 100\% & 0\%\\
\nopagebreak
 &  & Gold (XCEC) & 100\% & 0\%\\
\nopagebreak
 &  & Palladium (XNYM) & 100\% & 0\%\\
\nopagebreak
 & \multirow[t]{-8}{*}{\raggedright\arraybackslash financialization} & Silver (XCEC) & 100\% & 0\%\\
\nopagebreak
 &  & Corn-\#2 yellow (XCBT) & 100\% & 0\%\\
\nopagebreak
 &  & Soybean oil (XCBT) & 100\% & 0\%\\
\nopagebreak
 &  & Soybeans (XCBT) & 100\% & 0\%\\
\nopagebreak
 &  & Wheat-SRW (XCBT) & 100\% & 0\%\\
\nopagebreak
 &  & Crude oil-WTI (XNYM) & 100\% & 0\%\\
\nopagebreak
 &  & Gasoline (XNYM) & 100\% & 0\%\\
\nopagebreak
 &  & Heating oil (XNYM) & 100\% & 0\%\\
\nopagebreak
 & \multirow[t]{-8}{*}{\raggedright\arraybackslash crisis} & Copper (XCEC) & 100\% & 0\%\\
\nopagebreak
 &  & Corn-\#2 yellow (XCBT) & 100\% & 0\%\\
\nopagebreak
 &  & Soybean oil (XCBT) & 100\% & 0\%\\
\nopagebreak
 &  & Soybeans (XCBT) & 100\% & 0\%\\
\nopagebreak
 &  & Crude oil-WTI (XNYM) & 100\% & 0\%\\
\nopagebreak
 &  & Gasoline (XNYM) & 100\% & 0\%\\
\nopagebreak
 &  & Heating oil (XNYM) & 100\% & 0\%\\
\nopagebreak
 &  & Copper (XCEC) & 100\% & 0\%\\
\nopagebreak
\multirow[t]{-32}{*}{\raggedright\arraybackslash \textbf{market}} & \multirow[t]{-8}{*}{\raggedright\arraybackslash post-crisis} & Silver (XCEC) & 100\% & 0\%\\
\cmidrule{1-5}\pagebreak[0]
 &  & Cattle-feeder (XCME) & 12.28\% & 67.96\%\\
\nopagebreak
 &  & Cattle-live (XCME) & 16.47\% & 50.3\%\\
\nopagebreak
 &  & Lean hogs (XCME) & 13.77\% & 58.98\%\\
\nopagebreak
 &  & Cotton-\#2 (IFUS) & 18.56\% & 46.11\%\\
\nopagebreak
 &  & Orange juice (IFUS) & 55.39\% & 18.56\%\\
\nopagebreak
 &  & Gold (XCEC) & 36.83\% & 61.68\%\\
\nopagebreak
 &  & Platinum (XNYM) & 91.92\% & 2.1\%\\
\nopagebreak
 & \multirow[t]{-8}{*}{\raggedright\arraybackslash past} & Silver (XCEC) & 100\% & 0\%\\
\nopagebreak
 &  & Oats (XCBT) & 97.91\% & 0\%\\
\nopagebreak
 &  & Cattle-feeder (XCME) & 0\% & 97.49\%\\
\nopagebreak
 &  & Lean hogs (XCME) & 0\% & 72.38\%\\
\nopagebreak
 &  & Copper (XCEC) & 28.45\% & 46.44\%\\
\nopagebreak
 &  & Gold (XCEC) & 100\% & 0\%\\
\nopagebreak
 &  & Palladium (XNYM) & 100\% & 0\%\\
\nopagebreak
 &  & Platinum (XNYM) & 100\% & 0\%\\
\nopagebreak
 & \multirow[t]{-8}{*}{\raggedright\arraybackslash financialization} & Silver (XCEC) & 100\% & 0\%\\
\nopagebreak
 &  & Natural gas (XNYM) & 0\% & 100\%\\
\nopagebreak
 &  & Crude oil-WTI (XNYM) & 5.76\% & 46.09\%\\
\nopagebreak
 &  & Gasoline (XNYM) & 52.67\% & 0\%\\
\nopagebreak
 &  & Copper (XCEC) & 0\% & 81.48\%\\
\nopagebreak
 &  & Gold (XCEC) & 100\% & 0\%\\
\nopagebreak
 &  & Palladium (XNYM) & 100\% & 0\%\\
\nopagebreak
 &  & Platinum (XNYM) & 100\% & 0\%\\
\nopagebreak
 & \multirow[t]{-8}{*}{\raggedright\arraybackslash crisis} & Silver (XCEC) & 99.59\% & 0\%\\
\nopagebreak
 &  & Corn-\#2 yellow (XCBT) & 0\% & 62.77\%\\
\nopagebreak
 &  & Soybean meal (XCBT) & 34.04\% & 0\%\\
\nopagebreak
 &  & Wheat-SRW (XCBT) & 0\% & 97.52\%\\
\nopagebreak
 &  & Natural gas (XNYM) & 0\% & 100\%\\
\nopagebreak
 &  & Gold (XCEC) & 80.14\% & 0\%\\
\nopagebreak
 &  & Palladium (XNYM) & 100\% & 0\%\\
\nopagebreak
 &  & Platinum (XNYM) & 96.81\% & 0\%\\
\nopagebreak
\multirow[t]{-32}{*}{\raggedright\arraybackslash \textbf{CHP}} & \multirow[t]{-8}{*}{\raggedright\arraybackslash post-crisis} & Silver (XCEC) & 57.45\% & 0\%\\
\cmidrule{1-5}\pagebreak[0]
 &  & Oats (XCBT) & 17.1\% & 50.3\%\\
\nopagebreak
 &  & Soybean meal (XCBT) & 19.71\% & 49.1\%\\
\nopagebreak
 &  & Cattle-live (XCME) & 12.46\% & 58.68\%\\
\nopagebreak
 &  & Natural gas (XNYM) & 82.61\% & 1.2\%\\
\nopagebreak
 &  & Crude oil-WTI (XNYM) & 99.71\% & 0\%\\
\nopagebreak
 &  & Gasoline (XNYM) & 99.71\% & 0\%\\
\nopagebreak
 &  & Heating oil (XNYM) & 99.71\% & 0\%\\
\nopagebreak
 & \multirow[t]{-8}{*}{\raggedright\arraybackslash past} & Palladium (XNYM) & 10.14\% & 52.69\%\\
\nopagebreak
 &  & Soybean meal (XCBT) & 0.75\% & 80.09\%\\
\nopagebreak
 &  & Natural gas (XNYM) & 87.97\% & 0\%\\
\nopagebreak
 &  & Crude oil-WTI (XNYM) & 92.48\% & 0\%\\
\nopagebreak
 &  & Gasoline (XNYM) & 92.11\% & 0\%\\
\nopagebreak
 &  & Heating oil (XNYM) & 92.11\% & 0\%\\
\nopagebreak
 &  & Gold (XCEC) & 16.54\% & 37.04\%\\
\nopagebreak
 &  & Platinum (XNYM) & 19.55\% & 31.02\%\\
\nopagebreak
 & \multirow[t]{-8}{*}{\raggedright\arraybackslash financialization} & Silver (XCEC) & 49.25\% & 17.13\%\\
\nopagebreak
 &  & Soybean oil (XCBT) & 10.61\% & 58.38\%\\
\nopagebreak
 &  & Natural gas (XNYM) & 95.45\% & 0\%\\
\nopagebreak
 &  & Crude oil-WTI (XNYM) & 95.45\% & 0\%\\
\nopagebreak
 &  & Gasoline (XNYM) & 92.05\% & 0\%\\
\nopagebreak
 &  & Heating oil (XNYM) & 94.7\% & 0\%\\
\nopagebreak
 &  & Copper (XCEC) & 23.11\% & 35.14\%\\
\nopagebreak
 &  & Platinum (XNYM) & 51.14\% & 17.84\%\\
\nopagebreak
 & \multirow[t]{-8}{*}{\raggedright\arraybackslash crisis} & Silver (XCEC) & 25.76\% & 34.05\%\\
\nopagebreak
 &  & Corn-\#2 yellow (XCBT) & 14.97\% & 48.33\%\\
\nopagebreak
 &  & Oats (XCBT) & 19.73\% & 53.53\%\\
\nopagebreak
 &  & Wheat-SRW (XCBT) & 10.54\% & 43.49\%\\
\nopagebreak
 &  & Cattle-feeder (XCME) & 26.53\% & 54.65\%\\
\nopagebreak
 &  & Natural gas (XNYM) & 98.64\% & 0\%\\
\nopagebreak
 &  & Crude oil-WTI (XNYM) & 98.64\% & 0\%\\
\nopagebreak
 &  & Gasoline (XNYM) & 97.28\% & 0\%\\
\nopagebreak
\multirow[t]{-32}{*}{\raggedright\arraybackslash \textbf{open interest}} & \multirow[t]{-8}{*}{\raggedright\arraybackslash post-crisis} & Heating oil (XNYM) & 97.28\% & 0\%\\
\cmidrule{1-5}\pagebreak[0]
 &  & Corn-\#2 yellow (XCBT) & 94.52\% & 1.15\%\\
\nopagebreak
 &  & Lean hogs (XCME) & 43.8\% & 36.68\%\\
\nopagebreak
 &  & Cotton-\#2 (IFUS) & 77.81\% & 13.18\%\\
\nopagebreak
 &  & Sugar-\#11 (IFUS) & 25.07\% & 55.87\%\\
\nopagebreak
 &  & Gasoline (XNYM) & 16.43\% & 60.17\%\\
\nopagebreak
 &  & Heating oil (XNYM) & 15.56\% & 33.24\%\\
\nopagebreak
 &  & Palladium (XNYM) & 0\% & 59.03\%\\
\nopagebreak
 & \multirow[t]{-8}{*}{\raggedright\arraybackslash past} & Platinum (XNYM) & 0\% & 89.68\%\\
\nopagebreak
 &  & Soybean meal (XCBT) & 17.27\% & 48.18\%\\
\nopagebreak
 &  & Soybeans (XCBT) & 20.45\% & 31.98\%\\
\nopagebreak
 &  & Cattle-live (XCME) & 19.09\% & 48.18\%\\
\nopagebreak
 &  & Natural gas (XNYM) & 78.18\% & 0\%\\
\nopagebreak
 &  & Copper (XCEC) & 0\% & 89.88\%\\
\nopagebreak
 &  & Gold (XCEC) & 0\% & 17\%\\
\nopagebreak
 &  & Platinum (XNYM) & 0\% & 92.71\%\\
\nopagebreak
 & \multirow[t]{-8}{*}{\raggedright\arraybackslash financialization} & Silver (XCEC) & 0\% & 30.77\%\\
\nopagebreak
 &  & Soybean meal (XCBT) & 0\% & 82.54\%\\
\nopagebreak
 &  & Soybeans (XCBT) & 0\% & 58.33\%\\
\nopagebreak
 &  & Lean hogs (XCME) & 69.84\% & 21.83\%\\
\nopagebreak
 &  & Cotton-\#2 (IFUS) & 38.1\% & 48.02\%\\
\nopagebreak
 &  & Sugar-\#11 (IFUS) & 34.52\% & 50.79\%\\
\nopagebreak
 &  & Natural gas (XNYM) & 76.59\% & 0\%\\
\nopagebreak
 &  & Gasoline (XNYM) & 8.33\% & 60.32\%\\
\nopagebreak
 & \multirow[t]{-8}{*}{\raggedright\arraybackslash crisis} & Copper (XCEC) & 0\% & 40.87\%\\
\nopagebreak
 &  & Corn-\#2 yellow (XCBT) & 89.69\% & 6.77\%\\
\nopagebreak
 &  & Wheat-SRW (XCBT) & 96.91\% & 0\%\\
\nopagebreak
 &  & Cattle-feeder (XCME) & 17.87\% & 69.55\%\\
\nopagebreak
 &  & Cattle-live (XCME) & 27.84\% & 60.9\%\\
\nopagebreak
 &  & Coffee-C (IFUS) & 100\% & 0\%\\
\nopagebreak
 &  & Lumber (XCME) & 40.89\% & 39.85\%\\
\nopagebreak
 &  & Sugar-\#11 (IFUS) & 69.07\% & 20.68\%\\
\nopagebreak
\multirow[t]{-32}{*}{\raggedright\arraybackslash \textbf{term structure}} & \multirow[t]{-8}{*}{\raggedright\arraybackslash post-crisis} & Natural gas (XNYM) & 71.48\% & 17.29\%\\*
\end{longtable}
\endgroup{}

\newpage

\begingroup\fontsize{9}{11}\selectfont

\begin{longtable}[t]{>{}llrrrr}
\caption{\label{tab:co-movement-correlations}This table shows the average pairwise correlation amongst factor picks for each factor and period independently. The figures are shown independently for phases of aggregate backwardation (aggregate CHP $\leq$ period median) and aggregate contango (aggregate CHP > period median). The results are reported in order for the market, CHP, open interest and term structure factors used as the picking factor. Factors as well as aggregate CHP construction and corresponding regime definitions along with details on the picking process are discussed in section \ref{co-movement-methods} while the results are discussed in section \ref{co-movement-results}.}\\
\toprule
picking factor & regime & past & financialisation & crisis & post-crisis\\
\midrule
\endfirsthead
\caption[]{\textit{(continued)}}\\
\toprule
picking factor & regime & past & financialisation & crisis & post-crisis\\
\midrule
\endhead

\endfoot
\bottomrule
\endlastfoot
 & all & 23.68\% & 34.2\% & 50.07\% & 26.13\%\\
\nopagebreak
 & backwardation & 22.21\% & 36.28\% & 45.4\% & 22.53\%\\
\nopagebreak
\multirow[t]{-3}{*}{\raggedright\arraybackslash \textbf{market}} & contango & 25.31\% & 32.6\% & 52.73\% & 29.07\%\\
\cmidrule{1-6}\pagebreak[0]
 & all & 8.05\% & 22.94\% & 40.1\% & 19.84\%\\
\nopagebreak
 & backwardation & 8.4\% & 22.48\% & 43.5\% & 16.63\%\\
\nopagebreak
\multirow[t]{-3}{*}{\raggedright\arraybackslash \textbf{CHP}} & contango & 7.64\% & 23.6\% & 38.72\% & 22.32\%\\
\cmidrule{1-6}\pagebreak[0]
 & all & 12.42\% & 32.35\% & 41.96\% & 16.13\%\\
\nopagebreak
 & backwardation & 11.62\% & 35.44\% & 41.9\% & 14.44\%\\
\nopagebreak
\multirow[t]{-3}{*}{\raggedright\arraybackslash \textbf{open interest}} & contango & 12.99\% & 29.69\% & 42.18\% & 17.75\%\\
\cmidrule{1-6}\pagebreak[0]
 & all & 7.07\% & 23.41\% & 23.15\% & 8.86\%\\
\nopagebreak
 & backwardation & 5.63\% & 24.56\% & 18.99\% & 8.37\%\\
\nopagebreak
\multirow[t]{-3}{*}{\raggedright\arraybackslash \textbf{term structure}} & contango & 8.3\% & 22.66\% & 26.34\% & 9.15\%\\*
\end{longtable}
\endgroup{}

\begingroup\fontsize{9}{11}\selectfont

\begin{longtable}[t]{>{}llllrrrr}
\caption{\label{tab:co-movement-time-reg-picks-facts-picks}This table shows the average time series $R^{2}$ for models where the returns series for the various sets of factor picks are regressed against factor mimicking portfolios independently with the resulting $R^{2}$ averaged across the picks. For each picking factor and period, the market as well as the long-short factor mimicking portfolios are constructed from the corresponding set of factor picks. For each long-short factor (average CHP: CHP; open interest nearby growth: open interest; and term structure) the models are constructed with the factor itself (factor) as well as both legs independently (long vs. short). The results are reported in order for the market, CHP, open interest and term structure factors used as the picking factor, for the four periods of interest (past: 1997-2003; financialisation: 2004-2008; crisis: 2008-2013; post-crisis: 2013-2018) and shown independently for phases of aggregate backwardation (aggregate CHP $\leq$ period median) and aggregate contango (aggregate CHP > period median). Factors as well as aggregate CHP construction and corresponding regime definitions along with details on the picking process are discussed in section \ref{co-movement-methods} while the results are discussed in section \ref{co-movement-results}.}\\
\toprule
picking factor & factor & leg & regime & past & financialisation & crisis & post-crisis\\
\midrule
\endfirsthead
\caption[]{\textit{(continued)}}\\
\toprule
picking factor & factor & leg & regime & past & financialisation & crisis & post-crisis\\
\midrule
\endhead

\endfoot
\bottomrule
\endlastfoot
 &  &  & all & 32.72\% & 42.33\% & 56.22\% & 36.48\%\\
\nopagebreak
 &  &  & backwardation & 31.67\% & 43.85\% & 52.22\% & 34.13\%\\
\nopagebreak
 & \multirow[t]{-3}{*}{\raggedright\arraybackslash market} &  & contango & 33.87\% & 41.14\% & 58.62\% & 38.39\%\\
\nopagebreak
 &  &  & all & 5.61\% & 12.44\% & 6.89\% & 6.83\%\\
\nopagebreak
 &  &  & backwardation & 5.02\% & 11.4\% & 7.33\% & 5.58\%\\
\nopagebreak
 &  & \multirow[t]{-6}{*}{\raggedright\arraybackslash factor} & contango & 7.73\% & 15.34\% & 8.2\% & 9.49\%\\
\nopagebreak
 &  &  & all & 17.09\% & 27.87\% & 40.38\% & 17.61\%\\
\nopagebreak
 &  &  & backwardation & 17.36\% & 26.97\% & 33.21\% & 13.37\%\\
\nopagebreak
 &  & \multirow[t]{-3}{*}{\raggedright\arraybackslash long} & contango & 17.39\% & 29.12\% & 44.06\% & 21.69\%\\
\nopagebreak
 &  &  & all & 16.49\% & 21.39\% & 35.95\% & 16.37\%\\
\nopagebreak
 &  &  & backwardation & 13.19\% & 25.78\% & 32.18\% & 12.42\%\\
\nopagebreak
 & \multirow[t]{-9}{*}{\raggedright\arraybackslash CHP} & \multirow[t]{-3}{*}{\raggedright\arraybackslash short} & contango & 20.53\% & 18.68\% & 39.1\% & 20.49\%\\
\nopagebreak
 &  &  & all & 23.2\% & 7.78\% & 11.68\% & 19.69\%\\
\nopagebreak
 &  &  & backwardation & 24.03\% & 8.65\% & 8.56\% & 19.29\%\\
\nopagebreak
 &  & \multirow[t]{-3}{*}{\raggedright\arraybackslash factor} & contango & 22.67\% & 7.38\% & 14.32\% & 20.15\%\\
\nopagebreak
 &  &  & all & 25.03\% & 26.41\% & 39.29\% & 27.68\%\\
\nopagebreak
 &  &  & backwardation & 26.05\% & 28.34\% & 34.39\% & 27.3\%\\
\nopagebreak
 &  & \multirow[t]{-3}{*}{\raggedright\arraybackslash long} & contango & 24.42\% & 24.86\% & 41.79\% & 28.25\%\\
\nopagebreak
 &  &  & all & 23.67\% & 22.99\% & 34.48\% & 17.13\%\\
\nopagebreak
 &  &  & backwardation & 21.81\% & 26.13\% & 33.14\% & 14.96\%\\
\nopagebreak
 & \multirow[t]{-9}{*}{\raggedright\arraybackslash open interest} & \multirow[t]{-3}{*}{\raggedright\arraybackslash short} & contango & 26.47\% & 20.67\% & 35.71\% & 19.57\%\\
\nopagebreak
 &  &  & all & 5.92\% & 7.26\% & 4.29\% & 1.98\%\\
\nopagebreak
 &  &  & backwardation & 6.44\% & 7.24\% & 9.11\% & 1.51\%\\
\nopagebreak
 &  & \multirow[t]{-3}{*}{\raggedright\arraybackslash factor} & contango & 5.52\% & 9.23\% & 3.01\% & 2.81\%\\
\nopagebreak
 &  &  & all & 11.61\% & 24.74\% & 35.76\% & 17.54\%\\
\nopagebreak
 &  &  & backwardation & 10.72\% & 27.25\% & 33.61\% & 18.51\%\\
\nopagebreak
 &  & \multirow[t]{-3}{*}{\raggedright\arraybackslash long} & contango & 12.92\% & 23.08\% & 37.43\% & 17.53\%\\
\nopagebreak
 &  &  & all & 16.22\% & 24.13\% & 35.98\% & 20.6\%\\
\nopagebreak
 &  &  & backwardation & 17.07\% & 25.62\% & 30.3\% & 16.38\%\\
\nopagebreak
\multirow[t]{-30}{*}{\raggedright\arraybackslash \textbf{market}} & \multirow[t]{-9}{*}{\raggedright\arraybackslash term structure} & \multirow[t]{-3}{*}{\raggedright\arraybackslash short} & contango & 16.15\% & 24.36\% & 39.28\% & 23.56\%\\
\cmidrule{1-8}\pagebreak[0]
 &  &  & all & 18.62\% & 38.1\% & 47.61\% & 28.42\%\\
\nopagebreak
 &  &  & backwardation & 19.04\% & 36.9\% & 50.48\% & 25.17\%\\
\nopagebreak
 & \multirow[t]{-3}{*}{\raggedright\arraybackslash market} &  & contango & 18.39\% & 39.54\% & 46.48\% & 30.9\%\\
\nopagebreak
 &  &  & all & 7.19\% & 21.38\% & 13\% & 14.34\%\\
\nopagebreak
 &  &  & backwardation & 6.78\% & 19.58\% & 12.57\% & 18.44\%\\
\nopagebreak
 &  & \multirow[t]{-6}{*}{\raggedright\arraybackslash factor} & contango & 8.32\% & 23.66\% & 13.51\% & 14.5\%\\
\nopagebreak
 &  &  & all & 10.45\% & 31.09\% & 38.06\% & 24.09\%\\
\nopagebreak
 &  &  & backwardation & 11.18\% & 30.55\% & 41.19\% & 22.78\%\\
\nopagebreak
 &  & \multirow[t]{-3}{*}{\raggedright\arraybackslash long} & contango & 10.21\% & 31.88\% & 36.81\% & 25.3\%\\
\nopagebreak
 &  &  & all & 8.27\% & 9.84\% & 15.7\% & 10.44\%\\
\nopagebreak
 &  &  & backwardation & 8.12\% & 9.87\% & 16.25\% & 15.17\%\\
\nopagebreak
 & \multirow[t]{-9}{*}{\raggedright\arraybackslash CHP} & \multirow[t]{-3}{*}{\raggedright\arraybackslash short} & contango & 9.33\% & 9.98\% & 15.57\% & 11.21\%\\
\nopagebreak
 &  &  & all & 1.47\% & 4.76\% & 9.16\% & 11.59\%\\
\nopagebreak
 &  &  & backwardation & 2.34\% & 4.23\% & 7.98\% & 11.57\%\\
\nopagebreak
 &  & \multirow[t]{-3}{*}{\raggedright\arraybackslash factor} & contango & 1\% & 5.64\% & 10.58\% & 11.9\%\\
\nopagebreak
 &  &  & all & 6.27\% & 24.94\% & 25.22\% & 12.58\%\\
\nopagebreak
 &  &  & backwardation & 6.82\% & 26.16\% & 19.14\% & 12.47\%\\
\nopagebreak
 &  & \multirow[t]{-3}{*}{\raggedright\arraybackslash long} & contango & 6.42\% & 24.13\% & 28.34\% & 12.75\%\\
\nopagebreak
 &  &  & all & 5.59\% & 13.44\% & 35.65\% & 14.08\%\\
\nopagebreak
 &  &  & backwardation & 6.49\% & 16.16\% & 39.06\% & 12.01\%\\
\nopagebreak
 & \multirow[t]{-9}{*}{\raggedright\arraybackslash open interest} & \multirow[t]{-3}{*}{\raggedright\arraybackslash short} & contango & 5.2\% & 11.43\% & 34.37\% & 16.51\%\\
\nopagebreak
 &  &  & all & 3.5\% & 5.28\% & 8.11\% & 5.96\%\\
\nopagebreak
 &  &  & backwardation & 3.79\% & 8.13\% & 9.48\% & 6.88\%\\
\nopagebreak
 &  & \multirow[t]{-3}{*}{\raggedright\arraybackslash factor} & contango & 4.9\% & 4.01\% & 7.64\% & 5.86\%\\
\nopagebreak
 &  &  & all & 7.45\% & 9.78\% & 15.63\% & 12.48\%\\
\nopagebreak
 &  &  & backwardation & 7.12\% & 9.85\% & 15.97\% & 11.94\%\\
\nopagebreak
 &  & \multirow[t]{-3}{*}{\raggedright\arraybackslash long} & contango & 8.85\% & 10.75\% & 15.59\% & 13.31\%\\
\nopagebreak
 &  &  & all & 6.26\% & 15.74\% & 28.56\% & 13.03\%\\
\nopagebreak
 &  &  & backwardation & 6.51\% & 19.46\% & 28.59\% & 10.21\%\\
\nopagebreak
\multirow[t]{-30}{*}{\raggedright\arraybackslash \textbf{CHP}} & \multirow[t]{-9}{*}{\raggedright\arraybackslash term structure} & \multirow[t]{-3}{*}{\raggedright\arraybackslash short} & contango & 6.58\% & 13.22\% & 28.76\% & 17.2\%\\
\cmidrule{1-8}\pagebreak[0]
 &  &  & all & 28.43\% & 41.66\% & 49.79\% & 29.2\%\\
\nopagebreak
 &  &  & backwardation & 28\% & 44.29\% & 49.09\% & 27.73\%\\
\nopagebreak
 & \multirow[t]{-3}{*}{\raggedright\arraybackslash market} &  & contango & 28.84\% & 39.55\% & 50.49\% & 30.43\%\\
\nopagebreak
 &  &  & all & 9.58\% & 14.48\% & 11.28\% & 9.83\%\\
\nopagebreak
 &  &  & backwardation & 9.81\% & 13.64\% & 11.15\% & 10.3\%\\
\nopagebreak
 &  & \multirow[t]{-6}{*}{\raggedright\arraybackslash factor} & contango & 9.47\% & 16.17\% & 11.62\% & 11.76\%\\
\nopagebreak
 &  &  & all & 12.63\% & 27.21\% & 32.29\% & 18.56\%\\
\nopagebreak
 &  &  & backwardation & 12.41\% & 28.26\% & 35.25\% & 15.67\%\\
\nopagebreak
 &  & \multirow[t]{-3}{*}{\raggedright\arraybackslash long} & contango & 13.06\% & 26.85\% & 31.22\% & 20.78\%\\
\nopagebreak
 &  &  & all & 10.71\% & 23.23\% & 16.94\% & 8.09\%\\
\nopagebreak
 &  &  & backwardation & 9.72\% & 24.06\% & 16.56\% & 13.98\%\\
\nopagebreak
 & \multirow[t]{-9}{*}{\raggedright\arraybackslash CHP} & \multirow[t]{-3}{*}{\raggedright\arraybackslash short} & contango & 11.79\% & 23.43\% & 17.35\% & 9.21\%\\
\nopagebreak
 &  &  & all & 18.25\% & 15.83\% & 9.45\% & 11.79\%\\
\nopagebreak
 &  &  & backwardation & 19.96\% & 14.24\% & 6.39\% & 12.24\%\\
\nopagebreak
 &  & \multirow[t]{-3}{*}{\raggedright\arraybackslash factor} & contango & 16.97\% & 17.33\% & 11.49\% & 11.87\%\\
\nopagebreak
 &  &  & all & 24.83\% & 30.24\% & 33.8\% & 13.61\%\\
\nopagebreak
 &  &  & backwardation & 25.07\% & 31.83\% & 26.7\% & 13.11\%\\
\nopagebreak
 &  & \multirow[t]{-3}{*}{\raggedright\arraybackslash long} & contango & 24.77\% & 29.24\% & 37.38\% & 14.3\%\\
\nopagebreak
 &  &  & all & 8.32\% & 17.57\% & 31.68\% & 11.76\%\\
\nopagebreak
 &  &  & backwardation & 7.65\% & 17.99\% & 32.14\% & 11.62\%\\
\nopagebreak
 & \multirow[t]{-9}{*}{\raggedright\arraybackslash open interest} & \multirow[t]{-3}{*}{\raggedright\arraybackslash short} & contango & 9.17\% & 17.62\% & 31.72\% & 12.44\%\\
\nopagebreak
 &  &  & all & 2.3\% & 6.95\% & 7.93\% & 3.63\%\\
\nopagebreak
 &  &  & backwardation & 3.56\% & 5.35\% & 10.01\% & 5.48\%\\
\nopagebreak
 &  & \multirow[t]{-3}{*}{\raggedright\arraybackslash factor} & contango & 2.61\% & 8.26\% & 7.22\% & 3.24\%\\
\nopagebreak
 &  &  & all & 7.54\% & 20.81\% & 17.64\% & 12.3\%\\
\nopagebreak
 &  &  & backwardation & 7.45\% & 22.37\% & 16.44\% & 14.08\%\\
\nopagebreak
 &  & \multirow[t]{-3}{*}{\raggedright\arraybackslash long} & contango & 7.84\% & 19.88\% & 18.58\% & 11.56\%\\
\nopagebreak
 &  &  & all & 8.85\% & 19.61\% & 32.42\% & 11.02\%\\
\nopagebreak
 &  &  & backwardation & 12.38\% & 24.66\% & 32.66\% & 7.46\%\\
\nopagebreak
\multirow[t]{-30}{*}{\raggedright\arraybackslash \textbf{open interest}} & \multirow[t]{-9}{*}{\raggedright\arraybackslash term structure} & \multirow[t]{-3}{*}{\raggedright\arraybackslash short} & contango & 7.69\% & 16.26\% & 32.72\% & 14.76\%\\
\cmidrule{1-8}\pagebreak[0]
 &  &  & all & 20\% & 33.71\% & 33.64\% & 19.37\%\\
\nopagebreak
 &  &  & backwardation & 18.65\% & 35.1\% & 29.47\% & 19.58\%\\
\nopagebreak
 & \multirow[t]{-3}{*}{\raggedright\arraybackslash market} &  & contango & 21.25\% & 32.72\% & 36.91\% & 19.29\%\\
\nopagebreak
 &  &  & all & 4.68\% & 7.97\% & 8.75\% & 3.74\%\\
\nopagebreak
 &  &  & backwardation & 4.7\% & 7.11\% & 10.38\% & 8.99\%\\
\nopagebreak
 &  & \multirow[t]{-6}{*}{\raggedright\arraybackslash factor} & contango & 5.19\% & 9.3\% & 8.01\% & 3.69\%\\
\nopagebreak
 &  &  & all & 11.93\% & 28.13\% & 22.6\% & 7.81\%\\
\nopagebreak
 &  &  & backwardation & 14.34\% & 27.9\% & 21.19\% & 8.65\%\\
\nopagebreak
 &  & \multirow[t]{-3}{*}{\raggedright\arraybackslash long} & contango & 10.77\% & 28.61\% & 24.63\% & 8.45\%\\
\nopagebreak
 &  &  & all & 5.54\% & 9.15\% & 13.2\% & 5.52\%\\
\nopagebreak
 &  &  & backwardation & 4.6\% & 8.86\% & 12.38\% & 11.89\%\\
\nopagebreak
 & \multirow[t]{-9}{*}{\raggedright\arraybackslash CHP} & \multirow[t]{-3}{*}{\raggedright\arraybackslash short} & contango & 7.11\% & 10.4\% & 14.58\% & 5.3\%\\
\nopagebreak
 &  &  & all & 8.62\% & 8.02\% & 5.96\% & 10.47\%\\
\nopagebreak
 &  &  & backwardation & 9.54\% & 7.27\% & 6.5\% & 10.93\%\\
\nopagebreak
 &  & \multirow[t]{-3}{*}{\raggedright\arraybackslash factor} & contango & 7.77\% & 8.87\% & 6.99\% & 10.46\%\\
\nopagebreak
 &  &  & all & 13.53\% & 15.51\% & 16.65\% & 12.44\%\\
\nopagebreak
 &  &  & backwardation & 13.3\% & 16.33\% & 12.47\% & 12.45\%\\
\nopagebreak
 &  & \multirow[t]{-3}{*}{\raggedright\arraybackslash long} & contango & 13.92\% & 15.23\% & 20.05\% & 12.48\%\\
\nopagebreak
 &  &  & all & 5.94\% & 14.55\% & 14.78\% & 7.79\%\\
\nopagebreak
 &  &  & backwardation & 5.61\% & 16.34\% & 13.08\% & 8.88\%\\
\nopagebreak
 & \multirow[t]{-9}{*}{\raggedright\arraybackslash open interest} & \multirow[t]{-3}{*}{\raggedright\arraybackslash short} & contango & 6.29\% & 13.09\% & 17.1\% & 7.81\%\\
\nopagebreak
 &  &  & all & 5.1\% & 8.38\% & 8.8\% & 6.83\%\\
\nopagebreak
 &  &  & backwardation & 7.27\% & 6.31\% & 8.91\% & 9.15\%\\
\nopagebreak
 &  & \multirow[t]{-3}{*}{\raggedright\arraybackslash factor} & contango & 5.15\% & 10.07\% & 9.39\% & 5.96\%\\
\nopagebreak
 &  &  & all & 7.6\% & 12.13\% & 11.03\% & 7.86\%\\
\nopagebreak
 &  &  & backwardation & 7.13\% & 12.38\% & 10.13\% & 9.57\%\\
\nopagebreak
 &  & \multirow[t]{-3}{*}{\raggedright\arraybackslash long} & contango & 8.38\% & 12.17\% & 12.68\% & 7.96\%\\
\nopagebreak
 &  &  & all & 6.03\% & 16.11\% & 20.26\% & 9.96\%\\
\nopagebreak
 &  &  & backwardation & 9.07\% & 15.15\% & 15.34\% & 10.19\%\\
\nopagebreak
\multirow[t]{-30}{*}{\raggedright\arraybackslash \textbf{term structure}} & \multirow[t]{-9}{*}{\raggedright\arraybackslash term structure} & \multirow[t]{-3}{*}{\raggedright\arraybackslash short} & contango & 6.06\% & 17.5\% & 24.31\% & 10.3\%\\*
\end{longtable}
\endgroup{}

\newpage

\begingroup\fontsize{9}{11}\selectfont

\begin{longtable}[t]{>{}llllrrrr}
\caption{\label{tab:co-movement-time-reg-US-facts-picks}This table shows the average time series $R^{2}$ for models where the returns series for the twenty four US commodities are regressed against factor mimicking portfolios independently with the resulting $R^{2}$ averaged across all the twenty four commodities. For each picking factor and period, the market as well as the long-short factor mimicking portfolios are constructed from the corresponding set of factor picks. For each long-short factor (average CHP: CHP; open interest nearby growth: open interest; and term structure) the models are constructed with the factor itself (factor) as well as both legs independently (long vs. short). The results are reported in order for the market, CHP, open interest and term structure factors used as the picking factor, for the four periods of interest (past: 1997-2003; financialisation: 2004-2008; crisis: 2008-2013; post-crisis: 2013-2018) and shown independently for phases of aggregate backwardation (aggregate CHP $\leq$ period median) and aggregate contango (aggregate CHP > period median). Factors as well as aggregate CHP construction and corresponding regime definitions along with details on the picking process are discussed in section \ref{co-movement-methods} while the results are discussed in section \ref{co-movement-results}.}\\
\toprule
picking factor & factor & leg & regime & past & financialisation & crisis & post-crisis\\
\midrule
\endfirsthead
\caption[]{\textit{(continued)}}\\
\toprule
picking factor & factor & leg & regime & past & financialisation & crisis & post-crisis\\
\midrule
\endhead

\endfoot
\bottomrule
\endlastfoot
 &  &  & all & 11.67\% & 21.24\% & 28.16\% & 14.85\%\\
\nopagebreak
 &  &  & backwardation & 11.34\% & 22.98\% & 26.68\% & 13.36\%\\
\nopagebreak
 & \multirow[t]{-3}{*}{\raggedright\arraybackslash market} &  & contango & 12.09\% & 19.79\% & 29.81\% & 16.31\%\\
\nopagebreak
 &  &  & all & 1.96\% & 5.44\% & 2.52\% & 3.24\%\\
\nopagebreak
 &  &  & backwardation & 1.83\% & 4.93\% & 3.16\% & 4.06\%\\
\nopagebreak
 &  & \multirow[t]{-6}{*}{\raggedright\arraybackslash factor} & contango & 2.85\% & 6.82\% & 2.83\% & 3.73\%\\
\nopagebreak
 &  &  & all & 6.64\% & 13.42\% & 18.81\% & 7.99\%\\
\nopagebreak
 &  &  & backwardation & 6.99\% & 13.31\% & 16.43\% & 7.93\%\\
\nopagebreak
 &  & \multirow[t]{-3}{*}{\raggedright\arraybackslash long} & contango & 6.6\% & 13.78\% & 20.61\% & 9.08\%\\
\nopagebreak
 &  &  & all & 6.12\% & 10.75\% & 18.35\% & 7.61\%\\
\nopagebreak
 &  &  & backwardation & 4.75\% & 13.22\% & 16.28\% & 5.87\%\\
\nopagebreak
 & \multirow[t]{-9}{*}{\raggedright\arraybackslash CHP} & \multirow[t]{-3}{*}{\raggedright\arraybackslash short} & contango & 7.85\% & 9.05\% & 20.59\% & 9.7\%\\
\nopagebreak
 &  &  & all & 8.25\% & 3.72\% & 4.49\% & 6.9\%\\
\nopagebreak
 &  &  & backwardation & 8.64\% & 4.35\% & 3.3\% & 6.81\%\\
\nopagebreak
 &  & \multirow[t]{-3}{*}{\raggedright\arraybackslash factor} & contango & 8.08\% & 3.44\% & 5.7\% & 7.11\%\\
\nopagebreak
 &  &  & all & 8.51\% & 12.69\% & 18.26\% & 9.71\%\\
\nopagebreak
 &  &  & backwardation & 8.85\% & 13.94\% & 17.03\% & 9.45\%\\
\nopagebreak
 &  & \multirow[t]{-3}{*}{\raggedright\arraybackslash long} & contango & 8.4\% & 11.82\% & 19.42\% & 10.13\%\\
\nopagebreak
 &  &  & all & 9.47\% & 11.69\% & 17.99\% & 8.18\%\\
\nopagebreak
 &  &  & backwardation & 8.98\% & 14.42\% & 17.72\% & 6.81\%\\
\nopagebreak
 & \multirow[t]{-9}{*}{\raggedright\arraybackslash open interest} & \multirow[t]{-3}{*}{\raggedright\arraybackslash short} & contango & 10.39\% & 9.76\% & 18.66\% & 9.79\%\\
\nopagebreak
 &  &  & all & 2.1\% & 3.26\% & 1.76\% & 0.79\%\\
\nopagebreak
 &  &  & backwardation & 2.31\% & 3.4\% & 4.15\% & 0.77\%\\
\nopagebreak
 &  & \multirow[t]{-3}{*}{\raggedright\arraybackslash factor} & contango & 2.06\% & 4.12\% & 1.21\% & 1.05\%\\
\nopagebreak
 &  &  & all & 4.51\% & 13.46\% & 17.75\% & 7.25\%\\
\nopagebreak
 &  &  & backwardation & 4.26\% & 15.48\% & 17.21\% & 7.69\%\\
\nopagebreak
 &  & \multirow[t]{-3}{*}{\raggedright\arraybackslash long} & contango & 5.01\% & 12.06\% & 18.8\% & 7.44\%\\
\nopagebreak
 &  &  & all & 5.69\% & 11.42\% & 17.51\% & 7.68\%\\
\nopagebreak
 &  &  & backwardation & 5.91\% & 12.62\% & 15.24\% & 6.11\%\\
\nopagebreak
\multirow[t]{-30}{*}{\raggedright\arraybackslash \textbf{market}} & \multirow[t]{-9}{*}{\raggedright\arraybackslash term structure} & \multirow[t]{-3}{*}{\raggedright\arraybackslash short} & contango & 5.87\% & 11.09\% & 19.48\% & 8.94\%\\
\cmidrule{1-8}\pagebreak[0]
 &  &  & all & 6.77\% & 16.67\% & 25.41\% & 12.77\%\\
\nopagebreak
 &  &  & backwardation & 6.85\% & 17.81\% & 24.61\% & 10.81\%\\
\nopagebreak
 & \multirow[t]{-3}{*}{\raggedright\arraybackslash market} &  & contango & 6.84\% & 15.98\% & 26.51\% & 14.52\%\\
\nopagebreak
 &  &  & all & 2.54\% & 8.73\% & 4.39\% & 5.06\%\\
\nopagebreak
 &  &  & backwardation & 2.46\% & 8.28\% & 4.36\% & 6.61\%\\
\nopagebreak
 &  & \multirow[t]{-6}{*}{\raggedright\arraybackslash factor} & contango & 3.04\% & 9.5\% & 4.67\% & 5.16\%\\
\nopagebreak
 &  &  & all & 3.9\% & 12.72\% & 17.36\% & 9.4\%\\
\nopagebreak
 &  &  & backwardation & 4.13\% & 13.64\% & 19.11\% & 8.14\%\\
\nopagebreak
 &  & \multirow[t]{-3}{*}{\raggedright\arraybackslash long} & contango & 4\% & 12.3\% & 16.6\% & 10.8\%\\
\nopagebreak
 &  &  & all & 2.83\% & 3.82\% & 7.51\% & 4.17\%\\
\nopagebreak
 &  &  & backwardation & 2.82\% & 3.72\% & 7.27\% & 6.11\%\\
\nopagebreak
 & \multirow[t]{-9}{*}{\raggedright\arraybackslash CHP} & \multirow[t]{-3}{*}{\raggedright\arraybackslash short} & contango & 3.31\% & 4.19\% & 7.95\% & 4.3\%\\
\nopagebreak
 &  &  & all & 0.54\% & 1.68\% & 4.1\% & 4.1\%\\
\nopagebreak
 &  &  & backwardation & 0.9\% & 1.54\% & 2.96\% & 4.14\%\\
\nopagebreak
 &  & \multirow[t]{-3}{*}{\raggedright\arraybackslash factor} & contango & 0.38\% & 2.27\% & 5.48\% & 4.26\%\\
\nopagebreak
 &  &  & all & 2.36\% & 10.1\% & 15.23\% & 4.56\%\\
\nopagebreak
 &  &  & backwardation & 2.57\% & 10.72\% & 10.19\% & 4.46\%\\
\nopagebreak
 &  & \multirow[t]{-3}{*}{\raggedright\arraybackslash long} & contango & 2.45\% & 9.77\% & 18.64\% & 4.71\%\\
\nopagebreak
 &  &  & all & 1.94\% & 6.35\% & 17.65\% & 6.96\%\\
\nopagebreak
 &  &  & backwardation & 2.25\% & 8.71\% & 17.99\% & 5.62\%\\
\nopagebreak
 & \multirow[t]{-9}{*}{\raggedright\arraybackslash open interest} & \multirow[t]{-3}{*}{\raggedright\arraybackslash short} & contango & 1.88\% & 4.71\% & 17.98\% & 8.46\%\\
\nopagebreak
 &  &  & all & 1.31\% & 1.97\% & 2.83\% & 2.2\%\\
\nopagebreak
 &  &  & backwardation & 1.59\% & 2.9\% & 3.32\% & 2.55\%\\
\nopagebreak
 &  & \multirow[t]{-3}{*}{\raggedright\arraybackslash factor} & contango & 1.85\% & 1.65\% & 2.74\% & 2.37\%\\
\nopagebreak
 &  &  & all & 2.62\% & 4.52\% & 8.05\% & 5.5\%\\
\nopagebreak
 &  &  & backwardation & 2.65\% & 5.03\% & 7.26\% & 5.11\%\\
\nopagebreak
 &  & \multirow[t]{-3}{*}{\raggedright\arraybackslash long} & contango & 3.08\% & 4.51\% & 8.86\% & 6.15\%\\
\nopagebreak
 &  &  & all & 2.43\% & 7\% & 15.51\% & 6.71\%\\
\nopagebreak
 &  &  & backwardation & 2.49\% & 9.01\% & 13.83\% & 4.78\%\\
\nopagebreak
\multirow[t]{-30}{*}{\raggedright\arraybackslash \textbf{CHP}} & \multirow[t]{-9}{*}{\raggedright\arraybackslash term structure} & \multirow[t]{-3}{*}{\raggedright\arraybackslash short} & contango & 2.94\% & 5.61\% & 16.98\% & 9.57\%\\
\cmidrule{1-8}\pagebreak[0]
 &  &  & all & 10.6\% & 18.43\% & 26.36\% & 12.03\%\\
\nopagebreak
 &  &  & backwardation & 10.25\% & 20.18\% & 25.13\% & 10.91\%\\
\nopagebreak
 & \multirow[t]{-3}{*}{\raggedright\arraybackslash market} &  & contango & 11.05\% & 17.11\% & 27.83\% & 13.17\%\\
\nopagebreak
 &  &  & all & 4\% & 5.34\% & 4.54\% & 3.71\%\\
\nopagebreak
 &  &  & backwardation & 4.1\% & 4.98\% & 4.37\% & 3.61\%\\
\nopagebreak
 &  & \multirow[t]{-6}{*}{\raggedright\arraybackslash factor} & contango & 4.02\% & 6.14\% & 4.86\% & 4.74\%\\
\nopagebreak
 &  &  & all & 6.78\% & 13.48\% & 18.66\% & 7.42\%\\
\nopagebreak
 &  &  & backwardation & 6.63\% & 14.34\% & 20.18\% & 5.9\%\\
\nopagebreak
 &  & \multirow[t]{-3}{*}{\raggedright\arraybackslash long} & contango & 7.12\% & 13.11\% & 18.16\% & 8.77\%\\
\nopagebreak
 &  &  & all & 3.94\% & 9.11\% & 7.49\% & 3.11\%\\
\nopagebreak
 &  &  & backwardation & 3.56\% & 9.31\% & 7.08\% & 5.61\%\\
\nopagebreak
 & \multirow[t]{-9}{*}{\raggedright\arraybackslash CHP} & \multirow[t]{-3}{*}{\raggedright\arraybackslash short} & contango & 4.45\% & 9.44\% & 8.05\% & 3.32\%\\
\nopagebreak
 &  &  & all & 6.4\% & 5.64\% & 3.37\% & 4.11\%\\
\nopagebreak
 &  &  & backwardation & 6.93\% & 5.3\% & 2.53\% & 4.35\%\\
\nopagebreak
 &  & \multirow[t]{-3}{*}{\raggedright\arraybackslash factor} & contango & 6.07\% & 6.09\% & 4.25\% & 4.1\%\\
\nopagebreak
 &  &  & all & 8.48\% & 12.35\% & 15.81\% & 4.76\%\\
\nopagebreak
 &  &  & backwardation & 8.5\% & 13.24\% & 11.7\% & 4.54\%\\
\nopagebreak
 &  & \multirow[t]{-3}{*}{\raggedright\arraybackslash long} & contango & 8.63\% & 11.82\% & 18.57\% & 5.08\%\\
\nopagebreak
 &  &  & all & 4.07\% & 12.5\% & 19.46\% & 4.99\%\\
\nopagebreak
 &  &  & backwardation & 3.64\% & 13.52\% & 18.93\% & 4.95\%\\
\nopagebreak
 & \multirow[t]{-9}{*}{\raggedright\arraybackslash open interest} & \multirow[t]{-3}{*}{\raggedright\arraybackslash short} & contango & 4.64\% & 11.82\% & 20.29\% & 5.39\%\\
\nopagebreak
 &  &  & all & 0.83\% & 2.54\% & 3.13\% & 1.45\%\\
\nopagebreak
 &  &  & backwardation & 1.3\% & 2.22\% & 3.71\% & 2.52\%\\
\nopagebreak
 &  & \multirow[t]{-3}{*}{\raggedright\arraybackslash factor} & contango & 1.13\% & 2.95\% & 3.02\% & 1.16\%\\
\nopagebreak
 &  &  & all & 3.09\% & 7.65\% & 8.04\% & 5.26\%\\
\nopagebreak
 &  &  & backwardation & 2.8\% & 8.53\% & 7.11\% & 5.96\%\\
\nopagebreak
 &  & \multirow[t]{-3}{*}{\raggedright\arraybackslash long} & contango & 3.73\% & 7.22\% & 9\% & 5.21\%\\
\nopagebreak
 &  &  & all & 3.55\% & 9.48\% & 16.63\% & 4.56\%\\
\nopagebreak
 &  &  & backwardation & 4.8\% & 12.03\% & 15.62\% & 3.13\%\\
\nopagebreak
\multirow[t]{-30}{*}{\raggedright\arraybackslash \textbf{open interest}} & \multirow[t]{-9}{*}{\raggedright\arraybackslash term structure} & \multirow[t]{-3}{*}{\raggedright\arraybackslash short} & contango & 3.21\% & 7.83\% & 17.77\% & 6.32\%\\
\cmidrule{1-8}\pagebreak[0]
 &  &  & all & 9.13\% & 18.37\% & 25.17\% & 8.95\%\\
\nopagebreak
 &  &  & backwardation & 8.79\% & 20.14\% & 22.77\% & 8.19\%\\
\nopagebreak
 & \multirow[t]{-3}{*}{\raggedright\arraybackslash market} &  & contango & 9.55\% & 16.99\% & 27.12\% & 10.02\%\\
\nopagebreak
 &  &  & all & 1.98\% & 3.42\% & 3.3\% & 1.38\%\\
\nopagebreak
 &  &  & backwardation & 1.89\% & 3.08\% & 3.79\% & 3.4\%\\
\nopagebreak
 &  & \multirow[t]{-6}{*}{\raggedright\arraybackslash factor} & contango & 2.33\% & 4.07\% & 3.17\% & 1.32\%\\
\nopagebreak
 &  &  & all & 4.8\% & 13.48\% & 15.65\% & 3.14\%\\
\nopagebreak
 &  &  & backwardation & 5.63\% & 14.34\% & 13.9\% & 3.23\%\\
\nopagebreak
 &  & \multirow[t]{-3}{*}{\raggedright\arraybackslash long} & contango & 4.55\% & 13.11\% & 17.32\% & 3.58\%\\
\nopagebreak
 &  &  & all & 2.58\% & 5.81\% & 7.05\% & 2.51\%\\
\nopagebreak
 &  &  & backwardation & 1.91\% & 5.91\% & 5.92\% & 5.04\%\\
\nopagebreak
 & \multirow[t]{-9}{*}{\raggedright\arraybackslash CHP} & \multirow[t]{-3}{*}{\raggedright\arraybackslash short} & contango & 3.55\% & 6.2\% & 8.3\% & 2.29\%\\
\nopagebreak
 &  &  & all & 3.93\% & 3.97\% & 2.83\% & 3.6\%\\
\nopagebreak
 &  &  & backwardation & 4.31\% & 3.56\% & 2.88\% & 3.87\%\\
\nopagebreak
 &  & \multirow[t]{-3}{*}{\raggedright\arraybackslash factor} & contango & 3.67\% & 4.63\% & 3.53\% & 3.57\%\\
\nopagebreak
 &  &  & all & 6.42\% & 8.62\% & 12.51\% & 4.45\%\\
\nopagebreak
 &  &  & backwardation & 6.24\% & 9.01\% & 9.32\% & 4.44\%\\
\nopagebreak
 &  & \multirow[t]{-3}{*}{\raggedright\arraybackslash long} & contango & 6.73\% & 8.59\% & 14.98\% & 4.53\%\\
\nopagebreak
 &  &  & all & 2.49\% & 8.49\% & 10.4\% & 3.46\%\\
\nopagebreak
 &  &  & backwardation & 2.6\% & 10.34\% & 8.32\% & 3.83\%\\
\nopagebreak
 & \multirow[t]{-9}{*}{\raggedright\arraybackslash open interest} & \multirow[t]{-3}{*}{\raggedright\arraybackslash short} & contango & 2.58\% & 6.84\% & 12.73\% & 3.64\%\\
\nopagebreak
 &  &  & all & 2.17\% & 3.38\% & 3.69\% & 2.53\%\\
\nopagebreak
 &  &  & backwardation & 3.39\% & 2.69\% & 3.48\% & 3.52\%\\
\nopagebreak
 &  & \multirow[t]{-3}{*}{\raggedright\arraybackslash factor} & contango & 2.06\% & 4.03\% & 4.16\% & 2.16\%\\
\nopagebreak
 &  &  & all & 3.91\% & 5.88\% & 5.59\% & 3.55\%\\
\nopagebreak
 &  &  & backwardation & 3.65\% & 6.37\% & 4.61\% & 3.98\%\\
\nopagebreak
 &  & \multirow[t]{-3}{*}{\raggedright\arraybackslash long} & contango & 4.38\% & 5.7\% & 6.74\% & 3.94\%\\
\nopagebreak
 &  &  & all & 2.44\% & 7.8\% & 12.81\% & 3.63\%\\
\nopagebreak
 &  &  & backwardation & 4.14\% & 8.02\% & 9.12\% & 3.54\%\\
\nopagebreak
\multirow[t]{-30}{*}{\raggedright\arraybackslash \textbf{term structure}} & \multirow[t]{-9}{*}{\raggedright\arraybackslash term structure} & \multirow[t]{-3}{*}{\raggedright\arraybackslash short} & contango & 2.27\% & 8.05\% & 15.82\% & 4.21\%\\*
\end{longtable}
\endgroup{}

\newpage

\begingroup\fontsize{9}{11}\selectfont

\begin{longtable}[t]{>{}llllrrrr}
\caption{\label{tab:co-movement-time-reg-UK-facts-picks}This table shows the average time series $R^{2}$ for models where the returns series for the six UK metals are regressed against factor mimicking portfolios independently with the resulting $R^{2}$ averaged across the six UK metals. For each picking factor and period, the market as well as the long-short factor mimicking portfolios are constructed from the corresponding set of factor picks. For each long-short factor (average CHP: CHP; open interest nearby growth: open interest; and term structure) the models are constructed with the factor itself (factor) as well as both legs independently (long vs. short). The results are reported in order for the market, CHP, open interest and term structure factors used as the picking factor, for the four periods of interest (past: 1997-2003; financialisation: 2004-2008; crisis: 2008-2013; post-crisis: 2013-2018) and shown independently for phases of aggregate backwardation (aggregate CHP $\leq$ period median) and aggregate contango (aggregate CHP > period median). Factors as well as aggregate CHP construction and corresponding regime definitions along with details on the picking process are discussed in section \ref{co-movement-methods} while the results are discussed in section \ref{co-movement-results}.}\\
\toprule
picking factor & factor & leg & regime & past & financialisation & crisis & post-crisis\\
\midrule
\endfirsthead
\caption[]{\textit{(continued)}}\\
\toprule
picking factor & factor & leg & regime & past & financialisation & crisis & post-crisis\\
\midrule
\endhead

\endfoot
\bottomrule
\endlastfoot
 &  &  & all & 0.25\% & 10.17\% & 31.69\% & 11.52\%\\
\nopagebreak
 &  &  & backwardation & 0.32\% & 11.66\% & 31.95\% & 8.64\%\\
\nopagebreak
 & \multirow[t]{-3}{*}{\raggedright\arraybackslash market} &  & contango & 0.3\% & 9.16\% & 31.72\% & 14.26\%\\
\nopagebreak
 &  &  & all & 0.06\% & 3.39\% & 0.42\% & 0.38\%\\
\nopagebreak
 &  &  & backwardation & 0.36\% & 1.84\% & 0.26\% & 0.25\%\\
\nopagebreak
 &  & \multirow[t]{-6}{*}{\raggedright\arraybackslash factor} & contango & 0.05\% & 5.36\% & 0.66\% & 0.82\%\\
\nopagebreak
 &  &  & all & 0.15\% & 13.12\% & 18.65\% & 5.45\%\\
\nopagebreak
 &  &  & backwardation & 0.58\% & 12.24\% & 18.24\% & 5.07\%\\
\nopagebreak
 &  & \multirow[t]{-3}{*}{\raggedright\arraybackslash long} & contango & 0.1\% & 14.17\% & 19.06\% & 5.98\%\\
\nopagebreak
 &  &  & all & 0.05\% & 3\% & 24.38\% & 16.07\%\\
\nopagebreak
 &  &  & backwardation & 0.11\% & 4.23\% & 14.79\% & 6.56\%\\
\nopagebreak
 & \multirow[t]{-9}{*}{\raggedright\arraybackslash CHP} & \multirow[t]{-3}{*}{\raggedright\arraybackslash short} & contango & 0.06\% & 2.24\% & 32.21\% & 26.59\%\\
\nopagebreak
 &  &  & all & 0.04\% & 0.83\% & 1.45\% & 0.07\%\\
\nopagebreak
 &  &  & backwardation & 0.05\% & 1.4\% & 0.07\% & 0.03\%\\
\nopagebreak
 &  & \multirow[t]{-3}{*}{\raggedright\arraybackslash factor} & contango & 0.2\% & 0.55\% & 3.21\% & 0.15\%\\
\nopagebreak
 &  &  & all & 0.11\% & 6.59\% & 17.68\% & 2.57\%\\
\nopagebreak
 &  &  & backwardation & 0.17\% & 7.76\% & 17.27\% & 1.58\%\\
\nopagebreak
 &  & \multirow[t]{-3}{*}{\raggedright\arraybackslash long} & contango & 0.17\% & 5.85\% & 18.2\% & 3.57\%\\
\nopagebreak
 &  &  & all & 0.05\% & 3.99\% & 17.16\% & 6.86\%\\
\nopagebreak
 &  &  & backwardation & 0.39\% & 4.23\% & 19.91\% & 3.79\%\\
\nopagebreak
 & \multirow[t]{-9}{*}{\raggedright\arraybackslash open interest} & \multirow[t]{-3}{*}{\raggedright\arraybackslash short} & contango & 0.09\% & 3.85\% & 15.74\% & 10.3\%\\
\nopagebreak
 &  &  & all & 0.1\% & 1.82\% & 0.98\% & 0.17\%\\
\nopagebreak
 &  &  & backwardation & 0.42\% & 1.22\% & 0.04\% & 0.35\%\\
\nopagebreak
 &  & \multirow[t]{-3}{*}{\raggedright\arraybackslash factor} & contango & 0.06\% & 2.64\% & 2.2\% & 0.11\%\\
\nopagebreak
 &  &  & all & 0.26\% & 1.85\% & 11.63\% & 2.82\%\\
\nopagebreak
 &  &  & backwardation & 1.06\% & 2.98\% & 11.17\% & 1.53\%\\
\nopagebreak
 &  & \multirow[t]{-3}{*}{\raggedright\arraybackslash long} & contango & 0.03\% & 1.24\% & 12.09\% & 4.32\%\\
\nopagebreak
 &  &  & all & 0.03\% & 5.78\% & 20.53\% & 3.27\%\\
\nopagebreak
 &  &  & backwardation & 0.1\% & 5.75\% & 18.66\% & 2.59\%\\
\nopagebreak
\multirow[t]{-30}{*}{\raggedright\arraybackslash \textbf{market}} & \multirow[t]{-9}{*}{\raggedright\arraybackslash term structure} & \multirow[t]{-3}{*}{\raggedright\arraybackslash short} & contango & 0.12\% & 6.04\% & 21.67\% & 4.01\%\\
\cmidrule{1-8}\pagebreak[0]
 &  &  & all & 1.55\% & 20.13\% & 34.54\% & 7.43\%\\
\nopagebreak
 &  &  & backwardation & 1.53\% & 20.93\% & 35.18\% & 5.21\%\\
\nopagebreak
 & \multirow[t]{-3}{*}{\raggedright\arraybackslash market} &  & contango & 1.75\% & 19.83\% & 34.3\% & 9.41\%\\
\nopagebreak
 &  &  & all & 0.07\% & 5.5\% & 0.47\% & 2.12\%\\
\nopagebreak
 &  &  & backwardation & 0.12\% & 4.05\% & 1.15\% & 1.4\%\\
\nopagebreak
 &  & \multirow[t]{-6}{*}{\raggedright\arraybackslash factor} & contango & 0.09\% & 7.36\% & 0.25\% & 2.78\%\\
\nopagebreak
 &  &  & all & 1.02\% & 10.98\% & 21.81\% & 8.99\%\\
\nopagebreak
 &  &  & backwardation & 0.77\% & 10.77\% & 25.33\% & 5.57\%\\
\nopagebreak
 &  & \multirow[t]{-3}{*}{\raggedright\arraybackslash long} & contango & 1.36\% & 11.39\% & 19.89\% & 12.11\%\\
\nopagebreak
 &  &  & all & 0.41\% & 0.25\% & 6.19\% & 0.19\%\\
\nopagebreak
 &  &  & backwardation & 0.21\% & 0.47\% & 5.55\% & 0.24\%\\
\nopagebreak
 & \multirow[t]{-9}{*}{\raggedright\arraybackslash CHP} & \multirow[t]{-3}{*}{\raggedright\arraybackslash short} & contango & 0.61\% & 0.22\% & 6.67\% & 0.2\%\\
\nopagebreak
 &  &  & all & 0.03\% & 0.64\% & 0.06\% & 0.27\%\\
\nopagebreak
 &  &  & backwardation & 0.21\% & 0.06\% & 1.07\% & 0.12\%\\
\nopagebreak
 &  & \multirow[t]{-3}{*}{\raggedright\arraybackslash factor} & contango & 0.08\% & 1.68\% & 0.18\% & 1.33\%\\
\nopagebreak
 &  &  & all & 0.28\% & 10.41\% & 15.26\% & 0.52\%\\
\nopagebreak
 &  &  & backwardation & 0.39\% & 10.56\% & 10.18\% & 0.82\%\\
\nopagebreak
 &  & \multirow[t]{-3}{*}{\raggedright\arraybackslash long} & contango & 0.34\% & 10.47\% & 18.21\% & 0.38\%\\
\nopagebreak
 &  &  & all & 0.31\% & 8.77\% & 29.79\% & 4.53\%\\
\nopagebreak
 &  &  & backwardation & 0.25\% & 12.85\% & 30.49\% & 1.57\%\\
\nopagebreak
 & \multirow[t]{-9}{*}{\raggedright\arraybackslash open interest} & \multirow[t]{-3}{*}{\raggedright\arraybackslash short} & contango & 0.49\% & 5.85\% & 29.59\% & 8.33\%\\
\nopagebreak
 &  &  & all & 0.05\% & 4.73\% & 0.83\% & 0.6\%\\
\nopagebreak
 &  &  & backwardation & 0.1\% & 4.5\% & 0.6\% & 1.11\%\\
\nopagebreak
 &  & \multirow[t]{-3}{*}{\raggedright\arraybackslash factor} & contango & 0.27\% & 5.11\% & 0.95\% & 0.31\%\\
\nopagebreak
 &  &  & all & 0.32\% & 1.38\% & 5.13\% & 0.65\%\\
\nopagebreak
 &  &  & backwardation & 0.2\% & 1.98\% & 4.85\% & 0.44\%\\
\nopagebreak
 &  & \multirow[t]{-3}{*}{\raggedright\arraybackslash long} & contango & 0.58\% & 1.03\% & 5.35\% & 0.86\%\\
\nopagebreak
 &  &  & all & 0.26\% & 19.23\% & 18.98\% & 3.91\%\\
\nopagebreak
 &  &  & backwardation & 0.48\% & 21.81\% & 17.8\% & 4.62\%\\
\nopagebreak
\multirow[t]{-30}{*}{\raggedright\arraybackslash \textbf{CHP}} & \multirow[t]{-9}{*}{\raggedright\arraybackslash term structure} & \multirow[t]{-3}{*}{\raggedright\arraybackslash short} & contango & 0.22\% & 17.21\% & 19.65\% & 3.4\%\\
\cmidrule{1-8}\pagebreak[0]
 &  &  & all & 0.31\% & 8.13\% & 34.43\% & 3.37\%\\
\nopagebreak
 &  &  & backwardation & 0.35\% & 9.97\% & 35.43\% & 1.72\%\\
\nopagebreak
 & \multirow[t]{-3}{*}{\raggedright\arraybackslash market} &  & contango & 0.39\% & 6.8\% & 34.06\% & 5.08\%\\
\nopagebreak
 &  &  & all & 0.02\% & 1.39\% & 0.15\% & 1.38\%\\
\nopagebreak
 &  &  & backwardation & 0.06\% & 0.83\% & 0.45\% & 0.45\%\\
\nopagebreak
 &  & \multirow[t]{-6}{*}{\raggedright\arraybackslash factor} & contango & 0.07\% & 2.07\% & 0.11\% & 2.46\%\\
\nopagebreak
 &  &  & all & 0.11\% & 11.35\% & 19.64\% & 2.84\%\\
\nopagebreak
 &  &  & backwardation & 0.29\% & 11.19\% & 23.58\% & 1.06\%\\
\nopagebreak
 &  & \multirow[t]{-3}{*}{\raggedright\arraybackslash long} & contango & 0.07\% & 11.68\% & 17.62\% & 4.84\%\\
\nopagebreak
 &  &  & all & 0.15\% & 1.41\% & 5.91\% & 0.21\%\\
\nopagebreak
 &  &  & backwardation & 0.17\% & 1.89\% & 5.05\% & 0.25\%\\
\nopagebreak
 & \multirow[t]{-9}{*}{\raggedright\arraybackslash CHP} & \multirow[t]{-3}{*}{\raggedright\arraybackslash short} & contango & 0.24\% & 1.11\% & 6.54\% & 0.2\%\\
\nopagebreak
 &  &  & all & 0.04\% & 0.37\% & 0.03\% & 0.16\%\\
\nopagebreak
 &  &  & backwardation & 0.05\% & 0.3\% & 0.33\% & 0.46\%\\
\nopagebreak
 &  & \multirow[t]{-3}{*}{\raggedright\arraybackslash factor} & contango & 0.15\% & 0.46\% & 0.13\% & 0.09\%\\
\nopagebreak
 &  &  & all & 0.2\% & 5.54\% & 15.97\% & 0.62\%\\
\nopagebreak
 &  &  & backwardation & 0.14\% & 6.38\% & 12.59\% & 0.89\%\\
\nopagebreak
 &  & \multirow[t]{-3}{*}{\raggedright\arraybackslash long} & contango & 0.38\% & 5.01\% & 17.83\% & 0.53\%\\
\nopagebreak
 &  &  & all & 0.14\% & 7.03\% & 30.65\% & 0.26\%\\
\nopagebreak
 &  &  & backwardation & 0.35\% & 7.6\% & 30.31\% & 0.07\%\\
\nopagebreak
 & \multirow[t]{-9}{*}{\raggedright\arraybackslash open interest} & \multirow[t]{-3}{*}{\raggedright\arraybackslash short} & contango & 0.08\% & 6.65\% & 30.96\% & 0.9\%\\
\nopagebreak
 &  &  & all & 0.06\% & 0.1\% & 1.96\% & 0.05\%\\
\nopagebreak
 &  &  & backwardation & 0.31\% & 0.2\% & 1.75\% & 0.12\%\\
\nopagebreak
 &  & \multirow[t]{-3}{*}{\raggedright\arraybackslash factor} & contango & 0.07\% & 0.15\% & 2.05\% & 0.06\%\\
\nopagebreak
 &  &  & all & 0.19\% & 1.2\% & 4.58\% & 1.01\%\\
\nopagebreak
 &  &  & backwardation & 0.7\% & 2.21\% & 4.31\% & 0.29\%\\
\nopagebreak
 &  & \multirow[t]{-3}{*}{\raggedright\arraybackslash long} & contango & 0.06\% & 0.69\% & 4.78\% & 2.14\%\\
\nopagebreak
 &  &  & all & 0.07\% & 3.46\% & 22.01\% & 0.93\%\\
\nopagebreak
 &  &  & backwardation & 0.14\% & 4.11\% & 21.24\% & 0.08\%\\
\nopagebreak
\multirow[t]{-30}{*}{\raggedright\arraybackslash \textbf{open interest}} & \multirow[t]{-9}{*}{\raggedright\arraybackslash term structure} & \multirow[t]{-3}{*}{\raggedright\arraybackslash short} & contango & 0.08\% & 3.05\% & 22.44\% & 2.18\%\\
\cmidrule{1-8}\pagebreak[0]
 &  &  & all & 0.75\% & 15.9\% & 28.49\% & 1.27\%\\
\nopagebreak
 &  &  & backwardation & 0.59\% & 17.08\% & 27.92\% & 0.43\%\\
\nopagebreak
 & \multirow[t]{-3}{*}{\raggedright\arraybackslash market} &  & contango & 0.94\% & 15.1\% & 28.9\% & 2.52\%\\
\nopagebreak
 &  &  & all & 0.04\% & 0.34\% & 0.03\% & 0.03\%\\
\nopagebreak
 &  &  & backwardation & 0.09\% & 0.33\% & 0.06\% & 0.11\%\\
\nopagebreak
 &  & \multirow[t]{-6}{*}{\raggedright\arraybackslash factor} & contango & 0.03\% & 0.46\% & 0.05\% & 0.11\%\\
\nopagebreak
 &  &  & all & 0.68\% & 11.35\% & 10.54\% & 0.44\%\\
\nopagebreak
 &  &  & backwardation & 0.4\% & 11.19\% & 8.88\% & 0.57\%\\
\nopagebreak
 &  & \multirow[t]{-3}{*}{\raggedright\arraybackslash long} & contango & 1\% & 11.68\% & 11.43\% & 0.34\%\\
\nopagebreak
 &  &  & all & 0.34\% & 6.03\% & 5.82\% & 0.25\%\\
\nopagebreak
 &  &  & backwardation & 0.18\% & 6.06\% & 3.33\% & 0.13\%\\
\nopagebreak
 & \multirow[t]{-9}{*}{\raggedright\arraybackslash CHP} & \multirow[t]{-3}{*}{\raggedright\arraybackslash short} & contango & 0.53\% & 6.1\% & 7.87\% & 0.4\%\\
\nopagebreak
 &  &  & all & 0.05\% & 0.08\% & 0.27\% & 0.07\%\\
\nopagebreak
 &  &  & backwardation & 0.03\% & 0.13\% & 0.05\% & 0.51\%\\
\nopagebreak
 &  & \multirow[t]{-3}{*}{\raggedright\arraybackslash factor} & contango & 0.26\% & 0.33\% & 0.67\% & 0.25\%\\
\nopagebreak
 &  &  & all & 0.38\% & 3.41\% & 11.21\% & 0.29\%\\
\nopagebreak
 &  &  & backwardation & 0.16\% & 3.72\% & 7.67\% & 0.6\%\\
\nopagebreak
 &  & \multirow[t]{-3}{*}{\raggedright\arraybackslash long} & contango & 0.75\% & 3.24\% & 13.44\% & 0.18\%\\
\nopagebreak
 &  &  & all & 0.19\% & 7.89\% & 11.78\% & 0.3\%\\
\nopagebreak
 &  &  & backwardation & 0.34\% & 10.15\% & 8.1\% & 0.07\%\\
\nopagebreak
 & \multirow[t]{-9}{*}{\raggedright\arraybackslash open interest} & \multirow[t]{-3}{*}{\raggedright\arraybackslash short} & contango & 0.13\% & 6.06\% & 15.31\% & 1.33\%\\
\nopagebreak
 &  &  & all & 0.05\% & 2.34\% & 1.67\% & 0.3\%\\
\nopagebreak
 &  &  & backwardation & 0.09\% & 1.07\% & 0.86\% & 0.35\%\\
\nopagebreak
 &  & \multirow[t]{-3}{*}{\raggedright\arraybackslash factor} & contango & 0.04\% & 3.61\% & 2.43\% & 0.31\%\\
\nopagebreak
 &  &  & all & 0.29\% & 0.88\% & 2.66\% & 0.92\%\\
\nopagebreak
 &  &  & backwardation & 0.44\% & 1.81\% & 1.5\% & 0.44\%\\
\nopagebreak
 &  & \multirow[t]{-3}{*}{\raggedright\arraybackslash long} & contango & 0.22\% & 0.44\% & 3.64\% & 1.65\%\\
\nopagebreak
 &  &  & all & 0.09\% & 13.93\% & 11.04\% & 0.17\%\\
\nopagebreak
 &  &  & backwardation & 0.08\% & 10.91\% & 6.36\% & 0.06\%\\
\nopagebreak
\multirow[t]{-30}{*}{\raggedright\arraybackslash \textbf{term structure}} & \multirow[t]{-9}{*}{\raggedright\arraybackslash term structure} & \multirow[t]{-3}{*}{\raggedright\arraybackslash short} & contango & 0.11\% & 16.59\% & 14.67\% & 0.6\%\\*
\end{longtable}
\endgroup{}

\restoregeometry\newpage

\hypertarget{references}{%
\section*{References}\label{references}}
\addcontentsline{toc}{section}{References}

\hypertarget{refs}{}
\leavevmode\hypertarget{ref-acharya_limits_2013}{}%
Acharya, V.V., Lochstoer, L.A., Ramadorai, T., 2013. Limits to arbitrage and hedging: Evidence from commodity markets. Journal of Financial Economics 109, 441--465.

\leavevmode\hypertarget{ref-asness_devil_2013}{}%
Asness, C., Frazzini, A., 2013. The devil in hml's details. The Journal of Portfolio Management 39, 49--68.

\leavevmode\hypertarget{ref-asness_fact_2015}{}%
Asness, C., Frazzini, A., Israel, R., Moskowitz, T., 2015. Fact, fiction, and value investing. The Journal of Portfolio Management 42, 34--52.

\leavevmode\hypertarget{ref-asness_quality_2019}{}%
Asness, C.S., Frazzini, A., Pedersen, L.H., 2019. Quality minus junk. Review of Accounting Studies 24, 34--112.

\leavevmode\hypertarget{ref-baker_financialisation_2014}{}%
Baker, S.D., 2014. The financialization of storable commodities (SSRN Scholarly Paper No. ID 2348333). Social Science Research Network, Rochester, NY.

\leavevmode\hypertarget{ref-bakshi_understanding_2019}{}%
Bakshi, G., Gao, X., Rossi, A.G., 2019. Understanding the sources of risk underlying the cross section of commodity returns. Management Science 65, 619--641.

\leavevmode\hypertarget{ref-basak_model_2016}{}%
Basak, S., Pavlova, A., 2016. A model of financialization of commodities. The Journal of Finance 71, 1511--1556.

\leavevmode\hypertarget{ref-basu_definancialisation}{}%
Basu, D., Bauthéac, O., 2021. Commodity definancialisation. Working paper.

\leavevmode\hypertarget{ref-basu_capturing_2013}{}%
Basu, D., Miffre, J., 2013. Capturing the risk premium of commodity futures: The role of hedging pressure. Journal of Banking \& Finance 37, 2652--2664.

\leavevmode\hypertarget{ref-bohl_does_2013}{}%
Bohl, M.T., Stephan, P.M., 2013. Does futures speculation destabilize spot prices? New evidence for commodity markets. Journal of Agricultural and Applied Economics 45, 595--616.

\leavevmode\hypertarget{ref-boons_stock_2012}{}%
Boons, M., De Roon, F., Szymanowska, M., 2012. The stock market price of commodity risk, in: AFA 2012 Chicago Meetings Paper.

\leavevmode\hypertarget{ref-boons_basis_2019}{}%
Boons, M., Prado, M.P., 2019. Basis-momentum. The Journal of Finance 74, 239--279.

\leavevmode\hypertarget{ref-boyd_prevalence_2016}{}%
Boyd, N.E., Büyükşahin, B., Haigh, M.S., Harris, J.H., 2016. The prevalence, sources, and effects of herding. Journal of Futures Markets 36, 671--694.

\leavevmode\hypertarget{ref-brennan_supply_1958}{}%
Brennan, M.J., 1958. The supply of storage. The American Economic Review 48, 50--72.

\leavevmode\hypertarget{ref-brunetti_speculators_2016}{}%
Brunetti, C., Büyükşahin, B., Harris, J.H., 2016. Speculators, prices, and market volatility. Journal of Financial and Quantitative Analysis 51, 1545--1574.

\leavevmode\hypertarget{ref-brunetti_commodity_2014}{}%
Brunetti, C., Reiffen, D., 2014. Commodity index trading and hedging costs. Journal of Financial Markets 21, 153--180.

\leavevmode\hypertarget{ref-bruno_financialisation_2017}{}%
Bruno, V.G., Büyükşahin, B., Robe, M.A., 2017. The financialization of food? American Journal of Agricultural Economics 99, 243--264.

\leavevmode\hypertarget{ref-buyuksahin_speculators_2011}{}%
Büyükşahin, B., Harris, J.H., 2011. Do speculators drive crude oil futures prices? The Energy Journal 167--202.

\leavevmode\hypertarget{ref-buyuksahin_speculators_2014}{}%
Büyükşahin, B., Robe, M.A., 2014. Speculators, commodities and cross-market linkages. Journal of International Money and Finance, Understanding international commodity price fluctuations 42, 38--70.

\leavevmode\hypertarget{ref-carhart_persistence_1997}{}%
Carhart, M.M., 1997. On persistence in mutual fund performance. The Journal of finance 52, 57--82.

\leavevmode\hypertarget{ref-cheng_convective_2014}{}%
Cheng, I.-H., Kirilenko, A., Xiong, W., 2014. Convective risk flows in commodity futures markets. Review of Finance 19, 1733--1781.

\leavevmode\hypertarget{ref-cheng_financialisation_2014}{}%
Cheng, I.-H., Xiong, W., 2014. Financialization of commodity markets. Annual Review of Financial Economics 6, 419--441.

\leavevmode\hypertarget{ref-christoffersen_factor_2014}{}%
Christoffersen, P., Lunde, A., Olesen, K.V., 2014. Factor structure in commodity futures return and volatility. Journal of Financial and Quantitative Analysis 1--74.

\leavevmode\hypertarget{ref-cootner_returns_1960}{}%
Cootner, P.H., 1960. Returns to speculators: Telser versus keynes. Journal of political Economy 68, 396--404.

\leavevmode\hypertarget{ref-cortazar_commodity_2013}{}%
Cortazar, G., Kovacevic, I., Schwartz, E.S., 2013. Commodity and asset pricing models: An integration (NBER Working Papers No. 19167). National Bureau of Economic Research, Inc.

\leavevmode\hypertarget{ref-danthine_information_1978}{}%
Danthine, J.-P., 1978. Information, futures prices, and stabilizing speculation. Journal of Economic Theory 17, 79--98.

\leavevmode\hypertarget{ref-daskalaki_factors_2014}{}%
Daskalaki, C., Kostakis, A., Skiadopoulos, G., 2014. Are there common factors in individual commodity futures returns? Journal of Banking \& Finance 40, 346--363.

\leavevmode\hypertarget{ref-deschutter_food_2010}{}%
De Schutter, O., 2010. Food commodities speculation and food price crises: Regulation to reduce the risks of price volatility. United Nations Special Rapporteur on the Right to Food Briefing Note 2, 1--14.

\leavevmode\hypertarget{ref-domanski_financial_2007}{}%
Domanski, D., Heath, A., 2007. Financial investors and commodity markets. Working Paper.

\leavevmode\hypertarget{ref-duffie_challenges_2014}{}%
Duffie, D., 2014. Challenges to a policy treatment of sSpeculative trading motivated by differences in beliefs. The Journal of Legal Studies 43, S173--S182.

\leavevmode\hypertarget{ref-ederington_who_2002}{}%
Ederington, L., Lee, J.H., 2002. Who trades futures and how: Evidence from the heating oil futures market. The Journal of Business 75, 353--373.

\leavevmode\hypertarget{ref-ekeland_speculation_2016}{}%
Ekeland, I., Lautier, D., Villeneuve, B., 2016. Speculation in commodity futures markets: A simple equilibrium model.

\leavevmode\hypertarget{ref-etula_broker-dealer_2013}{}%
Etula, E., 2013. Broker-dealer risk appetite and commodity returns. Journal of Financial Econometrics 11, 486--521.

\leavevmode\hypertarget{ref-fama_common_1993}{}%
Fama, E.F., French, K.R., 1993. Common risk factors in the returns on stocks and bonds. Journal of Financial Economics 33, 3--56.

\leavevmode\hypertarget{ref-fama_five_factor_2015}{}%
Fama, E.F., French, K.R., 2015. A five-factor asset pricing model. Journal of Financial Economics 116, 1--22.

\leavevmode\hypertarget{ref-fattouh_role_2013}{}%
Fattouh, B., Kilian, L., Mahadeva, L., 2013. The role of speculation in oil markets: What have we learnt so far? The Energy Journal 34, 7--33.

\leavevmode\hypertarget{ref-fernandez_skewness_2018}{}%
Fernandez-Perez, A., Frijns, B., Fuertes, A.-M., Miffre, J., 2018. The skewness of commodity futures returns. Journal of Banking \& Finance 86, 143--158.

\leavevmode\hypertarget{ref-frazzini_betting_2014}{}%
Frazzini, A., Pedersen, L.H., 2014. Betting against beta. Journal of Financial Economics 111, 1--25.

\leavevmode\hypertarget{ref-fuertes_commodity_2015}{}%
Fuertes, A.-M., Miffre, J., Fernandez-Perez, A., 2015. Commodity strategies based on momentum, term structure, and idiosyncratic volatility. Journal of Futures Markets 35, 274--297.

\leavevmode\hypertarget{ref-gilbert_how_2010}{}%
Gilbert, C.L., 2010a. How to understand high food prices. Journal of Agricultural Economics 61, 398--425.

\leavevmode\hypertarget{ref-gilbert_speculative_2010}{}%
Gilbert, C.L., 2010b. Speculative influences on commodity futures prices 2006-2008. UNCTAD, Geneva.

\leavevmode\hypertarget{ref-gilbert_role_2014}{}%
Gilbert, C.L., Pfuderer, S., 2014. The role of index trading in price formation in the grains and oilseeds markets. Journal of Agricultural Economics 65, 303--322.

\leavevmode\hypertarget{ref-goldstein_speculation_2014}{}%
Goldstein, I., Li, Y., Yang, L., 2014. Speculation and hedging in segmented markets. The Review of Financial Studies 27, 881--922.

\leavevmode\hypertarget{ref-goldstein_commodity_2017}{}%
Goldstein, I., Yang, L., 2017. Commodity financialization and information transmission (SSRN Scholarly Paper No. ID 2555996). Social Science Research Network, Rochester, NY.

\leavevmode\hypertarget{ref-gorton_fundamentals_2012}{}%
Gorton, G.B., Hayashi, F., Rouwenhorst, K.G., 2012. The fundamentals of commodity futures returns. Review of Finance 17, 35--105.

\leavevmode\hypertarget{ref-grossman_impossibility_1980}{}%
Grossman, S.J., Stiglitz, J.E., 1980. On the impossibility of informationally efficient markets. The American economic review 70, 393--408.

\leavevmode\hypertarget{ref-hamilton_effects_2015}{}%
Hamilton, J.D., Wu, J.C., 2015. Effects of index-fund investing on commodity futures prices. International Economic Review 56, 187--205.

\leavevmode\hypertarget{ref-henderson_new_2015}{}%
Henderson, B.J., Pearson, N.D., Wang, L., 2015. New evidence on the financialization of commodity markets. The Review of Financial Studies 28, 1285--1311.

\leavevmode\hypertarget{ref-herman_not_2011}{}%
Herman, M.-O., Kelly, R., Nash, R., 2011. Not a game, speculation v. Food security: Regulating financial markets to grow a better future. Oxfam policy and practice: agriculture, food and land 11, 127--138.

\leavevmode\hypertarget{ref-hicks_value_1939}{}%
Hicks, J.R., 1939. Value and capital, 1st ed. Oxford University Press, Cambridge.

\leavevmode\hypertarget{ref-hicks_value_1946}{}%
Hicks, J.R., 1946. Value and capital, 2nd ed. Oxford University Press, Cambridge.

\leavevmode\hypertarget{ref-hirshleifer_speculation_1975}{}%
Hirshleifer, J., 1975. Speculation and equilibrium: Information, risk, and markets. The Quarterly Journal of Economics 89, 519--542.

\leavevmode\hypertarget{ref-hirshleifer_reply_1976}{}%
Hirshleifer, J., 1976. Reply to comments on" speculation and equilibrium: Information, risk, and markets". The Quarterly Journal of Economics 689--696.

\leavevmode\hypertarget{ref-hirshleifer_theory_1977}{}%
Hirshleifer, J., 1977. The theory of speculation under alternative regimes of markets. The Journal of Finance 32, 975--999.

\leavevmode\hypertarget{ref-hong_what_2012}{}%
Hong, H., Yogo, M., 2012. What does futures market interest tell us about the macroeconomy and asset prices? Journal of Financial Economics 105, 473--490.

\leavevmode\hypertarget{ref-hou_digesting_2015}{}%
Hou, K., Xue, C., Zhang, L., 2015. Digesting anomalies: An investment approach. The Review of Financial Studies 28, 650--705.

\leavevmode\hypertarget{ref-houthakker_restatement_1957}{}%
Houthakker, H., 1957. Restatement of the theory of normal backwardation (Cowles Foundation Discussion Papers No. 44). Cowles Foundation for Research in Economics, Yale University.

\leavevmode\hypertarget{ref-houthakker_speculators_1957}{}%
Houthakker, H.S., 1957. Can speculators forecast prices? The Review of Economics and Statistics 39, 143--151.

\leavevmode\hypertarget{ref-irwin_commodity_2013}{}%
Irwin, S.H., 2013. Commodity index investment and food prices: Does the "masters hypothesis" explain recent price spikes? Agricultural Economics 44, 29--41.

\leavevmode\hypertarget{ref-irwin_index_2011}{}%
Irwin, S.H., Sanders, D.R., 2011. Index funds, financialization, and commodity futures markets. Applied Economic Perspectives and Policy 33, 1--31.

\leavevmode\hypertarget{ref-irwin_financialisation_2012}{}%
Irwin, S.H., Sanders, D.R., 2012. Financialization and structural change in commodity futures markets. Journal of Agricultural and Applied Economics 44, 371--396.

\leavevmode\hypertarget{ref-irwin_devil_2009}{}%
Irwin, S.H., Sanders, D.R., Merrin, R.P., 2009. Devil or angel? The role of speculation in the recent commodity price boom (and bust). Journal of Agricultural and Applied Economics 41, 377--391.

\leavevmode\hypertarget{ref-juvenal_speculation_2015}{}%
Juvenal, L., Petrella, I., 2015. Speculation in the oil market. Journal of Applied Econometrics 30, 621--649.

\leavevmode\hypertarget{ref-kaldor_speculation_1939}{}%
Kaldor, N., 1939. Speculation and economic stability. The Review of Economic Studies 7, 1--27.

\leavevmode\hypertarget{ref-kang_tale_2020}{}%
Kang, W., Rouwenhorst, K.G., Tang, K., 2020. A tale of two premiums: The role of hedgers and speculators in commodity futures markets. The Journal of Finance 75, 377--417.

\leavevmode\hypertarget{ref-keynes_some_1923}{}%
Keynes, J.M., 1923. Some aspects of commodity markets. Manchester Guardian Commercial: European Reconstruction Series 13, 784--786.

\leavevmode\hypertarget{ref-keynes_treatise_1930}{}%
Keynes, J.M., 1930. Treatise on money. Macmillan, London.

\leavevmode\hypertarget{ref-kilian_role_2014}{}%
Kilian, L., Murphy, D.P., 2014. The role of inventories and speculative trading in the global market for crude oil. Journal of Applied Econometrics 29, 454--478.

\leavevmode\hypertarget{ref-kim_does_2015}{}%
Kim, A., 2015. Does futures speculation destabilize commodity markets? Journal of Futures Markets 35, 696--714.

\leavevmode\hypertarget{ref-knittel_simple_2016}{}%
Knittel, C.R., Pindyck, R.S., 2016. The simple economics of commodity price speculation. American Economic Journal: Macroeconomics 8, 85--110.

\leavevmode\hypertarget{ref-kyle_informed_1989}{}%
Kyle, A.S., 1989. Informed speculation with imperfect competition. The Review of Economic Studies 56, 317--355.

\leavevmode\hypertarget{ref-leclercq_equilibrium_2014}{}%
Leclercq, E., Praz, R., 2014. Equilibrium commodity trading (SSRN Scholarly Paper No. ID 2464400). Social Science Research Network, Rochester, NY.

\leavevmode\hypertarget{ref-manera_modeling_2016}{}%
Manera, M., Nicolini, M., Vignati, I., 2016. Modeling futures price volatility in energy markets: Is there a role for financial speculation? Energy Economics, Energy markets 53, 220--229.

\leavevmode\hypertarget{ref-masters_testimony_2008}{}%
Masters, M.W., 2008. Testimony before the committee on homeland security and governmental affairs. U.S. Senate 20.

\leavevmode\hypertarget{ref-masters_accidental_2008}{}%
Masters, M., White, A., 2008. The accidental hunt brothers: How institutional investors are driving up food and energy prices. Working paper.

\leavevmode\hypertarget{ref-miffre_momentum_2007}{}%
Miffre, J., Rallis, G., 2007. Momentum strategies in commodity futures markets. Journal of Banking \& Finance 31, 1863--1886.

\leavevmode\hypertarget{ref-moskowitz_time_2012}{}%
Moskowitz, T.J., Ooi, Y.H., Pedersen, L.H., 2012. Time series momentum. Journal of Financial Economics, Special issue on investor sentiment 104, 228--250.

\leavevmode\hypertarget{ref-sakkas_factor_2020}{}%
Sakkas, A., Tessaromatis, N., 2020. Factor based commodity investing. Journal of Banking \& Finance 115, 105807.

\leavevmode\hypertarget{ref-sanders_impact_2011}{}%
Sanders, D.R., Irwin, S.H., 2011a. The impact of index funds in commodity futures markets: A systems approach. The Journal of Alternative Investments 14, 40--49.

\leavevmode\hypertarget{ref-sanders_new_2011}{}%
Sanders, D.R., Irwin, S.H., 2011b. New evidence on the impact of index funds in u.s. Grain futures markets. Canadian Journal of Agricultural Economics/Revue canadienne d'agroeconomie 59, 519--532.

\leavevmode\hypertarget{ref-sanders_adequacy_2010}{}%
Sanders, D.R., Irwin, S.H., Merrin, R.P., 2010. The adequacy of speculation in agricultural futures markets: Too much of a good thing? Applied Economic Perspectives and Policy 32, 77--94.

\leavevmode\hypertarget{ref-schumann_hunger_2011}{}%
Schumann, H., 2011. The hunger-makers: How deutsche bank, goldman sachs and other financial institutions are speculating with food at the expense of the poorest. Foodwatch, Berlin.

\leavevmode\hypertarget{ref-schwartz_stochastic_1997}{}%
Schwartz, E.S., 1997. The stochastic behavior of commodity prices: Implications for valuation and hedging. The Journal of Finance 52, 923--973.

\leavevmode\hypertarget{ref-schwartz_short_2000}{}%
Schwartz, E., Smith, J.E., 2000. Short-term variations and long-term dynamics in commodity prices. Management Science 46, 893--911.

\leavevmode\hypertarget{ref-senate_excessive_2009}{}%
Senate, U.S., 2009. Excessive speculation in the wheat market. Majority and Minority Staff Report. Permanent Subcommittee on Investigations 24, 107--108.

\leavevmode\hypertarget{ref-simsek_speculation_2013}{}%
Simsek, A., 2013. Speculation and risk sharing with new financial assets. The Quarterly Journal of Economics 128, 1365--1396.

\leavevmode\hypertarget{ref-singleton_investor_2013}{}%
Singleton, K.J., 2013. Investor flows and the 2008 boom/bust in oil prices. Management Science 60, 300--318.

\leavevmode\hypertarget{ref-sockin_informational_2015}{}%
Sockin, M., Xiong, W., 2015. Informational frictions and commodity markets. The Journal of Finance 70, 2063--2098.

\leavevmode\hypertarget{ref-stoll_commodity_2011}{}%
Stoll, H.R., Whaley, R.E., 2011. Commodity index investing: Speculation or diversification? The Journal of Alternative Investments 14, 50--60.

\leavevmode\hypertarget{ref-stout_why_1998}{}%
Stout, L.A., 1998. Why the law hates speculators: Regulation and private ordering in the market for otc derivatives. Duke Law Journal 48, 701--786.

\leavevmode\hypertarget{ref-stulz_rethink_1996}{}%
Stulz, R.M., 1996. Rethinking risk management. Journal of Applied Corporate Finance 9, 8--25.

\leavevmode\hypertarget{ref-szymanowska_anatomy_2014}{}%
Szymanowska, M., De Roon, F., Nijman, T., Van Den Goorbergh, R., 2014. An anatomy of commodity futures risk premia. The Journal of Finance 69, 453--482.

\leavevmode\hypertarget{ref-tang_index_2012}{}%
Tang, K., Xiong, W., 2012. Index investment and the financialization of commodities. Financial Analysts Journal 68, 54--74.

\leavevmode\hypertarget{ref-telser_futures_1958}{}%
Telser, L.G., 1958. Futures trading and the storage of cotton and wheat. Journal of Political Economy 66, 233--255.

\leavevmode\hypertarget{ref-till_long_2007}{}%
Till, H., 2007. A long-term perspective on commodity futures returns, in: Intelligent Commodity Investing. Risk books, London.

\leavevmode\hypertarget{ref-tokic_speculation_2012}{}%
Tokic, D., 2012. Speculation and the 2008 oil bubble: The dcot report analysis. Energy Policy 45, 541--550.

\leavevmode\hypertarget{ref-turnovsky_determination_1983}{}%
Turnovsky, S.J., 1983. The determination of spot and futures prices with storable commodities. Econometrica 51, 1363--1387.

\leavevmode\hypertarget{ref-unctad_global_2009}{}%
UNCTAD, S.T.F. on S.I., Cooperation, E., 2009. The global economic crisis: Systemic failures and multilateral remedies. UNCTAD, Geneva.

\leavevmode\hypertarget{ref-weymar_supply_1966}{}%
Weymar, F.H., 1966. The supply of storage revisited. The American Economic Review 56, 1226--1234.

\leavevmode\hypertarget{ref-working_price_1933}{}%
Working, H., 1933. Price relations between july and september wheat futures at chicago since 1885. Wheat Studies 9, 187.

\leavevmode\hypertarget{ref-working_theory_1948}{}%
Working, H., 1948. Theory of the inverse carrying charge in futures markets. Journal of Farm Economics 30, 1--28.

\leavevmode\hypertarget{ref-working_theory_1949}{}%
Working, H., 1949. The theory of price of storage. The American Economic Review 1254--1262.

\leavevmode\hypertarget{ref-working_hedging_1953}{}%
Working, H., 1953. Hedging reconsidered. Journal of Farm Economics 35, 544--561.

\leavevmode\hypertarget{ref-yang_investment_2013}{}%
Yang, F., 2013. Investment shocks and the commodity basis spread. Journal of Financial Economics 110, 164--184.


\end{document}

